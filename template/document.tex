%use scrbook and set dotted line
\documentclass[toc=chapterentrywithdots,openany]{scrbook}
%use a4paper for a4 paper and size 
\usepackage[a4paper,left=30mm,right=30mm,top=30mm, bottom=30mm]{geometry}
%use german
\usepackage{german}
%silence a not relevant warning
%https://tex.stackexchange.com/questions/349473/suppress-warning-usage-of-package-fancyhdr-togetherscrbook-with-a-koma-scr
\usepackage{silence}
\WarningFilter{scrbook}{Usage of package `fancyhdr'}

%use colorpackage to set font colors
\usepackage{xcolor}
%use fancyhdr to set the header
\usepackage{fancyhdr}
%use helvetica font
\usepackage[]{helvet}
%use tocbibind to see the toc in the toc
\usepackage{tocbibind}
%user hyperref to add links
\usepackage{hyperref}
%use graphicx to add images
\usepackage{graphicx}
%path for the images
\graphicspath{ {figures/} }
%renewcommand to apply the font
\renewcommand\familydefault{\sfdefault}

%use graphicx to add images
\usepackage{graphicx}

%use inputenc to set the encoding
\usepackage[utf8]{inputenc}


%start the document
\begin{document}

%titlepage
%variableto fill
\newcommand{\Thema}{Mein Thema}
\newcommand{\Name}{Vorname nachname}
\newcommand{\Gutachter}{Dr. Herbert Bauer}
\newcommand{\Matrikelnummer}{123456}
\newcommand{\Abgabedatum}{09.01.2022}




\begin{titlepage}
	\newgeometry{left=2cm, right=2cm, top=2cm, bottom=2cm}
	%center the logo and display the FOM graphic
	\begin{center}
		\includegraphics[width=2.3cm]{assets/fomLogo.pdf}\\
		\vspace{.5cm}
		%strong text
		\begin{Large}\textbf{FOM Hochschule für Ökonomie und Management}\end{Large}\\
		\vspace{.5cm}
		Hochschulzentrum München

		\vspace{2cm}
	\end{center}

	%create big space
	\bigskip

	\begin{center}
		%bold text
		\textbf{Seminararbeit}\\
		\vspace{0.2cm}
		Im Rahmen des Moduls\\
		\vspace{0.5cm}
		Arbeitsmethoden und Softwareunterstützung\\
		\vspace{2cm}
		Über das Thema\\
		\vspace{0.5cm}
		%bold text
		\begin{Large}\textbf{\textbf{\Thema}}\end{Large}\\

		\vspace{2cm}
		von\\
		\vspace{0.5cm}
		\begin{Large}\textbf{\textbf{\Name}}\end{Large}\\
	\end{center}

	%postion in bottom left
	\begin{figure}[b]

		Gutachter: \Gutachter       \\
		Matrikelnummer: \Matrikelnummer \\
		Abgabedatum: \Abgabedatum
	\end{figure}

\end{titlepage}

%pagenumbering
\fancypagestyle{plain}{%
	\fancyhf{} % clear all header and footer fields
	\fancyhead{} % clear all header fields
	\fancyfoot{} % clear all footer fields
	\fancyhead[C]{\textcolor{gray}\thepage} 
	\renewcommand{\headrulewidth}{0pt}
	\renewcommand{\footrulewidth}{0pt}
}
%use roman for the page number
\pagenumbering{Roman}
%set the toc to 2
\setcounter{page}{2}


%inhaltsverzeichnis
%pagenumbering
\fancypagestyle{plain}{%
	\fancyhf{} % clear all header and footer fields
	\fancyhead{} % clear all header fields
	\fancyfoot{} % clear all footer fields
	\fancyhead[C]{\textcolor{gray}\thepage} 
	\renewcommand{\headrulewidth}{0pt}
	\renewcommand{\footrulewidth}{0pt}
}
%use roman for the page number
\pagenumbering{Roman}
%set the toc to 2
\setcounter{page}{2}
%abbildungen
\input{abbildungsverzeichnis}
\input{tabellenverzeichnis}
%tabellen
%symbole und formeln
\input{formel-symbolverzeichnis}

\newpage

%reset pagestyle to continue with arabic numbering
\pagestyle{plain}
%pagenumbering
\fancypagestyle{plain}{%
	\fancyhf{} % clear all header and footer fields
	\fancyhead{} % clear all header fields
	\fancyfoot{} % clear all footer fields
	\fancyhead[C]{\textcolor{gray}\thepage}
	\renewcommand{\headrulewidth}{0pt}
	\renewcommand{\footrulewidth}{0pt}
}
\pagenumbering{arabic}

\chapter{Einleitung der Arbeit}

\section{Hintergrund und Ausgangssituation}



\section{Forschungsziel, Forschungsfrage und These}



\section{Aufbau der Arbeit}


\newpage
\chapter{Hauptteil}

\section{Forschungsgegenstand im Detail}
\subsection{Beschreibung des Standes der Technik beim Autonomen Fahren}

Lorem \ac{GCD} \ac{LCM}ipsum dolor sit amet, consetetur sadipscing elitr, sed diam nonumy eirmod tempor invidunt ut labore et dolore magna aliquyam erat, sed diam voluptua.
At vero eos et accusam et justo duo dolores et ea rebum.
Lorem ipsum dolor sit amet, consetetur sadipscing elitr, sed diam nonumy eirmod tempor invidunt ut labore et dolore magna aliquyam erat, sed diam voluptua.
At vero eos et accusam et justo duo dolores et ea rebum.
Stet clita kasd gubergren, no sea takimata sanctus est Lorem ipsum dolor sit amet.
Automation in cars has a long history.  Lorem ipsum dolor sit amet, consetetur sadipscing elitr, sed diam nonumy eirmod tempor invidunt ut labore et dolore magna aliquyam erat, sed diam voluptua.
Stet clita kasd gubergren, no sea takimata sanctus est Lorem ipsum dolor sit amet.
Automation in cars has a long history.  Lorem ipsum dolor sit amet, consetetur sadipscing elitr, sed diam nonumy eirmod tempor invidunt ut labore et dolore magna aliquyam erat, sed diam voluptua.
At vero eos et accusam et justo duo dolores et ea rebum.
Stet clita kasd gubergren, no sea takimata sanctus est Lorem ipsum dolor sit amet Acceptance of autonomous driving will depend on how far a consensus on these norms can be found, first among experts, then in society at large.
One ethical condition, however, should be crucial: in no case should the ethical algorithms be put in practice as nontransparent black boxes.
The built-in norms should, as far as possible, be understood and commonly shared.

\subsection{Größte Unfallrisiken und deren Vermeidung}

Lorem ipsum dolor sit amet, consetetur sadipscing elitr, sed diam nonumy eirmod tempor invidunt ut labore et dolore magna aliquyam erat, sed diam voluptua.
At vero eos et accusam et justo duo dolores et ea rebum.
Stet clita kasd gubergren, no sea takimata sanctus est Lorem ipsum dolor sit amet.
Automation in cars has a long history.  Lorem ipsum dolor sit amet, consetetur sadipscing elitr, sed diam nonumy eirmod tempor invidunt ut labore et dolore magna aliquyam erat, sed diam voluptua.
At vero eos et accusam et justo duo dolores et ea rebum.
Stet clita kasd gubergren, no sea takimata sanctus est Lorem ipsum dolor sit amet Acceptance of autonomous driving will depend on how far a consensus on these norms can be found, first among experts, then in society at large.
One ethical condition, however, should be crucial: in no case should the ethical algorithms be put in practice as nontransparent black boxes.
The built-in norms should, as far as possible, be understood and commonly shared.


\subsection{Ethische Problemsituationen}

Lorem ipsum dolor sit amet, consetetur sadipscing elitr, sed diam nonumy eirmod tempor invidunt ut labore et dolore magna aliquyam erat, sed diam voluptua.
At vero eos et accusam et justo duo dolores et ea rebum.
Stet clita kasd gubergren, no sea takimata sanctus est Lorem ipsum dolor sit amet.
Automation in cars has a long history.  Lorem ipsum dolor sit amet, consetetur sadipscing elitr, sed diam nonumy eirmod tempor invidunt ut labore et dolore magna aliquyam erat, sed diam voluptua.
At vero eos et accusam et justo duo dolores et ea rebum.
Stet clita kasd gubergren, no sea takimata sanctus est Lorem ipsum dolor sit amet Acceptance of autonomous driving will depend on how far a consensus on these norms can be found, first among experts, then in society at large.
One ethical condition, however, should be crucial: in no case should the ethical algorithms be put in practice as nontransparent black boxes.
The built-in norms should, as far as possible, be understood and commonly shared.

\section{These im Detail}

Lorem ipsum dolor sit amet, consetetur sadipscing elitr, sed diam nonumy eirmod tempor invidunt ut labore et dolore magna aliquyam erat, sed diam voluptua.
At vero eos et accusam et justo duo dolores et ea rebum.
Stet clita kasd gubergren, no sea takimata sanctus est Lorem ipsum dolor sit amet.
Automation in cars has a long history.  Lorem ipsum dolor sit amet, consetetur sadipscing elitr, sed diam nonumy eirmod tempor invidunt ut labore et dolore magna aliquyam erat, sed diam voluptua.
At vero eos et accusam et justo duo dolores et ea rebum.


\begin{figure}
	\caption[Architektur]{Architektur}
	\centering
	\includegraphics[]{assets/figures/Architektur.jpg}
	\begin{flushleft}
		Quelle: Martínez-Díaz, M./Soriguera, F./Pérez, I., Autonomous driving: a bird's eye view, 2019, S. 564.
	\end{flushleft}
\end{figure}

Stet clita kasd gubergren, no sea takimata sanctus est Lorem ipsum dolor sit amet Acceptance of autonomous driving will depend on how far a consensus on these norms can be found, first among experts, then in society at large.
One ethical condition, however, should be crucial: in no case should the ethical algorithms be put in practice as nontransparent black boxes.
The built-in norms should, as far as possible, be understood and commonly shared.

\section{Untersuchungsmethode}

Lorem ipsum dolor sit amet, consetetur sadipscing elitr, sed diam nonumy eirmod tempor invidunt ut labore et dolore magna aliquyam erat, sed diam voluptua.
At vero eos et accusam et justo duo dolores et ea rebum.
Stet clita kasd gubergren, no sea takimata sanctus est Lorem ipsum dolor sit amet.
Automation in cars has a long history.  Lorem ipsum dolor sit amet, consetetur sadipscing elitr, sed diam nonumy eirmod tempor invidunt ut labore et dolore magna aliquyam erat, sed diam voluptua.
At vero eos et accusam et justo duo dolores et ea rebum.\footnote{Vgl. Baumann, M. F. u. a., Taking responsibility: A responsible research and innovation (RRI) perspective on insurance issues of semi-autonomous driving, 2019, S. 558.}
Stet clita kasd gubergren, no sea takimata sanctus est Lorem ipsum dolor sit amet Acceptance of autonomous driving will depend on how far a consensus on these norms can be found, first among experts, then in society at large.
One ethical condition, however, should be crucial: in no case should the ethical algorithms be put in practice as nontransparent black boxes.
The built-in norms should, as far as possible, be understood and commonly shared.


\section{Literaturanalyse}

\begin{table}[h]
	\caption{Tolle Tabelle}
	\centering
	\begin{tabular}{ | c | c | c | }
		\hline
		cell1 & cell2 & cell3 \\
		cell4 & cell5 & cell6 \\
		cell7 & cell8 & cell9 \\
		\hline
	\end{tabular}
	\begin{flushleft}
		Quelle: Martínez-Díaz, M./Soriguera, F./Pérez, I., Autonomous driving: a bird's eye view, 2019, S. 564.
	\end{flushleft}
\end{table}

Lorem ipsum dolor sit amet, consetetur sadipscing elitr, sed diam nonumy eirmod tempor invidunt ut labore et dolore magna aliquyam erat, sed diam voluptua.
At vero eos et accusam et justo duo dolores et ea rebum.
Stet clita kasd gubergren, no sea takimata sanctus est Lorem ipsum dolor sit amet.
Automation in cars has a long history.  Lorem ipsum dolor sit amet, consetetur sadipscing elitr, sed diam nonumy eirmod tempor invidunt ut labore et dolore magna aliquyam erat, sed diam voluptua.
At vero eos et accusam et justo duo dolores et ea rebum.\footnote{Vgl. Baumann, M. F. u. a., Taking responsibility: A responsible research and innovation (RRI) perspective on insurance issues of semi-autonomous driving, 2019, S. 558.}
Stet clita kasd gubergren, no sea takimata sanctus est Lorem ipsum dolor sit amet Acceptance of autonomous driving will depend on how far a consensus on these norms can be found, first among experts, then in society at large.
One ethical condition, however, should be crucial: in no case should the ethical algorithms be put in practice as nontransparent black boxes.
The built-in norms should, as far as possible, be understood and commonly shared.

\section{Diskussion}

Lorem ipsum dolor sit amet, consetetur sadipscing elitr, sed diam nonumy eirmod tempor invidunt ut labore et dolore magna aliquyam erat, sed diam voluptua.
At vero eos et accusam et justo duo dolores et ea rebum.\footnote{Vgl. Baumann, M. F. u. a., Taking responsibility: A responsible research and innovation (RRI) perspective on insurance issues of semi-autonomous driving, 2019, S. 558.}
Stet clita kasd gubergren, no sea takimata sanctus est Lorem ipsum dolor sit amet.
Automation in cars has a long history.  Lorem ipsum dolor sit amet, consetetur sadipscing elitr, sed diam nonumy eirmod tempor invidunt ut labore et dolore magna aliquyam erat, sed diam voluptua.
At vero eos et accusam et justo duo dolores et ea rebum.
Stet clita kasd gubergren, no sea takimata sanctus est Lorem ipsum dolor sit amet Acceptance of autonomous driving will depend on how far a consensus on these norms can be found, first among experts, then in society at large.
One ethical condition, however, should be crucial: in no case should the ethical algorithms be put in practice as nontransparent black boxes.
The built-in norms should, as far as possible, be understood and commonly shared.

\subsection{Diskussion der These an Hand der Literatur}

Lorem ipsum dolor sit amet, consetetur sadipscing elitr, sed diam nonumy eirmod tempor invidunt ut labore et dolore magna aliquyam erat, sed diam voluptua.
At vero eos et accusam et justo duo dolores et ea rebum.\footnote{Vgl. Baumann, M. F. u. a., Taking responsibility: A responsible research and innovation (RRI) perspective on insurance issues of semi-autonomous driving, 2019, S. 558.}
Stet clita kasd gubergren, no sea takimata sanctus est Lorem ipsum dolor sit amet.
Automation in cars has a long history.  Lorem ipsum dolor sit amet, consetetur sadipscing elitr, sed diam nonumy eirmod tempor invidunt ut labore et dolore magna aliquyam erat, sed diam voluptua.
At vero eos et accusam et justo duo dolores et ea rebum.
Stet clita kasd gubergren, no sea takimata sanctus est Lorem ipsum dolor sit amet Acceptance of autonomous driving will depend on how far a consensus on these norms can be found, first among experts, then in society at large.
One ethical condition, however, should be crucial: in no case should the ethical algorithms be put in practice as nontransparent black boxes.
The built-in norms should, as far as possible, be understood and commonly shared.

\subsection{Qualität der Aussage}

Lorem ipsum dolor sit amet, consetetur sadipscing elitr, sed diam nonumy eirmod tempor invidunt ut labore et dolore magna aliquyam erat, sed diam voluptua.
At vero eos et accusam et justo duo dolores et ea rebum.
Stet clita kasd gubergren, no sea takimata sanctus est Lorem ipsum dolor sit amet.
Automation in cars has a long history.  Lorem ipsum dolor sit amet, consetetur sadipscing elitr, sed diam nonumy eirmod tempor invidunt ut labore et dolore magna aliquyam erat, sed diam voluptua.
At vero eos et accusam et justo duo dolores et ea rebum.\footnote{Vgl. Baumann, M. F. u. a., Taking responsibility: A responsible research and innovation (RRI) perspective on insurance issues of semi-autonomous driving, 2019, S. 558.}
Stet clita kasd gubergren, no sea takimata sanctus est Lorem ipsum dolor sit amet Acceptance of autonomous driving will depend on how far a consensus on these norms can be found, first among experts, then in society at large.
One ethical condition, however, should be crucial: in no case should the ethical algorithms be put in practice as nontransparent black boxes.
The built-in norms should, as far as possible, be understood and commonly shared.

\section{Fazit – Konsequnzen}

Lorem ipsum dolor sit amet, consetetur sadipscing elitr\footnote{Vgl. Baumann, M. F. u. a., Taking responsibility: A responsible research and innovation (RRI) perspective on insurance issues of semi-autonomous driving, 2019, S. 558.}, sed diam nonumy eirmod tempor invidunt ut labore et dolore magna aliquyam erat, sed diam voluptua.
At vero eos et accusam et justo duo dolores et ea rebum.
Stet clita kasd gubergren, no sea takimata sanctus est Lorem ipsum dolor sit amet.
Automation in cars has a long history.  Lorem ipsum dolor sit amet, consetetur sadipscing elitr, sed diam nonumy eirmod tempor invidunt ut labore et dolore magna aliquyam erat, sed diam voluptua.
At vero eos et accusam et justo duo dolores et ea rebum.
Stet clita kasd gubergren, no sea takimata sanctus est Lorem ipsum dolor sit amet Acceptance of autonomous driving will depend on how far a consensus on these norms can be found, first among experts, then in society at large.
One ethical condition, however, should be crucial: in no case should the ethical algorithms be put in practice as nontransparent black boxes.
The built-in norms should, as far as possible, be understood and commonly shared.

\chapter{Schluss}
Warum sollten man autonom Fahren?


\section{Kurzzusammenfassung der Arbeit}




%end of document
\end{document}