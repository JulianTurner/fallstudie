%use scrbook and set dotted line
\documentclass[toc=chapterentrywithdots,openany]{scrbook}


%stores all imports
%use a4paper for a4 paper and size 
\usepackage[a4paper,left=30mm,right=30mm,top=30mm, bottom=30mm]{geometry}
%use german
\usepackage{german}
%silence a not relevant warning
%https://tex.stackexchange.com/questions/349473/suppress-warning-usage-of-package-fancyhdr-togetherscrbook-with-a-koma-scr
\usepackage{silence}
\WarningFilter{scrbook}{Usage of package `fancyhdr'}
%use catption for setup
\usepackage{caption}
%add prefix to the figure name
\usepackage{titletoc}
%use colorpackage to set font colors
\usepackage{xcolor}
%use fancyhdr to set the header
\usepackage{fancyhdr}
%use helvetica font
\usepackage[]{helvet}
%user hyperref to add links
\usepackage{hyperref}
%use graphicx to add images
\usepackage{graphicx}
%use graphicx to add images
\usepackage{graphicx}
%use inputenc to set the encoding
\usepackage[utf8]{inputenc}
%prevent widows and orphans
\usepackage[all]{nowidow}
%create acronym
\usepackage{acronym}
%change font
\usepackage{lmodern}

%set the caption to the left side
\captionsetup{justification=raggedright,singlelinecheck=false}

%options for figures
%reset the counter to have no chapter numberings
\setcounter{figure}{0}
\renewcommand{\thefigure}{\arabic{figure}}

%reset the figure name
\renewcommand{\figurename}{Bild}

%format the figure appereance in tof
\titlecontents{figure}[0mm]%
{Bild }%
{\thecontentslabel\quad: ~}%
{}%
{\enspace\dotfill\enspace\thecontentspage}


%options for tables
%%reset the counter
\setcounter{table}{0}
%set page numbering to arabic
\renewcommand{\thetable}{\arabic{table}}

%format the apperence of the table
\titlecontents{table}[0mm]%
{Tabelle }%
{\thecontentslabel\quad: ~}%
{}%
{\enspace\dotfill\enspace\thecontentspage}
\usepackage[acronym]{glossaries}
\makeglossaries
\section*{\fontsize{20}{0} \selectfont Abkürzungsverzeichnis}
\begin{acronym}[]\itemsep0pt %der Parameter in Klammern sollte die längste Abkürzung sein. Damit wird der Abstand zwischen Abkürzung und Übersetzung festgelegt
	\acro{CO2}{Kohlenstoffdioxid}
	\acro{Kfz}{Kraftfahrzeug}
	\acro{Nfz}{Nutzfahrzeug}
	\acro{Pkw}{Personenkraftwagen}
	\acro{SAE}{Society of Automotive Engineers}



\end{acronym}
\addcontentsline{toc}{chapter}{Abkürzungsverzeichnis}

%renewcommand to apply the font
\renewcommand\familydefault{\sfdefault}
\setuptoc{toc}{totoc}
\newcommand{\paragraphtitle}[1]{\paragraph{#1}\mbox{}\\}
%start the document
\setcounter{tocdepth}{3} % seting level of numbering (default for "report" is 3). With ''-1'' you have non number also for chapters
\setcounter{secnumdepth}{3} % seting level of numbering (default for "report" is 3). With ''-1'' you have non number also for chapters
%\setcounter{tocdepth}{5} % if you want all the levels in your table of contents
\begin{document}

%titlepage
%variableto fill
\newcommand{\Thema}{Mein Thema}
\newcommand{\Name}{Vorname nachname}
\newcommand{\Gutachter}{Dr. Herbert Bauer}
\newcommand{\Matrikelnummer}{123456}
\newcommand{\Abgabedatum}{09.01.2022}




\begin{titlepage}
	\newgeometry{left=2cm, right=2cm, top=2cm, bottom=2cm}
	%center the logo and display the FOM graphic
	\begin{center}
		\includegraphics[width=2.3cm]{assets/fomLogo.pdf}\\
		\vspace{.5cm}
		%strong text
		\begin{Large}\textbf{FOM Hochschule für Ökonomie und Management}\end{Large}\\
		\vspace{.5cm}
		Hochschulzentrum München

		\vspace{2cm}
	\end{center}

	%create big space
	\bigskip

	\begin{center}
		%bold text
		\textbf{Seminararbeit}\\
		\vspace{0.2cm}
		Im Rahmen des Moduls\\
		\vspace{0.5cm}
		Arbeitsmethoden und Softwareunterstützung\\
		\vspace{2cm}
		Über das Thema\\
		\vspace{0.5cm}
		%bold text
		\begin{Large}\textbf{\textbf{\Thema}}\end{Large}\\

		\vspace{2cm}
		von\\
		\vspace{0.5cm}
		\begin{Large}\textbf{\textbf{\Name}}\end{Large}\\
	\end{center}

	%postion in bottom left
	\begin{figure}[b]

		Gutachter: \Gutachter       \\
		Matrikelnummer: \Matrikelnummer \\
		Abgabedatum: \Abgabedatum
	\end{figure}

\end{titlepage}
%inhaltsverzeichnis
%pagenumbering
\fancypagestyle{plain}{%
	\fancyhf{} % clear all header and footer fields
	\fancyhead{} % clear all header fields
	\fancyfoot{} % clear all footer fields
	\fancyhead[C]{\textcolor{gray}\thepage} 
	\renewcommand{\headrulewidth}{0pt}
	\renewcommand{\footrulewidth}{0pt}
}
%use roman for the page number
\pagenumbering{Roman}
%set the toc to 2
\setcounter{page}{2}


%Inhaltsverzeichnis
\tableofcontents

%abbildungen & tabellen
\listoffigures
\addcontentsline{toc}{chapter}{\protect\listfigurename}


%\listoftables
%\addcontentsline{toc}{chapter}{\protect\listtablename}
\printglossary[type=\acronymtype,title=Abkürzungsverzeichnis,nonumberlist]
\addcontentsline{toc}{chapter}{Abkürzungsverzeichnis}

\newpage

%reset pagestyle to continue with arabic numbering
\pagestyle{plain}
%pagenumbering
\fancypagestyle{plain}{%
	\fancyhf{} % clear all header and footer fields
	\fancyhead{} % clear all header fields
	\fancyfoot{} % clear all footer fields
	\fancyhead[C]{\textcolor{gray}\thepage}
	\renewcommand{\headrulewidth}{0pt}
	\renewcommand{\footrulewidth}{0pt}
}
\pagenumbering{arabic}

%start
\chapter{Einleitung der Arbeit}

\section{Hintergrund und Ausgangssituation}



\section{Forschungsziel, Forschungsfrage und These}



\section{Aufbau der Arbeit}


\newpage
\chapter{Hauptteil}

\section{Forschungsgegenstand im Detail}
\subsection{Beschreibung des Standes der Technik beim Autonomen Fahren}

Lorem \ac{GCD} \ac{LCM}ipsum dolor sit amet, consetetur sadipscing elitr, sed diam nonumy eirmod tempor invidunt ut labore et dolore magna aliquyam erat, sed diam voluptua.
At vero eos et accusam et justo duo dolores et ea rebum.
Lorem ipsum dolor sit amet, consetetur sadipscing elitr, sed diam nonumy eirmod tempor invidunt ut labore et dolore magna aliquyam erat, sed diam voluptua.
At vero eos et accusam et justo duo dolores et ea rebum.
Stet clita kasd gubergren, no sea takimata sanctus est Lorem ipsum dolor sit amet.
Automation in cars has a long history.  Lorem ipsum dolor sit amet, consetetur sadipscing elitr, sed diam nonumy eirmod tempor invidunt ut labore et dolore magna aliquyam erat, sed diam voluptua.
Stet clita kasd gubergren, no sea takimata sanctus est Lorem ipsum dolor sit amet.
Automation in cars has a long history.  Lorem ipsum dolor sit amet, consetetur sadipscing elitr, sed diam nonumy eirmod tempor invidunt ut labore et dolore magna aliquyam erat, sed diam voluptua.
At vero eos et accusam et justo duo dolores et ea rebum.
Stet clita kasd gubergren, no sea takimata sanctus est Lorem ipsum dolor sit amet Acceptance of autonomous driving will depend on how far a consensus on these norms can be found, first among experts, then in society at large.
One ethical condition, however, should be crucial: in no case should the ethical algorithms be put in practice as nontransparent black boxes.
The built-in norms should, as far as possible, be understood and commonly shared.

\subsection{Größte Unfallrisiken und deren Vermeidung}

Lorem ipsum dolor sit amet, consetetur sadipscing elitr, sed diam nonumy eirmod tempor invidunt ut labore et dolore magna aliquyam erat, sed diam voluptua.
At vero eos et accusam et justo duo dolores et ea rebum.
Stet clita kasd gubergren, no sea takimata sanctus est Lorem ipsum dolor sit amet.
Automation in cars has a long history.  Lorem ipsum dolor sit amet, consetetur sadipscing elitr, sed diam nonumy eirmod tempor invidunt ut labore et dolore magna aliquyam erat, sed diam voluptua.
At vero eos et accusam et justo duo dolores et ea rebum.
Stet clita kasd gubergren, no sea takimata sanctus est Lorem ipsum dolor sit amet Acceptance of autonomous driving will depend on how far a consensus on these norms can be found, first among experts, then in society at large.
One ethical condition, however, should be crucial: in no case should the ethical algorithms be put in practice as nontransparent black boxes.
The built-in norms should, as far as possible, be understood and commonly shared.


\subsection{Ethische Problemsituationen}

Lorem ipsum dolor sit amet, consetetur sadipscing elitr, sed diam nonumy eirmod tempor invidunt ut labore et dolore magna aliquyam erat, sed diam voluptua.
At vero eos et accusam et justo duo dolores et ea rebum.
Stet clita kasd gubergren, no sea takimata sanctus est Lorem ipsum dolor sit amet.
Automation in cars has a long history.  Lorem ipsum dolor sit amet, consetetur sadipscing elitr, sed diam nonumy eirmod tempor invidunt ut labore et dolore magna aliquyam erat, sed diam voluptua.
At vero eos et accusam et justo duo dolores et ea rebum.
Stet clita kasd gubergren, no sea takimata sanctus est Lorem ipsum dolor sit amet Acceptance of autonomous driving will depend on how far a consensus on these norms can be found, first among experts, then in society at large.
One ethical condition, however, should be crucial: in no case should the ethical algorithms be put in practice as nontransparent black boxes.
The built-in norms should, as far as possible, be understood and commonly shared.

\section{These im Detail}

Lorem ipsum dolor sit amet, consetetur sadipscing elitr, sed diam nonumy eirmod tempor invidunt ut labore et dolore magna aliquyam erat, sed diam voluptua.
At vero eos et accusam et justo duo dolores et ea rebum.
Stet clita kasd gubergren, no sea takimata sanctus est Lorem ipsum dolor sit amet.
Automation in cars has a long history.  Lorem ipsum dolor sit amet, consetetur sadipscing elitr, sed diam nonumy eirmod tempor invidunt ut labore et dolore magna aliquyam erat, sed diam voluptua.
At vero eos et accusam et justo duo dolores et ea rebum.


\begin{figure}
	\caption[Architektur]{Architektur}
	\centering
	\includegraphics[]{assets/figures/Architektur.jpg}
	\begin{flushleft}
		Quelle: Martínez-Díaz, M./Soriguera, F./Pérez, I., Autonomous driving: a bird's eye view, 2019, S. 564.
	\end{flushleft}
\end{figure}

Stet clita kasd gubergren, no sea takimata sanctus est Lorem ipsum dolor sit amet Acceptance of autonomous driving will depend on how far a consensus on these norms can be found, first among experts, then in society at large.
One ethical condition, however, should be crucial: in no case should the ethical algorithms be put in practice as nontransparent black boxes.
The built-in norms should, as far as possible, be understood and commonly shared.

\section{Untersuchungsmethode}

Lorem ipsum dolor sit amet, consetetur sadipscing elitr, sed diam nonumy eirmod tempor invidunt ut labore et dolore magna aliquyam erat, sed diam voluptua.
At vero eos et accusam et justo duo dolores et ea rebum.
Stet clita kasd gubergren, no sea takimata sanctus est Lorem ipsum dolor sit amet.
Automation in cars has a long history.  Lorem ipsum dolor sit amet, consetetur sadipscing elitr, sed diam nonumy eirmod tempor invidunt ut labore et dolore magna aliquyam erat, sed diam voluptua.
At vero eos et accusam et justo duo dolores et ea rebum.\footnote{Vgl. Baumann, M. F. u. a., Taking responsibility: A responsible research and innovation (RRI) perspective on insurance issues of semi-autonomous driving, 2019, S. 558.}
Stet clita kasd gubergren, no sea takimata sanctus est Lorem ipsum dolor sit amet Acceptance of autonomous driving will depend on how far a consensus on these norms can be found, first among experts, then in society at large.
One ethical condition, however, should be crucial: in no case should the ethical algorithms be put in practice as nontransparent black boxes.
The built-in norms should, as far as possible, be understood and commonly shared.


\section{Literaturanalyse}

\begin{table}[h]
	\caption{Tolle Tabelle}
	\centering
	\begin{tabular}{ | c | c | c | }
		\hline
		cell1 & cell2 & cell3 \\
		cell4 & cell5 & cell6 \\
		cell7 & cell8 & cell9 \\
		\hline
	\end{tabular}
	\begin{flushleft}
		Quelle: Martínez-Díaz, M./Soriguera, F./Pérez, I., Autonomous driving: a bird's eye view, 2019, S. 564.
	\end{flushleft}
\end{table}

Lorem ipsum dolor sit amet, consetetur sadipscing elitr, sed diam nonumy eirmod tempor invidunt ut labore et dolore magna aliquyam erat, sed diam voluptua.
At vero eos et accusam et justo duo dolores et ea rebum.
Stet clita kasd gubergren, no sea takimata sanctus est Lorem ipsum dolor sit amet.
Automation in cars has a long history.  Lorem ipsum dolor sit amet, consetetur sadipscing elitr, sed diam nonumy eirmod tempor invidunt ut labore et dolore magna aliquyam erat, sed diam voluptua.
At vero eos et accusam et justo duo dolores et ea rebum.\footnote{Vgl. Baumann, M. F. u. a., Taking responsibility: A responsible research and innovation (RRI) perspective on insurance issues of semi-autonomous driving, 2019, S. 558.}
Stet clita kasd gubergren, no sea takimata sanctus est Lorem ipsum dolor sit amet Acceptance of autonomous driving will depend on how far a consensus on these norms can be found, first among experts, then in society at large.
One ethical condition, however, should be crucial: in no case should the ethical algorithms be put in practice as nontransparent black boxes.
The built-in norms should, as far as possible, be understood and commonly shared.

\section{Diskussion}

Lorem ipsum dolor sit amet, consetetur sadipscing elitr, sed diam nonumy eirmod tempor invidunt ut labore et dolore magna aliquyam erat, sed diam voluptua.
At vero eos et accusam et justo duo dolores et ea rebum.\footnote{Vgl. Baumann, M. F. u. a., Taking responsibility: A responsible research and innovation (RRI) perspective on insurance issues of semi-autonomous driving, 2019, S. 558.}
Stet clita kasd gubergren, no sea takimata sanctus est Lorem ipsum dolor sit amet.
Automation in cars has a long history.  Lorem ipsum dolor sit amet, consetetur sadipscing elitr, sed diam nonumy eirmod tempor invidunt ut labore et dolore magna aliquyam erat, sed diam voluptua.
At vero eos et accusam et justo duo dolores et ea rebum.
Stet clita kasd gubergren, no sea takimata sanctus est Lorem ipsum dolor sit amet Acceptance of autonomous driving will depend on how far a consensus on these norms can be found, first among experts, then in society at large.
One ethical condition, however, should be crucial: in no case should the ethical algorithms be put in practice as nontransparent black boxes.
The built-in norms should, as far as possible, be understood and commonly shared.

\subsection{Diskussion der These an Hand der Literatur}

Lorem ipsum dolor sit amet, consetetur sadipscing elitr, sed diam nonumy eirmod tempor invidunt ut labore et dolore magna aliquyam erat, sed diam voluptua.
At vero eos et accusam et justo duo dolores et ea rebum.\footnote{Vgl. Baumann, M. F. u. a., Taking responsibility: A responsible research and innovation (RRI) perspective on insurance issues of semi-autonomous driving, 2019, S. 558.}
Stet clita kasd gubergren, no sea takimata sanctus est Lorem ipsum dolor sit amet.
Automation in cars has a long history.  Lorem ipsum dolor sit amet, consetetur sadipscing elitr, sed diam nonumy eirmod tempor invidunt ut labore et dolore magna aliquyam erat, sed diam voluptua.
At vero eos et accusam et justo duo dolores et ea rebum.
Stet clita kasd gubergren, no sea takimata sanctus est Lorem ipsum dolor sit amet Acceptance of autonomous driving will depend on how far a consensus on these norms can be found, first among experts, then in society at large.
One ethical condition, however, should be crucial: in no case should the ethical algorithms be put in practice as nontransparent black boxes.
The built-in norms should, as far as possible, be understood and commonly shared.

\subsection{Qualität der Aussage}

Lorem ipsum dolor sit amet, consetetur sadipscing elitr, sed diam nonumy eirmod tempor invidunt ut labore et dolore magna aliquyam erat, sed diam voluptua.
At vero eos et accusam et justo duo dolores et ea rebum.
Stet clita kasd gubergren, no sea takimata sanctus est Lorem ipsum dolor sit amet.
Automation in cars has a long history.  Lorem ipsum dolor sit amet, consetetur sadipscing elitr, sed diam nonumy eirmod tempor invidunt ut labore et dolore magna aliquyam erat, sed diam voluptua.
At vero eos et accusam et justo duo dolores et ea rebum.\footnote{Vgl. Baumann, M. F. u. a., Taking responsibility: A responsible research and innovation (RRI) perspective on insurance issues of semi-autonomous driving, 2019, S. 558.}
Stet clita kasd gubergren, no sea takimata sanctus est Lorem ipsum dolor sit amet Acceptance of autonomous driving will depend on how far a consensus on these norms can be found, first among experts, then in society at large.
One ethical condition, however, should be crucial: in no case should the ethical algorithms be put in practice as nontransparent black boxes.
The built-in norms should, as far as possible, be understood and commonly shared.

\section{Fazit – Konsequnzen}

Lorem ipsum dolor sit amet, consetetur sadipscing elitr\footnote{Vgl. Baumann, M. F. u. a., Taking responsibility: A responsible research and innovation (RRI) perspective on insurance issues of semi-autonomous driving, 2019, S. 558.}, sed diam nonumy eirmod tempor invidunt ut labore et dolore magna aliquyam erat, sed diam voluptua.
At vero eos et accusam et justo duo dolores et ea rebum.
Stet clita kasd gubergren, no sea takimata sanctus est Lorem ipsum dolor sit amet.
Automation in cars has a long history.  Lorem ipsum dolor sit amet, consetetur sadipscing elitr, sed diam nonumy eirmod tempor invidunt ut labore et dolore magna aliquyam erat, sed diam voluptua.
At vero eos et accusam et justo duo dolores et ea rebum.
Stet clita kasd gubergren, no sea takimata sanctus est Lorem ipsum dolor sit amet Acceptance of autonomous driving will depend on how far a consensus on these norms can be found, first among experts, then in society at large.
One ethical condition, however, should be crucial: in no case should the ethical algorithms be put in practice as nontransparent black boxes.
The built-in norms should, as far as possible, be understood and commonly shared.

\chapter{Schluss}
Warum sollten man autonom Fahren?


\section{Kurzzusammenfassung der Arbeit}



\newpage

\begin{thebibliography}{9}

    %Bücher
    \bibitem{Google}
    Georg Gilder:
    \textit{Das Leben nach Google. Der Absturz von Big Data und der Aufstieg der Blockchain}. Plassen: Kulmbach. 2018.

    \bibitem{Big Data}
    Viktor Mayer-Schöneberger, Kenneth Cukier:
    \textit{Big Data. Die Revolution, die unser Leben verändern wird}. Redline: München. 2017.

    \bibitem{10xDNA}
    Frank Thelen:
    \textit{10xDna}. Goldmann: Leipzig. 2021.

    \bibitem{Token}
    Shermin Voshmgir:
    \textit{Token Economy. Wie das Web3 das Internet revolutioniert}. Token Kitchen: Luxemburg. 2020.

    %Link
    \bibitem{Hyperland}
    Douglas Adams:
    \textit{Hyperland} (1990)
    \url{https://www.youtube.com/watch?v=1iAJPoc23-M&t=17s}
    Bearbeitungsstand: 23.05.2014,
    [Zugriff 2022-01-01]

    \bibitem{Web2.0}
    \textit{Web 2.0}
    \url{https://de.wikipedia.org/wiki/Web_2.0}
    Bearbeitungsstand: 03.01.2014,
    [Zugriff 2022-01-01]

    \bibitem{NewYorkTimes StartUp}
    Daisuke Wakabayashi, Mike Isaac:
    \textit{The New Get-Rich-Faster Job in Silicon Valley: Crypto Start-Ups} (20.12.2021)
    \url{https://www.nytimes.com/2021/12/20/technology/silicon-valley-cryptocurrency-start-ups.html?searchResultPosition=4}
    Bearbeitungsstand: 22.12.2021,
    [Zugriff 2022-01-01]

    \bibitem{NewYorkTimes BlackMarket}
    Nathaniel Popper:
    \textit{Bitcoin Has Lost Steam. But Criminals Still Love It.}
    \url{https://www.nytimes.com/2020/01/28/technology/bitcoin-black-market.html}
    Bearbeitungsstand: 28.01.2020,
    [Zugriff 2022-01-01]

    \bibitem{Bitcoin}
    \textit{Cost of a Bitcoin Attack}
    \url{https://gobitcoin.io/tools/cost-51-attack/}
    Bearbeitungsstand: 01.01.2022,
    [Zugriff 2022-01-01]

    \bibitem{Solutions}
    \textit{Global Spending on Blockchain Solutions Forecast}
    \url{https://www.idc.com/getdoc.jsp?containerId=prUS47617821}
    Bearbeitungsstand: 19.04.2021,
    [Zugriff 2022-01-01]





\end{thebibliography}
\addcontentsline{toc}{chapter}{Literaturverzeichnis}
%end

%supress page numbering
\pagenumbering{gobble}
\newpage

\chapter*{Ehrenwörtliche Erklärung}

Hiermit versichere ich, dass die vorliegende Arbeit von mir selbstständig und ohne uner- laubte Hilfe angefertigt worden ist, insbesondere dass ich alle Stellen, die wörtlich oder annähernd wörtlich aus Veröffentlichungen entnommen sind, durch Zitate als solche ge- kennzeichnet habe. Ich versichere auch, dass die von mir eingereichte schriftliche Version mit der digitalen Version übereinstimmt. Weiterhin erkläre ich, dass die Arbeit in gleicher oder ähnlicher Form noch keiner Prüfungsbehörde/Prüfungsstelle vorgelegen hat. Ich er- kläre mich damit einverstanden, dass die Arbeit der Öffentlichkeit zugänglich gemacht wird. Ich erkläre mich damit einverstanden, dass die Digitalversion dieser Arbeit zwecks Plagiatsprüfung auf die Server externer Anbieter hochgeladen werden darf. Die Plagiatsprüfung stellt keine Zurverfügungstellung für die Öffentlichkeit dar.

\vspace{6cm}
\begin{tabular}{@{}p{6cm}p{6cm}@{}}
    München, den \today & \hrulefill          \\
                        & Leonardo Ciria Buil \\
\end{tabular}
\chapter*{Abstract}
Beginnend bei der Produktion, über den Betrieb bis hin zur Entsorgen erzeugen Kraftfahrzeuge
negative Auswirkungen im Umwelt.
Durch Luftschadstoffe und Feinstaub kann es zu schwerwiegenden Krankheiten und Tod kommen.


Durch neue technologische Fortschritte wie dem autonomen Fahren können
Reduktionen von Luftschadstoffen,
Einsparpotenziale beim Kraftstoffverbrauch,
schnellerer Durchsatz neuer Technologien
oder eine Reduktion des Fahrzeugbestandes mit sich bringen.


In dieser Arbeit werden Punkte wie
wie Kraftfahrzeuge aufgebaut sind und welche Teilsysteme sie beinhalten,
in welche Stufen von Assistenzsystemen unterschieden wird,
welche Umweltbelastung durch Kraftfahrzeuge entstehen,
Erwartungen an autonome Kraftfahrzeuge,
die aktuelle gesetzliche Lage in Deutschland,
Einsparpotenziale der Luftschadstoffe durch Reduktion des Kraftstoffverbrauchs auf Autobahnen so wie dem städtischen Verkehr,
Potentiale bei Flottenfahrzeugen,
positive Auswirkungen auf die Umwelt durch eine Reduktion von Kraftfahrzeugen,
untersucht.


Das Ziel dieser Arbeit ist es, den Zusammenhang zwischen
autonomen Kraftfahrzeugen und einer Verringerung von Feinstaub näher zu untersuchen und
die Hypothese zu verifizieren.

Aufgrund der hohen Aktualität, kommt diese Arbeit zu dem Ergebnis,
dass ein eindeutiger Trend noch nicht abgebildet werden,
da hierfür Daten und Langzeitstudien nicht existieren.
Vielmehr kann die Summe von mehren neuen Technologien und geänderten Verhaltensmuster eine Trendwende schaffen.
Hierunter könnte Elektromobilität, sowie der Verzicht auf ein eigenes Fahrzeug durch verschiedene
autonome Mobilitätsdienstleister eine Rolle spielen.

%end of document
\end{document}
