\chapter*{Abstract}
Beginnend bei der Produktion, über den Betrieb bis hin zur Entsorgung haben Kraftfahrzeuge
negative Auswirkungen auf die Umwelt.
Durch Luftschadstoffe und Feinstaub kann es zu schwerwiegenden Krankheiten und Tod kommen.


Durch neue technologische Fortschritte wie autonomes Fahren können
Luftschadstoffreduktionen,
Einsparpotenziale beim Kraftstoffverbrauch,
schnellerer Durchsatz neuer Technologien
oder eine Reduktion des Fahrzeugbestandes mit sich bringen.


In dieser Arbeit werden Aufbau der Kraftfahrerzeuge und dessen Teilsysteme erklärt, sowie
Stufen und Merkmale von Assistenzsystemen.
Unter anderem werden Themengebiete wie Umweltbelastung
Erwartungen an autonome Kraftfahrzeuge,
die aktuelle gesetzliche Lage in Deutschland, Reduktion von Luftschadstoffen und Kraftstoffeinsparung
untersucht.


Das Ziel dieser Arbeit ist den Zusammenhang zwischen
autonomen Kraftfahrzeugen und einer Verringerung von Feinstaub näher zu untersuchen und
die Hypothese zu verifizieren.

Aufgrund der hohen Aktualität, kommt diese Arbeit zum Ergebnis,
dass ein eindeutiger Trend noch nicht abgebildet werden kann,
da hierfür Daten und Langzeitstudien nicht existieren.
Vielmehr kann die Summe von neuen Technologien und geänderten Verhaltensmuster eine Trendwende schaffen.

Hierunter könnte Elektromobilität, sowie der Verzicht auf ein eigenes Fahrzeug durch verschiedene
autonome Mobilitätsdienstleister eine Rolle spielen.