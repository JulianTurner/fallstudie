\chapter*{Abstract}
Beginnend bei der Produktion, über den Betrieb bis hin zur Entsorgen erzeugen Kraftfahrzeuge
negative Auswirkungen im Umwelt.
Durch Luftschadstoffe und Feinstaub kann es zu schwerwiegenden Krankheiten und Tod kommen.


Durch neue technologische Fortschritte wie dem autonomen Fahren können
Reduktionen von Luftschadstoffen,
Einsparpotenziale beim Kraftstoffverbrauch,
schnellerer Durchsatz neuer Technologien
oder eine Reduktion des Fahrzeugbestandes mit sich bringen.


In dieser Arbeit werden Punkte wie
wie Kraftfahrzeuge aufgebaut sind und welche Teilsysteme sie beinhalten,
in welche Stufen von Assistenzsystemen unterschieden wird,
welche Umweltbelastung durch Kraftfahrzeuge entstehen,
Erwartungen an autonome Kraftfahrzeuge,
die aktuelle gesetzliche Lage in Deutschland,
Einsparpotenziale der Luftschadstoffe durch Reduktion des Kraftstoffverbrauchs auf Autobahnen so wie dem städtischen Verkehr,
Potentiale bei Flottenfahrzeugen,
positive Auswirkungen auf die Umwelt durch eine Reduktion von Kraftfahrzeugen,
untersucht.


Das Ziel dieser Arbeit ist es, den Zusammenhang zwischen
autonomen Kraftfahrzeugen und einer Verringerung von Feinstaub näher zu untersuchen und
die Hypothese zu verifizieren.

Aufgrund der hohen Aktualität, kommt diese Arbeit zu dem Ergebnis,
dass ein eindeutiger Trend noch nicht abgebildet werden,
da hierfür Daten und Langzeitstudien nicht existieren.
Vielmehr kann die Summe von mehren neuen Technologien und geänderten Verhaltensmuster eine Trendwende schaffen.
Hierunter könnte Elektromobilität, sowie der Verzicht auf ein eigenes Fahrzeug durch verschiedene
autonome Mobilitätsdienstleister eine Rolle spielen.