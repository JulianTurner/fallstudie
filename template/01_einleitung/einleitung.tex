\chapter{Einleitung der Arbeit}

\section{Forschungsfrage}

Welche Auswirkungen haben Kraftfahrzeuge auf die Umwelt und wie kann autonomes Fahren die negativen Auswirkungen reduzieren?

\section{Hypothese}

Je mehr Fahrzeuge autonom fahren, desto geringer fällt die Feinstaubbelastung durch Kfz aus.



\section{TBD}

-Kraftfahrzeuge einfluss auf die Umwelt
-steigende Zulassung bei Fahrzeugen durch steigende Globalisierung
-steigende Temperaturen, Feinstaubbelastung
-steigende CO2-Emissionen durch Kfz
-durch bessere Fertigungsmethoden gibt es mehr Autos
-Umweltbelastung durch Umwege
-Umweltbelastung durch Fahrfehler
-Umweltbelastung durch Kfz-Fahrer
-Umweltbelastung durch Staus
-Umweltbelastung durch LKW

\section{Definitionen}
\subsection{Kraftfahrzeuge}

\textit{Als Kraftfahrzeuge im Sinne dieses Gesetzes gelten Landfahrzeuge, die durch Maschinenkraft bewegt werden, ohne an Bahngleise gebunden zu sein.}
\footnote{§ 1 Abs. 2 Straßenverkehrsgesetz}
\newline
\newline
Kraftfahrzeuge können in folgende Kategorien eingeteilt werden:
\begin{itemize}
	\item Personenkraftwagen
	\item Nutzfahrzeuge
	\item Krafträder
	\item Anhänger
	\item Wohnmobil
	\item Militärfahrzeuge
	\item Sonderfahrzeuge
\end{itemize}
	      	      	      	      	      
	      	      	      	      	      
\subsection{Autonomes Fahren}
	      	      	      	      	      
Hier wird erklärt was Autonomes Fahren ist
Was ist autonom 
Welche Stufen gibt es
Welche Systeme gibt es
	      	      	      	      	      
	      	      	      	      	      
\subsection{Umwelteinflüsse}
	      	      	      	      	      
Hier wird erklärt was Umwelteinflüsse sind
Umwelteinflüsse wie Abgas, Lärm, Feinstaub, Licht

\subsubsection{Feinstaubbelastung}
Was ist eine Feinstaubbelastung?
Durch natürliche Art oder menschliche erzeugt
\subsubsection{Co2 Emissionen}
Was ist Co2?
Was sind CO2 Emissionen?
Natürliche CO2 Emissionen oder menschliche CO2 Emissionen
