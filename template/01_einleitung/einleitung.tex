\chapter{Einleitung der Arbeit}

\section{Hintergrund und AusgangssituatioN}

Seit einigen Jahren hat sich die Welt in das Digitale geändert. Immer mehr Unternehmen streben einen optimierten und einwandfreien Prozesszyklus an. Hierbei wird meistens zu Maschinen oder Systemen
gegriffen, welche individuell auf das Unternehmenskonzept angepasst und entwickelt werden. Die aktuelle Pandemielage zeigt auch, wie wichtig es ist, mit Computersystemen und ohne direkten Kontakt zu arbeiten.
Genau diesen Ansatz verfolgt man mit der Künstliche Intelligenz. Durch unterschiedliche Methoden wie Machine learning oder Deep learning versucht man die menschliche Intelligenz nachzuahmen. Dadurch sollen in vielen
Branchen, dass Arbeiten ohne bzw. geringem Personal möglich sein. Für Unternehmen birgt die Künstliche Intelligenz große Chancen. Auf der einen Seite soll sie Personalkosten verringern, auf der anderen Prozessketten
optimieren. Eine handvoll Mitarbeiter sollen die Maschinen verfolgen, warten und kontrollieren, den Rest verrichtet die Künstliche Intelligenz selber. Für eine 
reibungslose Zusammenarbeit zwischen Mensch und KI wird das Personal entsprechend geschult. Neuanstellungen ohne das erforderliche Know-how werden für Arbeitnehmer 
erschwert.   



\section{Forschungsziel, Forschungsfrage und These}

Mit dieser Arbeit soll eine detaillierte Untersuchung vorgenommen werden, welche die Risiken und Chancen für Arbeitnehmer und Arbeitgeber 
hinter der Künstlichen Intelligenz abwägt. Darüber hinaus soll analysiert werden, welche Auswirkungen der Einsatz von KI auf den Arbeitsmarkt hat. 

\section{Aufbau der Arbeit}

