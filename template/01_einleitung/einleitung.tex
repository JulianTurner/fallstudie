\chapter{Einleitung der Arbeit}

\section{Forschungsfrage}

Welche Auswirkungen haben \ac{Kfz} auf die Umwelt und wie kann autonomes Fahren die negativen Auswirkungen reduzieren?

\section{Hypothese}

Je mehr \ac{Kfz} autonom fahren, desto geringer fällt die Feinstaubbelastung durch \ac{Kfz} aus.

\section{Definitionen}
\subsection{Kraftfahrzeuge}

\textit{Als Kraftfahrzeuge im Sinne dieses Gesetzes gelten Landfahrzeuge, die durch Maschinenkraft bewegt werden, ohne an Bahngleise gebunden zu sein.}
\footnote{Straßenverkehrsgesetz, § 1 Abs. 2}
\vspace{0.5cm}
\newline
\aclp{Kfz} können in folgende Kategorien eingeteilt werden\footnote{VERORDNUNG (EU) Nr. 678/2011 DER KOMMISSION
	vom 14. Juli 2011, TEIL A ABS.1 - https://eur-lex.europa.eu/eli/reg/2011/678/oj?locale=de}:
\begin{itemize}
	\item Klasse M: Vorwiegend für die Beförderung von Fahrgästen und deren Gepäck ausgelegte und gebaute Kraftfahrzeuge
	\item Klasse N: Vorwiegend für die Beförderung von Gütern ausgelegte und gebaute Kraftfahrzeuge
	\item Klasse O: Anhänger, die sowohl für die Beförderung von Gütern und Fahrgästen als auch für die Unterbringung von Personen ausgelegt und gebaut sind
	\item Klasse S: unvollständige Fahrzeuge, die der Unterklasse der Fahrzeuge mit besonderer Zweckbestimmung zugeordnet werden soll
\end{itemize}

\newpage

\subsection{Autonomes Fahren}

Beim autonomen Fahren, fährt ein \ac{Kfz} Verwaltungsgefäß selbständig.
Für \ac{Kfz} wurden von der \ac{SAE} Institut in der Norm SAE J3016\footnote{SAE J3016\textunderscore202104 - https://www.sae.org/standards/content/j3016\textunderscore202104} Automatisierungsgrade definiert.
\begin{itemize}
	\item Stufe 0 (Keine Automation)
	\item Stufe 1 (Assistenzsysteme)
	\item Stufe 2 (Teilautomatisierung)
	\item Stufe 3 (Bedingte Automatisierung)
	\item Stufe 4 (Hochautomatisierung)
	\item Stufe 5 (Vollautomatisierung)
\end{itemize}
\subsubsection{Was passiert in den Stufen?}
Die Stufen unterscheiden sich im wesentlichen nur durch die Anzahl der Automatisierungsgrade.

\vspace{0.5cm}

In der Stufe 0 (Keine Automation):
\begin{itemize}
	\item keine Assistenzsysteme
	\item \ac{Kfz} kann keine Fahraufgaben übernehmen
	\item Fahrer ist unter permanenter Kontrolle
\end{itemize}

\vspace{0.5cm}

In der Stufe 1 (Assistenzsysteme):
\begin{itemize}
	\item Assistenzsysteme wie \ac{GRA} oder eine Berganfahrhilfe
	\item Fahrer hat eine passive Unterstützung bei Fahraufgaben
	\item \ac{Kfz} kann keine Fahraufgaben übernehmen
	\item Fahrer ist unter permanenter Kontrolle
\end{itemize}

\vspace{0.5cm}

In der Stufe 2 (Teilautomatisierung):
\begin{itemize}
	\item Assistenzsysteme, wie der Spurführungsassistent oder Stauassistent
	      \begin{itemize}
		      \item automatisch bremsen
		      \item automatisch beschleunigen
		      \item automatisch lenken
	      \end{itemize}
	\item \ac{Kfz} kann Fahraufgaben teilautomatisiert übernehmen
	\item Fahrer kann sich für kurze Zeit von den Fahraufgaben abwenden
	\item Fahrer muss jeder Zeit die Fahraufgabe übernehmen können
\end{itemize}

\vspace{0.5cm}

In der Stufe 3 (Bedingte Automatisierung):
\begin{itemize}
	\item hochautomatisierte Assistenzsysteme
	\item \ac{Kfz} kann Fahraufgaben unter bestimmten Voraussetzungen vollständig übernehmen
	\item Fahrer kann sich unter bestimmten Voraussetzungen dauerhaft von den Fahraufgaben abwenden
	\item Fahrer muss innerhalb wenigen Sekunden die Fahraufgabe übernehmen können
\end{itemize}

\vspace{0.5cm}

In der Stufe 4 (Hochautomatisierung):
\begin{itemize}
	\item hochautomatisierte Assistenzsysteme
	\item \ac{Kfz} kann Fahraufgaben in hochkomplexen Verkehrssituationen vollständig übernehmen
	\item Fahrer dauerhaft von den Fahraufgaben abwenden
	\item Fahrer muss fahrtüchtig sein, um im Bedarfsfall die Fahraufgabe übernehmen zu können
\end{itemize}

\vspace{0.5cm}

In der Stufe 5 (Vollautomatisierung):
\begin{itemize}
	\item hochautomatisierte Assistenzsysteme
	\item \ac{Kfz} übernimmt alle Fahraufgaben vollständig
	\item Fahrer ist nicht erforderlich
	\item alle Personen im Wagen werden zu Passagieren
\end{itemize}

\subsection{Umwelteinflüsse}

\textit{Umwelt bezeichnet etwas, mit dem ein Lebewesen in Beziehungen steht.}\footnote{Ludwig Trepl: Allgemeine Ökologie. Band 1: Organismus und Umwelt. Frankfurt/M., Lang: 106ff.; vgl. 1. Uexküll, Jakob von 1909: Umwelt und Innenwelt der Tiere. Springer, Berlin 2005.}

Einfluss ist eine Wirkung auf ein Subjekt, das eine bestimmte Umweltbedingung erfüllt.

Umwelteinflüsse sind daher eine Wirkung auf Lebewesen.
\newline
\newline
Unter Umwelteinflüssen von \ac{Kfz} fallen \ac{ua}:
\begin{itemize}
	\item benötigte Flächen, für Infrastruktur, Parkplätze \ac{usw}
	\item der Verbrauch von Stoffen um Energie für \ac{Kfz} zu erzeugen oder Betriebszustände für Fahrbahnen herzustellen
	\item der Ausstoß von Gasen die \ac{zb} durch Verbrennung von Kraftstoff oder beim Laden einer Batterie entstehen
	\item der Verlust von Betriebsmitteln durch Leckage an Systemen
	\item der Ausstoß von festen Stoffe wie \ac{ua} Bremsstaub oder Abrieb der Reifen der beim Bremsen entsteht
	\item Wärme und Schall durch die Umwadlung von Energie oder Reibung von Komponenten die beim Betrieb des \ac{Kfz} entstehen
	\item Licht zur Beleuchtung der Fahrbahn oder Absicherung der Verkehrsführung
\end{itemize}


\subsection{Feinstaub}

Feinstaub kann natürlichen Ursprungs sein oder durch menschliches Handeln erzeugt werden,
und wird in die Kategorien primär und sekundär unterteilt.

Der primäre Feinstaub entsteht direkt aus der Quelle wie durch eine Verbrennung.  

Der sekundäre Feinstaub entsteht durch eine chemische Reaktionen in der Atmosphäre aus gasförmigen Substanzen, 
wie Schwefel- und Stickstoffoxiden, Ammoniak oder Kohlenwasserstoffen.

Wichtige durch menschliches Handeln verursachte Feinstaubquellen sind: 
\begin{itemize}
	\item \acfp{Kfz}
	\item Kraft- und Fernheizwerke
	\item Abfallverbrennungsanlagen
	\item Heizungen in Wohnhäusern
	\item bestimmte Industrieprozesse
\end{itemize}

In urbanen Regionen sind vor allem der Straßenverkehr und Bautätigkeiten große Feinstaubquellen.

Hierbei entsteht Feinstaub nicht nur aus dem Verbrennungsprozess in die Luft, sondern auch durch Bremsen-, Reifen- und Fahrbahnabrieb. 
Auch die Aufwirbelungen des Staubes von der Straßenoberfläche tragen dazu bei. 
Wichtige natürliche Quellen für Feinstaub sind Emissionen aus Vulkanen und Meeren aber durch Bodenerosionen, Wald- und Buschfeuer oder 
bestimmte biogene Gemische von Viren, Sporen, Bakterien oder Pilzen.