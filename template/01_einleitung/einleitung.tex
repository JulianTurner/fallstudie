\chapter{Einleitung der Arbeit}

\section{Forschungsfrage}

Welche Auswirkungen haben Kraftfahrzeuge auf die Umwelt und wie kann autonomes Fahren die negativen Auswirkungen reduzieren?

\section{Hypothese}

Je mehr Fahrzeuge autonom fahren, desto geringer fällt die Feinstaubbelastung durch Kfz aus.



\section{TBD}

-Kraftfahrzeuge einfluss auf die Umwelt
-steigende Zulassung bei Fahrzeugen durch steigende Globalisierung
-steigende Temperaturen, Feinstaubbelastung
-steigende CO2-Emissionen durch Kfz
-durch bessere Fertigungsmethoden gibt es mehr Autos
-Umweltbelastung durch Umwege
-Umweltbelastung durch Fahrfehler
-Umweltbelastung durch Kfz-Fahrer
-Umweltbelastung durch Staus
-Umweltbelastung durch LKW

\section{Definitionen}
\subsection{Kraftfahrzeuge}

\textit{Als Kraftfahrzeuge im Sinne dieses Gesetzes gelten Landfahrzeuge, die durch Maschinenkraft bewegt werden, ohne an Bahngleise gebunden zu sein.}
\footnote{Straßenverkehrsgesetz, § 1 Abs. 2}
\newline
\newline
Kraftfahrzeuge können in folgende Kategorien eingeteilt werden\footnote{VERORDNUNG (EU) Nr. 678/2011 DER KOMMISSION
	vom 14. Juli 2011, TEIL A ABS.1 - https://eur-lex.europa.eu/eli/reg/2011/678/oj?locale=de}:
\begin{itemize}
	\item Klasse M (Vorwiegend für die Beförderung von Fahrgästen und deren Gepäck ausgelegte und gebaute Kraftfahrzeuge)
	\item Klasse N (Vorwiegend für die Beförderung von Gütern ausgelegte und gebaute Kraftfahrzeuge.)
	\item Klasse O (Anhänger, die sowohl für die Beförderung von Gütern und Fahrgästen als auch für die Unterbringung von Personen ausgelegt und gebaut sind.)
	\item Klasse S (unvollständige Fahrzeuge, die der Unterklasse der Fahrzeuge mit besonderer Zweckbestimmung zugeordnet werden soll)
\end{itemize}


\subsection{Autonomes Fahren}

Autonomes fahren ist wenn ein Kraftfahrzeug verwaltungsgemäß selbständig fährt.
\newline
Für Kraftfahrzeuge wurden vom SAE Institut in der Norm SAE J3016\footnote{SAE J3016\textunderscore202104 - https://www.sae.org/standards/content/j3016\textunderscore202104} Automatisierungsgrade definiert.
\begin{itemize}
	\item Stufe 0 (Keine Automation)
	\item Stufe 1 (Assistenzsysteme)
	\item Stufe 2 (Teilautomatisierung)
	\item Stufe 3 (Bedingte Automatisierung)
	\item Stufe 4 (Hochautomatisierung)
	\item Stufe 5 (Vollautomatisierung)
\end{itemize}
\subsubsection{Was passiert in den Stufen?}
Stufe 0
Fahrer hate keine Assistenzsysteme und ist unter permanenter Kontrolle.
\newline
Stufe 1
Assistenzsysteme wie eine Geschwindigkeitsregelanlage oder eine Berganfahrhilfe unterstützen den Fahrer bei Fahraufgaben.
\newline
Stufe 2
Assistenzsysteme, wie der Spurführungsassistent oder Stauassistent unterstützen den Fahrer bei Fahraufgaben.
\newline
Sie können automatisch bremsen, beschleunigen und im Gegensatz zu Level 1 auch das Steuer teilautomatisiert übernehmen.


Stufe 3
Der Fahrer sich unter bestimmten Voraussetzungen dauerhaft vom Verkehrsgeschehen abwenden kann und die Fahraufgabe vollständig an das Kraftfahrzeug delegiert.

Das Kraftfahrzeug ist mittels hochautomatisierter Systeme in der Lage, in bestimmten Verkehrssituationen wie z.B. Autobahnen komplett selbständig zu fahren.
Der Fahrer muss jedoch innerhalb wenigen Sekunden die Fahraufgabe wieder übernehmen können.


Stufe 4
Das Kraftfahrzeug delegiert den überwiegenden Teil seiner Fahrt selbständig.
Das Kraftfahrzeug ist in der Lage auch in hochkomplexen Verkehrssituationen wie z. B. plötzlich auftretende Baustellen komplett selbständig zu fahren.
Der Fahrer muss dennoch fahrtüchtig sein, um im Bedarfsfall die Fahraufgabe übernehmen zu können.

Stufe 5
Das Fahrzeug übernimmt alle Fahrfunktionen.
Im Gegensatz zu Level 3 und 4 ist beim völlig autonomen Fahren weder eine Fahrtüchtigkeit noch eine Fahrerlaubnis erforderlich.
Alle Personen im Wagen werden zu Passagieren, wodurch zum Beispiel auch Menschen mit Handicap neue Möglichkeiten der Mobilität eröffnet werden.



Das Kraftfahrzeug
Welche Systeme gibt es?

Beipiele für Assistenzsysteme:
\begin{itemize}
	\item Antiblockiersystem (ABS)
	\item Antriebsschlupfregelung (ASR)
	\item Berganfahrhilfe
\end{itemize}

Beipiele für Teilautomatisierung:
\begin{itemize}
	\item Spurhalteassistent
	\item adaptive Geschwindigkeitsregelung
	\item Berganfahrhilfe
\end{itemize}

\subsection{Umwelteinflüsse}

Hier wird erklärt was Umwelteinflüsse sind
Umwelteinflüsse wie Abgas, Lärm, Feinstaub, Licht

\subsubsection{Feinstaubbelastung}
Was ist eine Feinstaubbelastung?
Durch natürliche Art oder menschliche erzeugt
\subsubsection{Co2 Emissionen}
Was ist Co2?
Was sind CO2 Emissionen?
Natürliche CO2 Emissionen oder menschliche CO2 Emissionen
