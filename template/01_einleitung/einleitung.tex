\chapter{Einleitung der Arbeit}

\section{Hintergrund und Ausgangssituation}

Seit einigen Jahren hat sich die Welt in das Digitale geändert. Die Kommunikation untereinander geschieht schneller denn je. 
Meetings werden virtuell veranstaltet und Tätigkeiten digital erledigt. In der aktuellen Lage wird nunmehr erkannt, wie wichtig es ist, mit Computern und ohne direkten Kontakt zu arbeiten.
Dies führt in einigen Branchen dazu, dass vermehrt Systeme zum Einsatz kommen, die ihre Arbeit selbst verrichten. Zunächst sollen diese Systeme die Arbeit vereinfachen und sicherer gestalten.
Jedoch sehen Unternehmen in der Künstlichen Intelligenz ein größeres Potenzial. Sie soll einen Umbruch herbeiführen, welches Produktionsketten verschnellert und Kosten senken soll. Fortgeschrittene Unternehmen
setzen Instrumente der Künstliche Intelligenz in verschiedenen Abteilungen ein.


\section{Forschungsziel, Forschungsfrage und These}

Lorem ipsum dolor sit amet, consetetur sadipscing elitr, sed diam nonumy eirmod tempor invidunt ut labore et dolore magna aliquyam erat, sed diam voluptua. 
At vero eos et accusam et justo duo dolores et ea rebum.\footnote{Vgl. Baumann, M. F. u. a., Taking responsibility: A responsible research and innovation (RRI) perspective on insurance issues of semi-autonomous driving, 2019, S. 558.} 
Stet clita kasd gubergren, no sea takimata sanctus est Lorem ipsum dolor sit amet. \footnote{Baumann, M. F. u. a., Taking responsibility: A responsible research and innovation (RRI) perspective on insurance issues of semi-autonomous driving, 2019, S. 557.}
Automation in cars has a long history.  Lorem ipsum dolor sit amet, consetetur sadipscing elitr, sed diam nonumy eirmod tempor invidunt ut labore et dolore magna aliquyam erat, sed diam voluptua. 
At vero eos et accusam et justo duo dolores et ea rebum. 
Stet clita kasd gubergren, no sea takimata sanctus est Lorem ipsum dolor sit amet Acceptance of autonomous driving will depend on how far a consensus on these norms can be found, first among experts, then in society at large. 
One ethical condition, however, should be crucial: in no case should the ethical algorithms be put in practice as nontransparent black boxes. 
The built-in norms should, as far as possible, be understood and commonly shared.



\section{Aufbau der Arbeit}

Lorem ipsum dolor sit amet, consetetur sadipscing elitr, sed diam nonumy eirmod tempor invidunt ut labore et dolore magna aliquyam erat, sed diam voluptua. 
At vero eos et accusam et justo duo dolores et ea rebum. \footnote{Vgl. Martínez-Díaz, M./Soriguera, F./Pérez, I., Autonomous driving: a bird's eye view, 2019, S. 563}
Stet clita kasd gubergren, no sea takimata sanctus est Lorem ipsum dolor sit amet. 
Automation in cars has a long history.  Lorem ipsum dolor sit amet, consetetur sadipscing elitr, sed diam nonumy eirmod tempor invidunt ut labore et dolore magna aliquyam erat, sed diam voluptua. 
At vero eos et accusam et justo duo dolores et ea rebum. 
Stet clita kasd gubergren, no sea takimata sanctus est Lorem ipsum dolor sit amet Acceptance of autonomous driving will depend on how far a consensus on these norms can be found, first among experts, then in society at large. 
One ethical condition, however, should be crucial: in no case should the ethical algorithms be put in practice as nontransparent black boxes. 
The built-in norms should, as far as possible, be understood and commonly shared.