\chapter{Einleitung der Arbeit}

\section{Forschungsfrage}

Welche Auswirkungen haben \acfp{Kfz} auf die Umwelt und wie kann autonomes Fahren die negativen Auswirkungen reduzieren?

\section{Hypothese}

Je mehr \aclp{Kfz} autonom fahren, desto geringer fällt die Feinstaubbelastung durch \acfp{Kfz} aus.



\section{TBD}

-Kraftfahrzeuge einfluss auf die Umwelt
-steigende Zulassung bei Fahrzeugen durch steigende Globalisierung
-steigende Temperaturen, Feinstaubbelastung
-steigende CO2-Emissionen durch Kfz
-durch bessere Fertigungsmethoden gibt es mehr Autos
-Umweltbelastung durch Umwege
-Umweltbelastung durch Fahrfehler
-Umweltbelastung durch Kfz-Fahrer
-Umweltbelastung durch Staus
-Umweltbelastung durch LKW

\section{Definitionen}
\subsection{\acfp{Kfz}}

\textit{Als Kraftfahrzeuge im Sinne dieses Gesetzes gelten Landfahrzeuge, die durch Maschinenkraft bewegt werden, ohne an Bahngleise gebunden zu sein.}
\footnote{Straßenverkehrsgesetz, § 1 Abs. 2}
\vspace{0.5cm}
\newline
\aclp{Kfz} können in folgende Kategorien eingeteilt werden\footnote{VERORDNUNG (EU) Nr. 678/2011 DER KOMMISSION
	vom 14. Juli 2011, TEIL A ABS.1 - https://eur-lex.europa.eu/eli/reg/2011/678/oj?locale=de}:
\begin{itemize}
	\item Klasse M (Vorwiegend für die Beförderung von Fahrgästen und deren Gepäck ausgelegte und gebaute Kraftfahrzeuge)
	\item Klasse N (Vorwiegend für die Beförderung von Gütern ausgelegte und gebaute Kraftfahrzeuge.)
	\item Klasse O (Anhänger, die sowohl für die Beförderung von Gütern und Fahrgästen als auch für die Unterbringung von Personen ausgelegt und gebaut sind.)
	\item Klasse S (unvollständige Fahrzeuge, die der Unterklasse der Fahrzeuge mit besonderer Zweckbestimmung zugeordnet werden soll)
\end{itemize}


\subsection{Autonomes Fahren}

Beim autonomen Fahren, fährt ein \ac{Kfz} verwaltungsgemäß selbständig.
Für \ac{Kfz} wurden von der \ac{SAE} Institut in der Norm SAE J3016\footnote{SAE J3016\textunderscore202104 - https://www.sae.org/standards/content/j3016\textunderscore202104} Automatisierungsgrade definiert.
\begin{itemize}
	\item Stufe 0 (Keine Automation)
	\item Stufe 1 (Assistenzsysteme)
	\item Stufe 2 (Teilautomatisierung)
	\item Stufe 3 (Bedingte Automatisierung)
	\item Stufe 4 (Hochautomatisierung)
	\item Stufe 5 (Vollautomatisierung)
\end{itemize}
\subsubsection{Was passiert in den Stufen?}
In der Stufe 0 (Keine Automation):
\begin{itemize}
	\item keine Assistenzsysteme
	\item \ac{Kfz} kann keine Fahraufgaben übernehmen
	\item Fahrer ist unter permanenter Kontrolle
\end{itemize}

In der Stufe 1 (Assistenzsysteme):
\begin{itemize}
	\item Assistenzsysteme wie Geschwindigkeitsregelanlage oder eine Berganfahrhilfe
	\item Fahrer hat eine passive Unterstützung bei Fahraufgaben
	\item \ac{Kfz} kann keine Fahraufgaben übernehmen
	\item Fahrer ist unter permanenter Kontrolle
\end{itemize}

In der Stufe 2 (Teilautomatisierung):
\begin{itemize}
	\item Assistenzsysteme, wie der Spurführungsassistent oder Stauassistent
	      \begin{itemize}
		      \item automatisch bremsen
		      \item automatisch beschleunigen
		      \item automatisch lenken
	      \end{itemize}
	\item \ac{Kfz} kann Fahraufgaben teilautomatisiert übernehmen
	\item Fahrer kann sich für kurze Zeit von den Fahraufgaben abwenden
	\item Fahrer muss jeder Zeit die Fahraufgabe übernehmen können
\end{itemize}

In der Stufe 3 (Bedingte Automatisierung):
\begin{itemize}
	\item hochautomatisierte Assistenzsysteme
	\item \ac{Kfz} kann Fahraufgaben unter bestimmten Voraussetzungen vollständig übernehmen
	\item Fahrer kann sich unter bestimmten Voraussetzungen dauerhaft von den Fahraufgaben abwenden
	\item Fahrer muss innerhalb wenigen Sekunden die Fahraufgabe übernehmen können
\end{itemize}

In der Stufe 4 (Hochautomatisierung):
\begin{itemize}
	\item hochautomatisierte Assistenzsysteme
	\item \ac{Kfz} kann Fahraufgaben in hochkomplexen Verkehrssituationen vollständig übernehmen
	\item Fahrer dauerhaft von den Fahraufgaben abwenden
	\item Fahrer muss fahrtüchtig sein, um im Bedarfsfall die Fahraufgabe übernehmen zu können
\end{itemize}

In der Stufe 5 (Vollautomatisierung):
\begin{itemize}
	\item hochautomatisierte Assistenzsysteme
	\item \ac{Kfz} übernimmt alle Fahraufgaben vollständig
	\item Fahrer ist nicht erforderlich
	\item alle Personen im Wagen werden zu Passagieren
\end{itemize}


\subsection{Umwelteinflüsse}

\textit{Umwelt bezeichnet etwas, mit dem ein Lebewesen in Beziehungen steht.}\footnote{Ludwig Trepl: Allgemeine Ökologie. Band 1: Organismus und Umwelt. Frankfurt/M., Lang: 106ff.; vgl. 1. Uexküll, Jakob von 1909: Umwelt und Innenwelt der Tiere. Springer, Berlin 2005.}

Einfluss ist eine Wirkung auf ein Subjekt, das eine bestimmte Umweltbedingung erfüllt.

Unter Umwelteinflüssen von \ac{Kfz} fallen \ac{ua}:
\begin{itemize}
	\item benötigte Flächen, für Infrastruktur, Parkplätze \ac{usw}
	\item der Verbrauch von Stoffen um Energie für \ac{Kfz} zu erzeugen oder Betriebszustände für Fahrbahnen herzustellen
	\item der Ausstoß von Gasen die \ac{zb} durch Verbrennung von Kraftstoff oder beim Laden einer Batterie entstehen
	\item der Verlust von Betriebsmitteln durch Leckage an Systemen
	\item der Ausstoß von festen Stoffe wie \ac{ua} Bremsstaub oder Abrieb der Reifen der beim Bremsen entsteht
	\item Wärme und Schall durch die Umwadlung von Energie oder Reibung von Komponenten die beim Betrieb des \ac{Kfz} entstehen
	\item Licht zur Beleuchtung der Fahrbahn oder Absicherung der Verkehrsführung
\end{itemize}


\subsection{Feinstaub}

Als Feinstaub bezeichnet man Teilchen in der Luft die nicht sofot zu Boden fliegen.
\newline
Feinstaub kann nicht mit dem bloßen Auge wargenommen werden.

flüchtiger organischer Verbindungen (VOCs), \ac{NO}, \ac{CO} sowie dem Treibhausgas \ac{CO2} \ac{NOX}
Reifenabrieb, Bodenerosionen und Staubaufwirbelung, erzeugen Autos zudem Feinstaub
Darüber hinaus trägt der Autoverkehr maßgeblich zur Bildung von bodennahem Ozon bei. In Kombination mit UV-Strahlen, entstehen aus sogenannten primären Schadstoffen, wie Stickoxid und Kohlenmonoxid, gefährliche Photooxidantien – darunter Ozon. Davon abzugrenzen ist allerdings das Ozon in der Stratosphäre, wo es uns vor schädlicher UV-Strahlung schützt.
In Großstädten, wo der Verkehr dicht und der Treibstoffverbrauch durch ein stetiges Stop-u-Go erhöht ist, ist die Umweltbelastung durch Autoabgase besonders hoch. Bei ungünstigen, windstillen Wetterlagen kann die Luftverschmutzung hier zeitweise so stark werden, dass es zu Smog kommt.
Folgen für Mensch und Umwelt
Die Schadstoffemission durch den Verkehr, hat gravierende Folgen für Menschen und Umwelt.

Dazu zählen:

gesundheitliche Probleme, z.B.
Erkrankungen der Atemwege, z.B. Asthma
Kopfschmerzen und Konzentrationsschwäche
Einschränkung der Leistungsfähigkeit
Herz-Kreislauf-Erkrankungen
Krebs
Smog
Klimawandel
Umweltverschmutzung, z.B. Eutrophierung von Gewässern und Böden
dadurch Beeinträchtigung der Ökosysteme
Ernteschäden durch Ozonbelastung
Beschädigung von Kulturgütern und Baumaterialien


Feinstaubpartikel der Fraktion PM0,1 sind so klein, dass sie nicht nur in unsere Lungenbläschen gelangen, sondern sogar in die Blutlaufbahn. Dadurch können langfristig Herzinfarkte oder Schlaganfälle ausgelöst werden. Dieselruß hingegen, setzt sich in den Schleimhäuten und im Lungengewebe fest, wo sie bei dauerhafter Belastung zu Entzündungen führen[2,3].

Ozon als sekundär gebildeter Luftschadstoff, ist in höheren Konzentrationen toxisch und reizt Atemwege und Schleimhäute. Es gilt als Reizgas und ist insbesondere während langanhaltenden heißen Sommertagen in Städten ein Problem.



Eine weitere Umweltbelastung stellt das durch Autos emittierte Treibhausgas CO2 dar. Dieses ist zwar ungiftig und birgt somit keine (direkte) gesundheitliche Gefahr, jedoch trägt es maßgeblich zum Klimawandel bei.

Zusätzlich zur Luftverschmutzung führen Autos außerdem zu vielen anderen Problemen. Der Straßenverkehr benötigt kostbare Bodenfläche, verlangt viel Energie und führt zudem zu Lärmverschmutzung mit gesundheitlichen Folgen. So kann chronische Lärmbelastung unter anderem zu Hörschäden und Stress führen[1].

verkehrslärm


Altfahrzeuge werden nach der Basler Konvention und der Abfallverbringungsverordnung als gefährliche Abfälle eingestuft und dürfen nur in OECD-Länder exportiert werden. Dennoch gibt es immer wieder Berichte (zum Beispiel bei Frontal21[7] am 31. März 2015), dass nicht nur Gebrauchtfahrzeuge, sondern auch Altfahrzeuge aus Deutschland nach Afrika, Nahost und in östlich der EU gelegene Länder exportiert werden. Dort werden sie, obwohl sie sicherheitstechnisch und abgastechnisch nicht mehr den deutschen Anforderungen entsprechen, oft noch lange Zeit gefahren. Viele Zielländer des Exportes von Gebraucht- und/oder Altfahrzeugen haben inzwischen Einschränkungen oder Verbote erlassen, um den unkontrollierten Import von unsicheren und umweltschädlichen Fahrzeugen zu unterbinden.[8]

\subsubsection{Co2 Emissionen}
Was ist \ac{CO2}?
Was sind \ac{CO2} Emissionen?
Natürliche \ac{CO2} Emissionen oder menschliche CO2 Emissionen
