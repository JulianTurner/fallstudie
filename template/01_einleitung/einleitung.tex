\chapter{Einleitung}

\section{Geschichte des Internets}

\textit{„This is a fantasy documentary. The pioneering work shown in Hyperland however, is very real.“} \cite{Hyperland}
Mit diesen Worten beginnt Dougles Adams 50-minütige Dokumentation aus dem Jahre 1990 mit dem treffenden Titel „Hyperland“. Der renommierte Science-Fiction-Autor stellt in diesem Dokumentarfilm dar, wie ein Internet mit (Hyper-)Links interaktiv navigierbar wäre, und dies Jahre vor den ersten Browsern.
Im darauffolgenden Jahr entwickelte Tim Berners-Lee das World Wide Web, welches bald einen neuen Standard fürs Internet setzen sollte.
Dieser Standard erleichterte das Programmieren der einzelnen Internetseiten, sowie die Navigation zwischen den einzelnen Seiten selbst, und das durch nicht mehr als einen Klick dank der Einführung eines Standards für die Erstellung visueller Webseiten ein (Hypertext).[4, Seite 379] Diese Version des Internets ist heutzutage als Web 1.0 bekannt und, auch wenn dies damals noch nicht vorherzusehen war, prägte es die Vielfältigkeit und Sinnvollheit, wozu sich das Internet entwickeln würde.

In weniger als einem Jahrzehnt würde sich das Internet noch ändern. Die Ära des Web 2.0 ist geprägt von vielen Technologien und Innovationen, sowie von der Idee, dass der Nutzer immer weniger eine ausschließliche Rolle als passiven Konsumenten einnehmen soll, sondern vielmehr die doppelte Rolle des Konsumenten in einigen Bereichen und des Produktes in anderen, aber auch die des Anbieters, da immer mehr Anwendungen aus dem Web 2.0 dem Laien erlaubten, Inhalte selbst zu erstellen, zu bearbeiten und zu verteilen. Um diese neue Rolle zu definieren, hat sich der Begriff Prosument (aus dem Englischen Prosumer) durchgesetzt.
\textit{„Versammle eine Menge an Leuten, die glauben, sie seien wegen der einen Sache dort, in Wirklichkeit aber wegen einer ganz anderen Sache da sind.“} [1, Seite 51]
Eine der prägnantesten Innovationen dieser Internet-Epoche ist die Weiterentwicklung der Serveranbindung, aber auch neue Formen von Sicherheit, sowie Technologien, welche die Kommunikation zwischen dem Computer eines Nutzers und dem Server auf verschiedene Arten verbessern. Wegen der immensen Anzahl an Informationen, welche durch diese zentralisierte Server fließen, spricht man allgemein von „Big Data“. Dieser Begriff soll später noch genauer erklärt werden.

Doch auch diese Zeit scheint sich langsam dem Ende zu neigen. Und damit ist die Zeit reif für den Anbruch des Web 3.0. Beim Web 3.0 steht hauptsächlich die Idee im Vordergrund, dass das Internet dezentral über Peer-to-Peer-Netzwerke laufen soll.[4, Seite 42] Das wohl verbreiteste Beispiel hierfür wären die Blockchain-Netzwerke. Diese neue Erfindung bringt jedoch eine Reihe von technischen und rechtlichen Herausforderungen mit sich, und in vielen Ländern ist das Wissen über die Mechanismen, Möglichkeiten und Risiken des Web 3 noch unzureichend. Hinzu kommt, dass viele der Mechanismen noch in den Kinderschuhen stecken und für viele Menschen zu abstrakt erscheinen. Worte wie Kryptowährung, Smart Contracts und Token sind allgegenwärtig, aber es gibt immer noch einen Mangel an Informationen über die Mechanismen hinter diesen Anwendungen und den aktuellen Stand der Technik.

\begin{figure}[!ht]
    \caption{3 Epochen des Internets}
    \includegraphics[scale=1]{assets/figures/internetgeschichte.png}
    \begin{flushleft}
        Quelle: Token S. 19
    \end{flushleft}
    \label{fig:birds}
\end{figure}

\section{Forschungsziel, Forschungsfrage und These}
\subsection{Forschungsziel}
Wenn man von Distributed Ledger Systemen hört, wie beispielsweise dem Blockchain-Netzwerk, ist überwiegend von den positiven Aspekten sowie dem Potenzial, welches diese anbieten, die Rede.
\textit{„Jede Technologie ist aber lediglich ein Werkzeug und zunächst neutral. Wie wir dieses Werkzeug einsetzen, ist fast nie eine technologische, sondern immer mehr eine humanistische Frage.“} [4, Seite 25]
Im Folgenden soll zunächst ein Überblick über die Grundlagen der Technologien von Big Data und Blockchain gegeben werden. Anschließend sollen beide verglichen werden.
Ziel dieser Arbeit ist herauszufinden, ob es als plausibel erscheint, dass Big Data durch Blockchain ersetzt werden soll, und falls ja, welche Zeitspanne dafür denkbar wäre.
\subsection{Forschungsfrage}
Welche Vor- und Nachteile haben die beiden Strukturen (BigData und Blockchain) und wie zukunftstauglich sind sie?
\subsection{These}
Im Laufe der kommenden 10 Jahren wird Blockchain die meisten zentralen Systeme ersetzt haben.



\section{Aufbau der Arbeit}
Die vorliegende Arbeit unterteilt sich in mehrere Abschnitte.
Zuerst werden die beiden Datenbanksysteme und deren jeweiliger Aufbau sowie der derzeitige Stand der Technik anhand von Anwendungsmöglichkeiten erklärt.
Daraufhin werden beide Netzwerktypen auf mehrere Charakteristika untersucht.
Aufgrund dessen wird schließlich auf die Umsetzungswahrscheinlichkeit auf Blockchain, sowie auf die Zukunfttauglichkeit beider geschlossen.
Zuletzt trifft der Autor ein Urteil und gibt seine Meinung zu der Frage, ob und, wenn ja, wann Blockchain Big Data ersetzten könnte.
