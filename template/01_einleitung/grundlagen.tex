\chapter{Blockchain Netzwerke}

\section{Definitionen und Aufbau}

\subsection{Big Data}
Es gibt keine exakte Definition für Big Data. Diese Bezeichnung wird aber oftmals als Sammelbegriff benutzt für Daten, welche die Verarbeitungskapazität herkömmlicher Datenbanksysteme übersteigen.
Doug Laney hat in einem Forschungsbericht Big Data an verschiedenen Variablen analysiert, die er auf die sog. „drei V“ zurückgeführt hat, nämlich volume, velocity and variety, also Umfang, Geschwindigkeit und Varianz. All diese Charakteristika können dazu führen, dass die Information nicht in die Datenbankstrukturen passt.
Dennoch hat sich diese Form der zentralen Datenbanken durchgesetzt, so dass heutzutage fast alles, was digital ist, über eine Datenbank auf einem Server läuft.
Obwohl Big Data die Basis der heutigen Digitalisierung ist, hat sich die den Datenbanken zugrundeliegende Technologie seit den 1960er Jahren kaum weiterentwickelt, weswegen unsere Daten lediglich durch Absicherung der Server geschützt sind.
\subsection{Peer-to-Peer Netzwerke}
Peer-to-Peer Verbindungen (kurz P2P) sind Netzwerke zwischen einzelnen Rechnern. Grundidee hinter P2P ist, dass Computer direkt Daten austauschen können, ohne dabei Umwege über Internetserver zu gehen.
\subsection{Torrent Netzwerke}
In einem Torrent-Netzwerk, wie beispielsweise BitTorrent, sind die Dateien nicht auf einem Server gespeichert, sondern sie werden in Teilsegmente unterteilt und auf mehreren Rechnern verteilt. Möchte man eine Datei aus solch einem Netzwerk herunterladen, dann müssen sämtliche Teile aus den verschiedenen Rechnern abgerufen werden. Durch diese Bündelung vieler Rechner erreicht man hohe Downloadgeschwindigkeiten, ohne zentrale Server betreiben zu müssen. Genau dieses Modell wäre ein P2P-Netzwerk.
\subsection{Distributed Ledger}
Distributed Ledger ist ein Oberbegriff für verschiedene Datenbanktechnologien, welche ein System zur dezentralen Speicherung von Daten wie beispielsweise Blockchain haben. Anders als bei einer zentralen Datenbank gibt es hier keinen zentralen Administrator. Zur Kommunikation zwischen den einzelnen dezentralen Rechnern wird ein P2P-Netz eingesetzt. Ein Torrent-Netzwerk wäre ein Beispiel dazu.
\subsection{Blockchain}
Bei einer Blockchain handelt es sich um eine Kette von Transaktionen, die in den sogenannten Blöcken stattfinden. Blockchain-Software-Architekturen wurden ursprünglich entwickelt, um digitale Transaktionen sicherer zu machen.
Die zugrundeliegende Technologie ähnelt den P2P Netzwerken.
Jeder, der an einem Blockchain-Netz teilnehmen möchte, kann sich die Software herunterladen und sie auf seinem Rechner ausführen. Der Rechner wird damit zu einem neuen Knoten (Node) im Netz. Dadurch wird die ohnehin schon enorme Sicherheit noch einmal erhöht. Alle Transaktionen, die seit der Erstellung des ersten Knoten (auch als „Genesis Block“ bekannt) durchgeführt wurden, sind als verbundene Blöcke in einer verschlüsselten Datei gespeichert. Diese Datei existiert als Kopie auf jedem Knoten des Netzwerks.
Mit der Blockchain wurde eine neue Technologie entwickelt, die es uns erstmals ermöglicht, der Datenbank im Kern zu vertrauen. Wie wir Daten speichern, verschlüsseln und fälschungssicher machen, wurde von Grund auf neu gedacht.
\subsection{Public/Private Blockchain}
Der Unterschied zwischen einer Public und einer Private Blockchain liegt darin, wer Mitglied des Networks sein und die Konsensmechanismen ausführen darf.
Jeder, der die im Protokoll festgelegten Regeln und Verfahren befolgt, kann einem öffentlichen Blockchain-Netzwerk beitreten.
Bitcoin zum Beispiel ist ein öffentliches Blockchain-Netzwerk.
Im Gegensatz dazu ist ein privates Blockchain-Netzwerk geschlossen. Private Netzwerke können nur per Einladung beigetreten werden. Mitglieder müssen auch nach bestimmten Regeln validiert werden. Hier wird bestimmt, wer was sehen kann und wer an welchen Transaktionen teilnehmen kann.
Private Blockchains werden in der Regel von Unternehmen oder Regierungen betrieben, dabei können sich u. U. Einzelpersonen oder Organisationen daran beteiligen.
\subsection{Token}
Kryptografische Token stellen programmierbare Vermögenswerte oder Zugriffsrechte dar, die von einem intelligenten Vertrag und einem zugrunde liegenden verteilten Ledger verwaltet werden. Sie sind nur für die Person zugänglich, die den privaten Schlüssel für diese Adresse besitzt, und können nur mit diesem privaten Schlüssel signiert werden. Token könnten die Finanzwelt auf die gleiche Weise beeinflussen wie die E-Mail das Postsystem.
\subsection{Smart Contracts}
Smart Contracts sind einfach Programme, die in einer Blockchain gespeichert sind und ausgeführt werden, wenn bestimmte Bedingungen erfüllt sind. Sie werden in der Regel verwendet, um die rechtsgültige Ausfertigung eines Vertrags zu automatisieren, so dass alle Beteiligten sofort Gewissheit über das Ergebnis haben, ohne dass ein Vermittler eingeschaltet werden muss oder Zeit verloren geht. Sie können auch einen Workflow automatisieren und die nächste Aktion auslösen, wenn die Bedingungen erfüllt sind.


\section{Anwendungsmöglichkeiten}
\subsection{Bitcoin}
Bitcoin ist eine dezentralisierte digitale Währung, die im Januar 2009 geschaffen wurde. Sie folgt den Ideen, die in einem Weißbuch des mysteriösen und pseudonymen Satoshi Nakamoto dargelegt wurden.12 Die Identität der Person(en), die die Technologie entwickelt haben, ist immer noch ein Geheimnis. Bitcoin verspricht niedrigere Transaktionsgebühren als herkömmliche Online-Zahlungsmechanismen, und im Gegensatz zu staatlich ausgegebenen Währungen wird es von einer dezentralen Behörde betrieben.


Bitcoin ist als eine Art Kryptowährung bekannt, weil es Kryptographie verwendet, um es sicher zu halten. Es gibt keine physischen Bitcoins, sondern nur Guthaben, die in einem öffentlichen Hauptbuch geführt werden, auf das jeder Zugriff hat (obwohl jeder Datensatz verschlüsselt ist). Alle Bitcoin-Transaktionen werden durch eine riesige Menge an Rechenleistung in einem als Mining bekannten Prozess überprüft. Bitcoin wird nicht von Banken oder Regierungen ausgegeben oder unterstützt, noch ist ein einzelner Bitcoin als Ware wertvoll. Obwohl Bitcoin in den meisten Teilen der Welt kein gesetzliches Zahlungsmittel ist, erfreut er sich großer Beliebtheit und hat die Einführung Hunderter anderer Kryptowährungen ausgelöst, die unter dem Begriff Altcoins zusammengefasst werden. Bitcoin wird im Handel üblicherweise als BTC abgekürzt.
\subsection{Steemit}
Steemit ist eine Blockchain-basierte Social-Media-dApp, die Gemeinschaften schafft, in denen Nutzer für das Teilen ihrer Stimme belohnt werden. Es ist eine neue Art der Aufmerksamkeitsökonomie. 
Hier sind Nutzer in der Lage Tokens zu gewinnen, sogenannte STEEM’s, welche gegen herkömmliche Währungen umgetauscht werden können. Im Grunde gibt es 4 verschiedene Möglichkeiten, diese zu erhalten: 
Inhalte posten: Jeder Nutzer kann Inhalte hochladen. Je mehr weitere Nutzer diesen Post hochstufen, desto mehr Tokens kann man bekommen.
Freiberufliche Tätigkeit: Hier kann man sich mit einem Community-Mitglied auf Steemit vernetzen und seine Leidenschaft oder spezielle Fähigkeiten teilen, die man der Community als bezahlte Dienstleistungen anbietet oder als eigenes Produkt vermarktet werden kann.
Teilnahme an Wettbewerben und Herausforderungen: Steemitblog und viele andere Communities veranstalten regelmäßig Wettbewerbe, an denen jeder Steemianer teilnehmen kann. Wenn man gewinnt und von offiziellen Steem-Curation-Accounts oder anderen großen Walen hochgevotet wird, erhält man Belohnungen für seine Beiträge.
Handeln mit Steem: Hier kauft man STEEM von Börsen und lädt man sein STEEM auf Steem Power auf. Man vermietet es dann an andere Steemit-Benutzer gegen tägliche Steem-Zahlungen an seine Brieftasche.