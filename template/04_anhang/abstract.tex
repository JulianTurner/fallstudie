\newpage

\chapter*{Abstract}

Im Laufe der Zeit hat sich das Internet über verschiedene sogenannte Epochen entwickelt, welche sich stark voneinander unterscheiden.
Wir befinden uns gegenwärtig in der Web2-Ära, aber schon mit einigen Web3-Elementen.
Ziel dieser Arbeit ist es herauszufinden, ob und wann die Web3-Epoche, welche insbesondere von Blockchain-Strukturen und ähnlichen P2P-Netzwerken geprägt ist, die gegenwärtige ersetzen wird.
Im Verlauf dieser Arbeit wird ein genauerer Blick auf den Blockchain-Trend geworfen, verschiedene Vor- und Nachteile beider Strukturen werden abgewägt.
Hierbei handelt es sich um:
\begin{itemize}
    \item Kontrolle für Unternehmen
    \item Sicherheit
    \item Fälschungssicher
    \item Anonymität
    \item Energieverbrauch
    \item Kosten
\end{itemize}
Anschließend wird die Logik hinter der Umsetzung und die Zukunftstauglichkeit näher angeschaut.
Im Verlaufe dieser Arbeit kommt der Autor zum Schluss, dass die Umsetzung auf Blockchain-basierende Datenbanksysteme sehr wahrscheinlich ist, ja es ist zu erwarten, dass bereits in den nächsten Jahren Blockchain-Netzwerke einen zunehmenden Anteil sämtlicher Datenbanksysteme abdecken wird.