\chapter{Umweltbelastung}
Durch die voranschreitende technische Entwicklung in Industrie, Gewerbe und Landwirtschaft verändert und belastet der Mensch zunehmend die Umwelt.
Umweltbelastungen können viele Ursachen haben, möglicherweise sind bessere Lösungen nicht umsetzbar oder wirtschaftlich nicht attraktiv.
Die Umweltbelastung entsteht auf verschieden Ebenen, die sich in ihrer Gegebenheit unterscheiden.
Es gibt energetischen Belastungen, wie Strahlen, Lärm und Erschütterungen.
Es gibt Umweltbelastungen durch feste Stoffe wie Abfälle die durch Bau und Abbruch entstehen, Abfälle aus Produktionen und Abfälle aus der Gewinnung von Bodenschätzen.
Auch flüssige Stoffe belasten die Umwelt. Sie entstehen durch Chemie Fabriken, Reste von Medikamenten die durch den Urin in das Abwasser gelangen oder durch Umweltkatastrophen bei der sich das Wasser mit andren Stoffen vermischt.
Ein Beispiel hierfür könnte ein Erbebben sein, welches ein Atomkraftwerk beschädigt und radioaktives Wasser ausläuft.
Die größte Umweltbelastung für die Umwelt ist aber die gasförmige Verschmutzung, welche die Luft verschmutzt.
Die Gasformringe Verschmutzung welche die Luft verunreinigt ist die eine von den größten Belastungen für die Umwelt.
Durch unsachgemäße Wiederverwertung können Luftverschmutzungen entstehen, wie zum Beispiel die Verbrennung von
Stromkabeln um das Kupfer aus der Isolierung zu trennen.
Die Luftverschmutzung ist ebenso verantwortlich für Krankheiten und vorzeitigen Tod von Menschen.
Feinstaub kann in den Körper eindringen und schwerwiegende Krankheiten auslösen.
Luftverschmutzung entsteht bei Tierhaltung sowie durch den Einsatz von Pestiziden.

Die Hauptursache sind Abgase die bei der Verbrennung von fossilen Kraftstoffen entstehen.
Ein großer Träger bei der Verbrennung von fossilen Kraftstoffen sind Kraftfahrzeuge.
