\chapter{Schluss}
% \section{Zusammenfassung der Erkenntnisse}
% Gerade Luftschadstoffe können verehrende Wirkung auf Menschen und deren Umwelt haben.
% Diese Luftschadstoffe entstehen nicht nur durch durch Produktion von Kraftfahrzeugen, vor allem aber durch den Betrieb.
% Gerade die Verbrennung von fossilen Kraftstoffen und der Abrieb von Reifen und Bremsen können in Betrachtung genommen werden.

% Im Allgemeinen kann festgehalten werden, dass autonomes Fahren Veränderungen mit sich bringen kann.
% Wie diese sich diese letztlich zeigen wird hängt von unterschiedlichen Faktoren ab.
\subsection{Zusammenfassung}
Durch das voranschreiten des technologischen Vorschritts werden die Folgen von negativen Umweltbelastungen zunehmend stärker.
Eine Folge dieser negativen Umweltbelastungen können sich anhand von Luftschadstoffen und Feinstaub bemerkbar machen.
Die Luftschadstoffe sind gefährlich für Menschen, da sie schwere Krankheiten und Tod verursachen können.
Gerade Kraftfahrzeuge tragen einen Teil zu diesen Luftschadstoffen bei.

Kraftfahrzeuge können je nach ihrer Beschaffenheit in verschiedene Klassen eingeteilt werden.
Neue Technologien könnten dazu beitragen die negativen Auswirkungen von Kraftfahrzeugen auf die Umwelt zu reduzieren.
Einer dieser neuen Technologien könnte das autonome Fahren sein.
Auch der Automatisierungsgrad kann in verschiedene Klassen eingeteilt werden.

Da Luftschadstoffe durch die Verbrennung von fossilen Kraftstoffen entstehen,
können Potentiale im Bereich der Kraftstoffeinsparung und der stetigen Modernisierung von Kraftfahrzeugen erschließen.

Neben den Luftschadstoffen ist auch Feinstaub eine negative Belastung.
Da Feinstaub nicht nur durch die Verbrennung von fossilen Kraftstoffen entsteht,
sondern auch durch den Abrieb der Reifen und Bremsen und Produktion von Kraftfahrzeugen,
ergeben sich hier Einspareffekte bei der Reduzierung von Fahrten und der Reduktion von Fahrzeugbeständen.

Aufgrund der hohen Aktualität kann ein eindeutiger Trend noch nicht abgebildet werden.
Vielmehr kann die Summe von mehreren neuen Technologien und geänderten Verhaltensmuster eine Trendwende schaffen.
Hierunter könnte Elektromobilität, sowie der Verzicht auf ein eigenes Fahrzeug durch verschiedene
autonome Mobilitätsdienstleister eine Rolle spielen.

\subsection{Offene Fragen und Themen für weitere Entwicklungen}
\subsubsection{Wie können Verstöße gegen Verkehrsregeln gewichtet werden?}

Der Straßenverkehr wird durch verschiedene Verkehrsregeln geregelt.
Es kann Situationen geben die durch ein Verstoß von Verkehrsregeln ein
schlimmeres Ausmaß verhindern.
Wie können solche Verstöße in Einklang mit den Verkehrsregeln gebracht werden?

\subsubsection{Wie können autonome Systeme Dilemma entscheiden?}
Es kann Situationen geben in jener ein Personenschaden nicht mehreren Parteien mehr verhindert werden kann.
Wie kann ein autonomes Fahrzeug zwischen den verschiedenen Parteien entscheiden?
Welche Faktoren werden für die Entscheidung benötigt?

\subsubsection{Wann ist ein autonomes Kraftfahrzeug sicher?}
Aktuell gibt es keine eindeutige Definition ab wann ein Kraftfahrzeug als sicher eingestuft werden kann.
Welche Eigenschaften müssen hierfür erfolgt werden?