\chapter{Einleitung}
Anhand der technologischen Entwicklung, erster Testfelder, sowie Prototypen Industrie stellt  die Frage,
welche Umwelteffekte beim autonomen Fahren auftreten werden.
Dies hängt mitunter mit neuen Antriebsformen (z. B. Elektroantrieb) und Betriebsmodellen (z. B. geteilte
Fahrzeuge) zusammen.

Je nach der Entwicklung in den einzelnen Bereichen werden auch sich unterschiedliche Kraftfahrzeuge bei einer Einführung des autonomen Fahrens
durchsetzten.
Als Umwelteffekte des Autonomen Fahrens bezeichnet man die Einflüsse die sich durch autonomen Fahren ändern, hierbei kann es zur Verringern oder Anstieg von Schadstoffausstößen kommen, oder eine
Veränderung in der Infrastruktur, da Parkplätze oder Seitenstreifen nicht mehr in gleicher Zahl benötigt werden oder der Ausbau von Straßen da das Verkehrsaufkommen zugenommen hat.
Der Umstieg auf autonomes Fahren erfolgt nicht in einem Schritt,
sondern schrittweise oder je nach Fahrzeugs
nur für einzelne Anwendungsfälle (siehe Anlage I, Stufen der Automatisierung, autonomes
Fahren bei Level 5). Im Öffentlichen Verkehr (ÖV) könnten sich Veränderungseffekte schneller einstellen.

Denn durch den Verkehr verursachten Schadstoffe sind eine wesentliche Ursache für den schnell voranschreitenden Klimawandel.


Durch die voranschreitende technische Entwicklung in Industrie, Gewerbe und Landwirtschaft verändert und belastet der Mensch zunehmend die Umwelt.
Umweltbelastungen können viele Ursachen haben, möglicherweise sind bessere Lösungen nicht umsetzbar oder wirtschaftlich nicht attraktiv.
Die Umweltbelastung entsteht auf verschieden Ebenen, die sich in ihrer Gegebenheit unterscheiden.
Es gibt energetischen Belastungen, wie Strahlen, Lärm und Erschütterungen.
Es gibt Umweltbelastungen durch feste Stoffe wie Abfälle die durch Bau und Abbruch entstehen, Abfälle aus Produktionen und Abfälle aus der Gewinnung von Bodenschätzen.
Auch flüssige Stoffe belasten die Umwelt. Sie entstehen durch Chemie Fabriken, Reste von Medikamenten die durch den Urin in das Abwasser gelangen oder durch Umweltkatastrophen bei der sich das Wasser mit andren Stoffen vermischt.
Ein Beispiel hierfür könnte ein Erbebben sein, welches ein Atomkraftwerk beschädigt und radioaktives Wasser ausläuft.
Die größte Umweltbelastung für die Umwelt ist aber die gasförmige Verschmutzung, welche die Luft verschmutzt.
Die Gasformringe Verschmutzung welche die Luft verunreinigt ist die eine von den größten Belastungen für die Umwelt.
Durch unsachgemäße Wiederverwertung können Luftverschmutzungen entstehen, wie zum Beispiel die Verbrennung von
Stromkabeln um das Kupfer aus der Isolierung zu trennen.
Die Luftverschmutzung ist ebenso verantwortlich für Krankheiten und vorzeitigen Tod von Menschen.
Feinstaub kann in den Körper eindringen und schwerwiegende Krankheiten auslösen.
Luftverschmutzung entsteht bei Tierhaltung sowie durch den Einsatz von Pestiziden.

Die Hauptursache sind Abgase die bei der Verbrennung von fossilen Kraftstoffen entstehen.
Ein großer Träger bei der Verbrennung von fossilen Kraftstoffen sind Kraftfahrzeuge.


Gemeint sind damit Fahrzeuge,
die nicht nur automatisch fahren, sondern von einem System gesteuert und disponiert werden und
damit aus Sicht der Nutzenden „autonom“ unterwegs sind. Die Erwartungen gehen weit auseinander: während die einen in der Verbreitung autonomer Fahrzeuge die Lösung aller Probleme sehen,
prognostizieren die anderen eine Verschärfung der bereits angespannten Verkehrslage.
Welche Bedeutung aber solche Fahrzeuge
in Zukunft haben werden, hängt neben der Bewältigung der technischen Komplexität
von der Art und Weise der politischen Regulierung ab. Ob solche Systeme überhaupt im öffentlichen
Straßenraum erscheinen und welche Wirkungen sich für den zukünftigen Verkehr daraus ergeben,
entscheidet maßgeblich der gesetzliche Rahmen.
Dieser gesetzliche Rahmen ist wiederum abhängig
vom Zielbild, wie wir künftig leben wollen und wie der Verkehr von morgen aussehen soll.
Eckpfeiler dieses Zielbildes sind zum einen die Sicherheit und die Aufenthaltsqualität im öffentlichen Raum und
zum anderen die drastische Reduktion schädlicher Emissionen sowie nicht zuletzt die Erreichung der
Klimaschutzziele im Verkehr. Dafür bedarf es einer umfassenden Verkehrswende, die mehr ist als eine Antriebswende.
Private Pkw werden in den nächsten Jahren serienmäßig mit weitgehenden Assistenzfunktionen angeboten, damit können sie eine zusätzliche Attraktivität erhalten und der Autoverkehr kann sogar
noch weiter zunehmen. Eine fortschreitende Automatisierung privater Autos mit einem solchen Effekt wäre im Sinne der angestrebten Verkehrswende kontraproduktiv.
Dieser Entwicklung ist vor allem mit einem Abbau der Privilegien für das private Auto, also in erster Linie mit einer konsequenten Internalisierung der externen Kosten zu begegnen.
Auf der anderen Seite bieten (teil-) automatisierte Fahrzeuge, nicht zuletzt neue Fahrzeugformate zwischen Pkw und Bus, bereits jetzt neue
Chancen für einen effizienten und attraktiven Öffentlichen Verkehr.
Im ersten Schritt automatisierte Shuttles als ein neues Element eines künftigen Öffentlichen Verkehrs zu etablieren.
Gleichzeitig sollte der neu zu schaffende Regulierungsrahmen die heute schon entstehenden on demand-Verkehre ermöglichen und den Weg dafür frei machen, deren optimale Verknüpfung
mit dem klassischen ÖPNV zu erproben.
Diese neuen Angebote operieren heute noch mit Fahrern.
Sie sind aber als Vorläufer autonomer Fahrzeugflotten zu verstehen, die in Zukunft möglich werden.
Mit diesen neuen Angebotsformen im Zusammenspiel mit dem klassischen Umweltverbund erscheint es perspektivisch möglich, in einem definierten Bediengebiet eine wirkliche Alternative zum
privaten Pkw zu kreieren und damit die Zahl der Fahrzeuge, zumindest in der Stadt, insgesamt auf die
vom Umweltbundesamt angestrebten 150 Kraftfahrzeuge je 1.000 Einwohner deutlich zu verringern.
Voraussetzung ist, dass die Kapazitäten der öffentlichen Verkehrsangebote verdoppelt und mindestens ein Viertel davon im digitalen on demand-Modus bedient werden kann.
Mit einem hochintegrierten intermodalen Öffentlichen Verkehr ist in weiterer Zukunft die Mobilität
sogar mit noch viel weniger Fahrzeugen zu gewährleisten.
Eine zukünftige Regulierungspraxis könnte
durch einen Mix aus Groß- und Kleinfahrzeugen, aus spurgeführten und getakteten sowie flexiblen
on demand-Verkehren die Zahl der Straßenfahrzeuge zur Abwicklung sämtlicher Personenkilometer
in den Städten auf bis zu 50 Einheiten pro 1.000 Einwohner insgesamt reduzieren.
Gegenüber konventionellen Bussystemen können automatisch fahrende Shuttles die Kosten des operativen Betriebes um rund die Hälfte senken.
Im Ergebnis bedeutet dies, dass mit automatisierten Fahrzeugflotten
der Verkehr in Zukunft verlässlicher, sozial ausgewogener, leistungsfähiger und vor allen Dingen mit
einem geringeren Ressourceneinsatz gestaltet werden kann.
Ein völlig neu aufgestellter und flexiblerer Öffentlicher Verkehr ist damit verbunden, dass sich die
bisherigen Zuständigkeiten und Branchengrenzen verschieben und die Organisation des gesamten
Verkehrs als eine öffentliche Regieaufgabe begriffen werden muss.

es offensichtlich, dass für einen nachhaltigen Verkehr weder die bisherige Aufteilung des öffentlichen Raumes noch die Organisation und Finanzierung des Verkehrs, insbesondere
des Öffentlichen Verkehrs, beibehalten werden kann. Die bisherige Dominanz privater Fahrzeuge im
öffentlichen Raum gerade in der Stadt wird bei der Einführung automatisierter Systeme erodieren.
Der private motorisierte Individualverkehr wird durch neue gemeinschaftliche Mobilitätsformen
(„Sharing“) schrittweise zurückgefahren.
Kerngedanke einer Verkehrswende ist, dass sich der individualisierte Verkehrswunsch vom Eigentum an einem Fahrzeug löst und durch die Nutzung eines vielfältigen Fahrzeugparks ersetzt wird.
Voraussetzung dafür ist allerdings, dass die on demandAngebote verkehrs- und innovationspolitisch unterstützt werden.
Dazu gehört, dass die bereits heute möglichen, digital basierten Services zugelassen und als Teil des neuen öffentlichen Verkehrsangebotes verstanden werden.
Der Wettlauf um die Automatisierung im Verkehr hat längst begonnen. Die Ausgangslage verschiedener Akteure für den Start in die Zukunft der Mobilität sind aber sehr unterschiedlich: Während
digitale Plattformbetreiber wie Uber und Waymo offensiv Milliarden in die Technologieentwicklung
investieren, befinden sich die klassischen Automobilhersteller noch mitten im konventionellen Geschäftsmodell. Es ist davon auszugehen, dass die Automobilunternehmen mit einer Steigerung des
Automatisierungsgrades primär die Attraktivität des privaten Fahrzeuges zurückgewinnen möchten.
Damit könnten sich die Zahl der Fahrzeuge und ihre Fahrleistung noch weiter erhöhen. Die Entwickler
(teil-) automatisierter Shuttles bleiben in ihren technologischen Potenzialen hinter diesen Entwicklungen weit hinterher.
Die Betreiber öffentlicher Verkehrssysteme wiederum spielen in diesem Technologiewettbewerb
bislang überhaupt keine Rolle, sie müssen für diesen Wettbewerb erst noch ertüchtigt werden. Dafür
brauchen sie nicht zuletzt verlässliche rechtliche Rahmenbedingungen.
