\chapter{Einleitung}
% Anhand der technologischen Entwicklung, erster Testfelder, sowie Prototypen Industrie stellt  die Frage,
% welche Umwelteffekte beim autonomen Fahren auftreten werden.
% Dies hängt mitunter mit neuen Antriebsformen (z. B. Elektroantrieb) und Betriebsmodellen (z. B. geteilte
% Fahrzeuge) zusammen.

% Je nach der Entwicklung in den einzelnen Bereichen werden auch sich unterschiedliche Kraftfahrzeuge bei einer Einführung des autonomen Fahrens
% durchsetzten.
% Als Umwelteffekte des Autonomen Fahrens bezeichnet man die Einflüsse die sich durch autonomen Fahren ändern, hierbei kann es zur Verringern oder Anstieg von Schadstoffausstößen kommen, oder eine
% Veränderung in der Infrastruktur, da Parkplätze oder Seitenstreifen nicht mehr in gleicher Zahl benötigt werden oder der Ausbau von Straßen da das Verkehrsaufkommen zugenommen hat.
% Der Umstieg auf autonomes Fahren erfolgt nicht in einem Schritt,
% sondern schrittweise oder je nach Fahrzeugs
% nur für einzelne Anwendungsfälle (siehe Anlage I, Stufen der Automatisierung, autonomes
% Fahren bei Level 5). Im Öffentlichen Verkehr (ÖV) könnten sich Veränderungseffekte schneller einstellen.

% Denn durch den Verkehr verursachten Schadstoffe sind eine wesentliche Ursache für den schnell voranschreitenden Klimawandel.


% Durch die voranschreitende technische Entwicklung in Industrie, Gewerbe und Landwirtschaft verändert und belastet der Mensch zunehmend die Umwelt.
% Umweltbelastungen können viele Ursachen haben, möglicherweise sind bessere Lösungen nicht umsetzbar oder wirtschaftlich nicht attraktiv.
% Die Umweltbelastung entsteht auf verschieden Ebenen, die sich in ihrer Gegebenheit unterscheiden.
% Es gibt energetischen Belastungen, wie Strahlen, Lärm und Erschütterungen.
% Es gibt Umweltbelastungen durch feste Stoffe wie Abfälle die durch Bau und Abbruch entstehen, Abfälle aus Produktionen und Abfälle aus der Gewinnung von Bodenschätzen.
% Auch flüssige Stoffe belasten die Umwelt. Sie entstehen durch Chemie Fabriken, Reste von Medikamenten die durch den Urin in das Abwasser gelangen oder durch Umweltkatastrophen bei der sich das Wasser mit andren Stoffen vermischt.
% Ein Beispiel hierfür könnte ein Erbebben sein, welches ein Atomkraftwerk beschädigt und radioaktives Wasser ausläuft.
% Die größte Umweltbelastung für die Umwelt ist aber die gasförmige Verschmutzung, welche die Luft verschmutzt.
% Die Gasformringe Verschmutzung welche die Luft verunreinigt ist die eine von den größten Belastungen für die Umwelt.
% Durch unsachgemäße Wiederverwertung können Luftverschmutzungen entstehen, wie zum Beispiel die Verbrennung von
% Stromkabeln um das Kupfer aus der Isolierung zu trennen.
% Die Luftverschmutzung ist ebenso verantwortlich für Krankheiten und vorzeitigen Tod von Menschen.
% Feinstaub kann in den Körper eindringen und schwerwiegende Krankheiten auslösen.
% Luftverschmutzung entsteht bei Tierhaltung sowie durch den Einsatz von Pestiziden.

% Die Hauptursache sind Abgase die bei der Verbrennung von fossilen Kraftstoffen entstehen.
% Ein großer Träger bei der Verbrennung von fossilen Kraftstoffen sind Kraftfahrzeuge.



\section{autonome Fahrzeuge}
Sind Fahrzeuge die nicht nur automatisch fahren sondern von einem System gesteuert werden.
Somit sind diese Fahrzeuge aus Sicht der Nutzenden autonom.

Während manche in der Verbreitung autonomer Fahrzeuge die Lösung vieler Probleme sehen können,
vermuten andere eine Verschlechterung der Verkehrs- und Umweltlage.

Die Bedeutung von autonomen Fahrzeugen, hängt sowohl von der technischen Komplexität sowie von politischen Regulierung ab.

In welchem Maß die Level 5 Systeme im Straßenverkehr teilnehmen entscheidet vorerst der gesetzliche Rahmen.
Dies ist wiederum abhängig wie der Verkehr von morgen aussehen soll.
\subsection{Gesetzliche Regelungen}
Was ist bereits wo erlaubt?
Welche Länder haben was freigegeben?


Hauptsächlich wird Sicherheit und die Senkung schädlicher Emissionen sowie das Erreichen der
Klimaschutzziele im Verkehr die größte Rolle spielen.
\subsection{Klimaschutzziele}
Wer bestimmt Klimaschutzziele?
Welche Klimaschutzziele gibt es?
Welche Klimaschutzziele sollen erreicht werden?



\newpage

\chapter{Umwelt- und Klimaeffekte}
Die Umwelt- und Klimaeffekte durch autonomes Fahren ist durch die Vielzahl der Verknüpfungen mit anderen Technologien wie Elektromobilität
oder Dienstleistungen von Fahrdiensten und Car-Sharing-Angeboten im direkten Vergleich kaum noch bestimmbar.

Grundsätzlich werden Klimaeffekte abhängig von:
\begin{itemize}
	\item den eingesetzten Technologien
	\item den Kosten künftiger Mobilitätsdienstleister
	\item gesetzlichen Bestimmungen
	\item der Nachfrage von Nutzern
\end{itemize}

Im direkten Vergleich von autonomen Fahrzeugen zu konventionellen Fahrzeugen fällt ein geringer der Kraftstoffverbrauch und der geringere Ausstoß Feinstaub auf.

Andere Faktoren können die Umwelt entlasten wie der Abbau von Fahrzeugbeständen, und die dadurch wegfallenden negativen Einflüsse durch die Produktion von Fahrzeugen.

\section{Kraftstoffeinsparungen}
Kraftstoffeinsparungen durch autonomes Fahren im Straßenverkehr können sich durch
eine Steigerung der Effizienz im Verkehrsfluss und
einer abgestimmten Fahrweise bei einer optimalen Routenführung ergeben.

Erste Wirkungen können sich bereits im Mischverkehr aus autonom und konventionell gesteuerten Kraftfahrzeugen bemerkbar machen durch weniger
Brems- und Beschleunigungsvorgängen.

Die Wirkungen könnten sich mit einer steigenden Marktdurchdringung von autonomen Fahrzeugen
durch sinkenden stockenden Verkehr sowie eine Reduzierung von Staus bemerkbar machen.

\subsection{Kraftstoffeinsparungen auf Autobahnen}
Erste Abschätzungen zu Kraftstoff\-einsparungen im Individualverkehr auf Autobahnen in\\
Deutschland wurde vom Fraunhofer
Institut für Arbeitswirtschaft und Organisation (IAO) vorgenommen.

Dort wurde die Wirkung von drei bereits existierenden Assistenzsystemen (Stau-Chauffeur, Spurwechsel-Chauffeur, Autobahn-Chauffeur)
der Automatisierungsstufe 3 auf Autobahnen begutachtet.

Die Wirkung wurde für zwei verschiedene Szenarien ausgewiesen.
Das erste Szenario basiert darauf dass alle Kraftfahrzeuge autonom fahren.
Im zweiten Szenario betrug der Anteil ungefähr 45.000 autonome Fahrzeuge.

Auf der Basis von einer 10-20 prozentigen Reduzierung des Kraftstoffverbrauchs ergab sich
im ersten Szenario ein jährliches Sparpotential zwischen 360 und 720 Millionen Euro und
im zweiten Szenario von 0,4 bis 0,8 Millionen Euro.

Rechnet man diese Werte auf einzelne autonome Kraftfahrzeuge um,
ergäbe dies eine Einsparung für jedes autonome Kraftfahrzeug von 8 bis 16 € pro Jahr.
\footnote{Fraunhofer-Institut für Arbeitswirtschaft und Organisation IAO, Hochautomatisiertes
	Fahren auf Autobahnen – Industriepolitische Schlussfolgerungen, Studie im Auftrag des
	Bundesministeriums für Wirtschaft und Energie, November 2015, S. 264ff.}

Eine weitere Einsparung auf Autobahnen durch autonome Fahrzeuge könnte mit
dem systembedingten verzicht der Überschreitung von Fahrgeschwindigkeiten über der Richtgeschwindigkeit.
Der Verzicht von höheren Geschwindigkeiten auf Autobahnen kann zu enormen Einsparungen im Kraftstoffverbrauch führen,
da mit steigender Geschwindigkeit der Kraftstoffverbrauch überproportional rasant ansteigt.

\subsection{Kraftstoffeinsparungen im Individualverkehr im städtischen Verkehr}
Zu den Kraftstoffeinsparungen auf Autobahnen werden von
autonomen Kraftfahrzeugen besonders im städtischen Verkehr
eine deutliche Reduzierung im Verbrauch von Kraftstoffen erwartet.

Große Einsparpotentiale bieten hierfür:
\begin{itemize}
	\item die Steigerung des Verkehrsflusses insbesondere an Knotenpunkten
	\item Verstetigung der Geschwindigkeit
	\item die Vermeidung von Verkehrsstörungen
\end{itemize}

Wie die Auswertung internationaler Studien
\footnote{Milakis, D., van Arem, B., van Wee, B., Policy and society related implications of
	automated driving: A review of literature and directions for future research, Journal of
	Intelligent Transportation Systems, 2017, Vol. 21, No. 4, 335f} gezeigt hatsss,
können Einsparungen beim Kraftstoffverbrauch von
bis zu 31 Prozent im städtischen Bereich sowie
bis zu 45 Prozent bei einer optimierten Knotenpunktsteuerung erreicht werden.
Diese Abschätzungen basieren auf Simulations- und Modellrechnungen.
Die Menge der  Kraftstoffeinsparungen hängt unter anderem von mehreren Faktoren ab:
\begin{itemize}
	\item die verwendete Automatisierungsstufe
	\item der Marktanteil von autonomen Fahrzeugen
	\item die Vernetzung zwischen den Fahrzeugen
	\item die Vernetzung zwischen den Fahrzeugen und der Infrastruktur
\end{itemize}

\subsection{Kraftstoffeinsparungen bei Flottenfahrzeugen}
Bei Flottenfahrzeugen wäre eine intensivere Nutzung der autonomen Fahrzeuge zu erwarten, wodurch sich der Nutzungszeitraum verkürzt.
Dadurch werden Flottenfahrzeuge früher ausrangiert.
Durch das schnelleren wechsel der Flottenfahrzeuge werden Fahrzeuge mit technologischen Erneuerung
die weniger Luftschadstoffe und Emissionen produzieren schneller eingesetzt.
Dies könnte im städtischen Bereich zu einer Reduktion von Verkehrsbedingen Luftschadstoffen und Emissionen führen.

\subsection{Positive Umwelteinwirkung durch die Reduktion von Fahrzeugen}
Positive Umwelteinwirkung durch autonomes Fahren kann durch eine
Reduktion an Fahrzeugen erwartet werden.

Mobilitatsdienleiter mit autonomen Taxen können
Kunden abholen und zu Zielort bringen,
und vermindern die Notwendigkeit zum Kauf eines Fahrzeugs.

Verschiedene Studien haben in verschiedenen Szenarien berechnet, ob und wie viele Kraftfahrzeuge sich durch autonome Fahrzeuge zusammenfassen ließen.
Eine Studie kam zu dem Resultat, dass in etwa 18.000 Flottenfahrzeuge den kompletten Individualverkehr von München innerhalb eines Einzuggebietes decken könnte.
Dadurch könnte auf ungefähr 200 Tausend Fahrzeuge verzichtet werden.
\footnote{Alternative Antriebe, Autonomes Fahren,
	Mobilitätsdienstleistungen: Neue Infrastrukturen für
	die Verkehrswende im Automobilsektor, S. 42}


Gleichzeitig würde
durch Robo-Taxis auch der Parksuchverkehr entfallen, der ja in den Innenstädten

Erstens können neue Fahrten, z.B. zu Freizeitzwecken, und damit Mehrverkehr
ausgelöst werden, wenn die Mobilitätskosten sinken und das Fahren dadurch
attraktiver wird.
Zweitens ist eine Verkehrsverlagerung von Fahrten hin zu Robo-Taxis denkbar,
die vorher mit dem öffentlichen Verkehr bzw. mit dem Fahrrad zurückgelegt
wurden. So könnten autonome Fahrangebote wegen des besseren, weil komfortableren Service präferiert werden.
Dies könnte zu höheren Pkw-Fahrleistungen führen.92

Drittens können sich längerfristig auch raumintensivere Siedlungsstrukturen und damit weitere Wege zu den Aktivitätsorten (Wohnen, Arbeit, Freizeit) entwickeln.
So erleichtern autonome Fahrzeuge längere Pendelwege, da
die Mobilitätszeit durch den Wegfall der Fahraufgabe anders oder effizienter,
z.B. zum Arbeiten oder zum Schlafen, genutzt werden kann. Dadurch können
Wohnorte in stadtfernen Regionen künftig weit attraktiver werden, zumal nicht
mit sinkenden Mietpreisen in Metropolen zu rechnen ist.

Viertens können autonome Fahrzeuge auch privat bzw. als Firmenfahrzeuge gekauft und verwendet werden.93 Vorstellbar sind kundenindividuelle
Robo-Shuttles oder etwa «autonom fahrende Arbeits- und Schlafplätze», die
etwa von Außendienstmitarbeitern genutzt werden. Dadurch könnten die
Zahl der Fahrzeuge sowie die Verkehrsleistungen steigen und weitere negative
Umwelteffekte entstehen (z.B. «Autonome Schlafwagenflotten», die Hotelübernachtungen einsparen).
Insgesamt wird deutlich, dass bei autonomen Fahrzeugen bzw. Fahrdiensten
sowohl positive als auch negative Klima- bzw. Umwelteffekte möglich wären. Hier
braucht es noch weitergehende Untersuchungen, um durch wirksame regulative
Rahmenbedingungen ggf. unerwünschte Umweltfolgen zu verhindern



































\section{LKW}
In den Marktstudien wird darauf verwiesen, dass sich die Wirkungen vor
allem aus Einsparungen bei den Personalkosten und bei den Kraftstoffkosten
zusammensetzen. PwC sieht durch Platooning ein Einspareffekt bei den
Kraftstoffkosten von rund 7Prozent. Roland Berger beziffert den Einspareffekt in
Höhe von 8 Prozent für das führende Fahrzeug und in Höhe von 14Prozent für die
folgenden Fahrzeuge bei einer Geschwindigkeit von 85 km/h.16 Das
International Transport Forum beziffert die Kraftstoffkosteneinsparungen
(infolge verbesserten Brems- und Beschleunigungsverhaltens) bei
automatisiertem „eco-driving“ (keine Platoons) in Höhe von 4-10Prozent, bei
teilweise manuell gesteuerten Platoons in Höhe von 6-10Prozent und bei
vollautomatisierten Platoons bei über 10Prozent



Klima-/Umwelteffekte



\section{Infrastrukturen für Autonomes Fahren}
Die Entwicklung der Zukunftstechnologien und des Markthochlaufs von autonomen Fahrzeugen (Level 4, 5) erfordert neben der Weiterentwicklung von Fahrzeugtechnologien für Autonomes Fahren (Level 5) auch materielle (5G, Datenstandards,)
immaterielle (z.B. Software/AI-Kompetenzen) und institutionelle Infrastrukturen
(rechtlich-regulativer Rahmen) als Erfolgsvoraussetzungen
\subsection{Materielle Infrastrukturen: Datenübertragung (5G), Datenstandards}
Als wesentliche technische Voraussetzungen für autonomes Fahren gelten Sensortechnologien, Software-Algorithmen und hochauflösende Karten. Für letzteres
haben die deutschen Hersteller durch den Kauf des Kartenanbieters Here bereits
die Voraussetzungen geschaffen. Gleichzeitig ist jedoch auch die Datenkommunikation zwischen Fahrzeug und Straßeninfrastruktur (Vehicle-to-Infrastructure) sowie der Fahrzeuge untereinander (Vehicle-to-Vehicle) und zu großen Rechenzentren (z.B. der Hersteller) fundamental. Das autonom fahrende Auto muss zu anderen
Fahrzeugen eine Verbindung herstellen und ihnen Dinge mitteilen können, die es
selbst gelernt hat.102 Reaktionsschnelle Mobilfunknetze (5G) sowie Datenstandards
für einen breitbandigen und sicheren Datenaustausch sind hierfür essentiell.103
So produzieren autonome Fahrzeuge exponentiell ansteigende Mengen an Daten
durch ihre Sensorik wie Laser- (Lidar), Radarsensoren und Kameras, die sich auf
4.000 Gigabyte täglich summieren können. In cloudbasierten Datenpools werden
diese Datenströme der autonom fahrenden Autos in Echtzeit in Datensätze verwandelt und an das Auto zurückübermittelt, damit es seine gesamte Umgebung
erkennt. Dabei spielen Anwendungen der Künstlichen Intelligenz (KI) und des
maschinellen Lernens eine entscheidende Rolle.104
Gleichzeitig stellen 5G-Funkkomponenten für autonome Fahrzeuge eine
zusätzliche Schutzebene bereit, indem diese mit Fahrzeugen im näheren Umfeld
und der Infrastruktur am Straßenrand zuverlässig und schnell kommunizieren,
z.B. wenn die Sensoren keine Sichtverbindung haben oder nachteilige Wetterbedingungen herrschen.105 Darüber hinaus könnten 5G-Netze mit geringen Latenzzeiten auch die Fernsteuerung autonomer Fahrzeuge ermöglichen. Flottenbetreiber
von autonomen Fahrdiensten könnten dann in Fällen, in denen das Fahrzeug überfordert ist, mittels Fernsteuerung eingreifen. Allerdings dürften künftig sicherheitskritische Funktionen von autonomen Fahrzeugen nicht auf Mobilfunkverbindungen angewiesen sein. So gibt etwa die Google-Tochter Waymo an, dass ihre
Fahrzeuge auch ohne eine ständige Verbindung sicher funktionieren würden.
Prinzipiell sieht auch das Bundesverkehrsministerium 5G in der Mobilität als
zwingende Infrastrukturvoraussetzung,106 wobei das geplante 5G-Netz für zahlreiche Zukunftstechnologien ein Engpassfaktor darstellt, darunter das Internet
of Things (IoT).107 Heutige Festnetz-, Mobilfunk- und Satellitennetze sind nicht
für Datenaufkommen, Reaktionsgeschwindigkeiten und die Versorgungssicherheit des Autonomen Fahrens ausgelegt. Automobilhersteller wie VW, Daimler und
BMW haben sich bereits 2016 in der 5G Automotive Association (5GAA) zusammengeschlossen, um die notwendige Standardisierung für vernetzte und autonome
Fahrzeuge voranzutreiben. Allerdings gibt es selbst zwischen den in der 5GAA versammelten Herstellern noch Wettstreite um technische Konzepte und Kommunikationsstandards.108 Prognosen gehen daher davon aus, dass die Hochlaufphase für
5G sich erst ab 2023 stark intensiviert und dann bis 2028 auch alle regulatorischen
Fragen abgeschlossen sein werden.109
Insgesamt sind 5G-Netze für Autonomes Fahren und das vernetzte Fahrzeug
von großer Bedeutung, da dadurch erst sichere kommerzielle Anwendungen von
vernetzten und autonomen Fahrzeugen möglich sind. Entsprechend muss deren
Aufbau beschleunigt und weiter intensiviert werden. Die Investitionen in den
5G-Aufbau werden in einer 2016 veröffentlichten Studie für die EU-Kommission
bis 2025 auf rund 56 Mrd. Euro veranschlagt. Dieselbe Studie prognostiziert
im Gegenzug bis zu 113 Mrd. Euro zusätzliche Umsätze durch 5G in vier Sektoren, darunter auch Automotive, sowie die Schaffung von 2,3 Mio. Arbeitsplätzen
europaweit.11



\chapter{Mobilitätsdienstleistungen und deren besondere Infrastrukturvoraussetzunge}
\section{Konzept, Klima-/Umweltrelevanz, Markttrends}
Mobilitätsdienstleistungen sind ein zentrales Zukunftsfeld für Automobilhersteller,
Digitalplayer wie Google, Start-ups wie Uber sowie von Städten und Kommunen
gleichermaßen. Die Vision eines «Mobility as a Service» (MaaS) beschreibt die Vision
einer bruchlosen, hoch vernetzten Reise- bzw. Mobilitätskette über verschiedene
Verkehrsträger hinweg: von der intermodalen Routenplanung über die Buchung on
Demand und der Bezahlung bis hin zur Abwicklung der Fahrten.123 Darüber hinaus
können noch weitere Services wie Parkplatzdienste, Charging-Dienste oder Entertainment-Dienste hinzugezählt werden (vgl. für eine Übersicht Abb. 7).
Aus Kundensicht erweitern sich durch die neuen Mobilitätsangebote wie
Car-Sharing, Bike- und Ride-Sharing die Mobilitätsangebotsoptionen. War bis vor
kurzem ein eher monomodales Verkehrsverhalten dominant, das alle Fahrten entweder mit dem privaten Auto oder mit dem ÖPNV vorsah, entwickelt sich mittels
internetfähiger Smartphones ein intermodales oder gar multimodales Mobilitätsmuster: D.h. Kunden können ihren Mobilitätsbedarf individuell und schnell per
Fingertipp mit einer breiten Vielfalt von Mobilitätsangeboten befriedigen und dabei
nach ihren Präferenzen die günstigste, schnellste oder komfortabelste Kombination von Verkehrsmitteln auswählen. Durch multimodale Apps wie Moovel, Qixxit,
Ubigo oder Whim kann die Reisekette mit dem Smartphone künftig in Echtzeit
einfach und sicher geplant, gebucht und teilweise bereits bezahlt werden. Das
Fernziel ist Interoperabilität, also die bruchlose Übersicht, Verfügbarkeit und Buchbarkeit als kundenindividuell optimierter Mix aus allen Mobilitätsangeboten –
ÖPNV, Taxi, Car-Sharing, Ride-Hailing und Ride-Sharing, Leihwagen oder BikeSharing etc. – für die urbane Mobilität.124
Mobilitätsdienstleistungen bzw. MaaS begründen gleichzeitig neue Geschäftsmodelle auf Basis der Mobilitätseffizienz-, der Mobilitätszeit- und der Mobilitätssystem-Revolution. Für Automobilhersteller ergibt sich durch Mobilitätsdienstleistungen in Kombination mit dem Autonomen Fahren die Chance von neue
Geschäftsfeldern als Ersatz für die sich perspektivisch auflösenden bisherigen
kommerziellen Pfeiler, die wesentlich auf den Autokauf bzw. Autobesitz und
Freude am manuellen Autofahren angelegt waren.125 Gleichsam erweitert sich
jedoch auch das Wettbewerbsumfeld durch Digitalplayer wie Apple, Google
oder Alibaba und Baidu, die ihre Ökosysteme aus Kommunikations- und Entertainment-Services um Mobilitätsdienste erweitern wollen. Außerdem drängen
Start-ups wie Uber, Lyft, BlablaCar und andere mit innovativen Services einer
digitalen Mobility-on-Demand auf den Plan.
Hinzu kommt, dass künftig mit den oben skizzierten Trends von Elektromobilität und autonomen Fahrzeugen bzw. Robo-Taxis weitere innovative und kostengünstige Mobilitätsangebote marktreif sein werden. Diese werden nicht nur
das Spektrum von Mobilitätsdienstleistungen erweitern. Sie führen auch zu einem
Verschmelzen von öffentlichem und privatem Verkehr, weil das autonome Fahrzeug prinzipiell sowohl privat genutzt als auch als Taxi, Car-Sharing-Fahrzeug
oder Rufbus eingesetzt werden kann.126 Dabei stellt sich auch die Frage, ob durch
die sich entwickelnden autonomen Fahrdienste der ÖPNV kannibalisiert wird
oder ob diese einen positiven Beitrag zur Minderung des Autoverkehrs bzw. der
Klimabelastung des Verkehrs leisten können.

\subsection{Umwelt-/Klimarelevanz}
Die lokale bzw. regionale Ebene wird eine wichtige Rolle bei der Ausgestaltung
der Mobilitätsdienstleistungen der Zukunft spielen und damit auch bei der Frage,
ob die neuen Angebote einen positiven Beitrag für den Klimaschutz im Verkehrsbereich erbringen können.127 Städte und Kommunen sind nicht nur mit
der Verkehrsplanung, den baulichen Infrastrukturen von Straßen und Parkplätzen und dem Betrieb des öffentlichen Personennahverkehrs betraut. Sie können
auch über die Ausgestaltung von integrierten Mobilitätsdienstleistungen mitentscheiden. Dabei ist der verkehrsplanerische Handlungsdruck in urbanen Gebieten
aufgrund zunehmender Flächenknappheit bei einem gleichzeitig wachsenden
Verkehr und Mobilitätsbedürfnis der Menschen hoch.
Eine zentrale Bedeutung kommt den Robo-Taxis bzw. autonomen Fahrangeboten zu, bei denen noch nicht klar ist, ob sie in der Gesamtbilanz positive
oder negative Umwelt- und Klimaeffekte haben werden (vgl. Kapitel 4). So werden zu Beginn der 2020er Jahre erste autonom fahrende Shuttles auf deutschen Straßen unterwegs sein und sich wahrscheinlich dynamisch entwickeln.
Studien rechnen bis zum Jahr 2030 damit, dass bereits 37 Prozent des PKWVerkehrsaufkommens autonom bzw. in geteilten Systemen (Car Pooling, RideSharing) absolviert werden könnten.128 Es ist damit zu rechnen, dass die
Automatisierung des Fahrens erhebliche Auswirkungen auf das Preisgefüge und
damit die Nutzung der Mobilitätsservices besitzen wird.129 Robo-Taxen könnten
Berechnungen zufolge bis zu 60 Prozent günstiger betrieben werden als konventionelle.130 In den Möglichkeiten der Kombination von autonomen Taxen und
Sharing-Konzepten liegt dabei enormes Potenzial für die Reduzierung des Fahrtaufkommens. Unter heute geltenden Regulierungsvorschriften (u.a. Personenbeförderungsgesetz, gesetzliche Regelungen für autonomes Fahren) ist allerdings
unklar, ob und wie dieses Potenzial gehoben werden kann.131 Es wird hierbei
wesentlich auch auf eine umweltgerechte Flankierung und Ausrichtung der politischen Regulation bzw. Anreizstrukturen ankommen, insbesondere auch auf
Ebene der Kommunen. Dabei müssen autonome Fahrangebote in ein multimodales Verkehrssystem integriert werden.