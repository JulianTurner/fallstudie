\chapter{Einleitung}
Durch die voranschreitende technische Entwicklung in Industrie, Gewerbe und Landwirtschaft verändert und belastet der Mensch zunehmend die Umwelt.
Diese Belastungen der Umwelt können viele Ursachen haben,
möglicherweise sind bessere Lösungen technisch nicht umsetzbar,
gesetzlich nicht erlaubt oder wirtschaftlich unattraktiv.
Der aktuelle menschliche Lebensstandard zeichnet sich durch einen
hohen Energiebedarf, viele verschiedene global industriell gefertigte Produkte
und einen hohen Verkehrsaufkommen aus.

Die Umweltbelastung entsteht auf verschieden Ebenen, die sich in ihrer Form und Gegebenheit unterscheiden.
Es gibt energetische Belastungen, wie Strahlen, Lärm und Erschütterungen.
Belastungen durch feste Stoffe wie Abfälle welche durch Bau und Abbruch entstehen,
Abfälle aus Produktionen und
Abfälle aus der Gewinnung von Bodenschätzen.

Auch Flüssigkeiten können die Umwelt belasten.
Diese für die Umwelt schädlichen Flüssigkeiten können
in der Produktion durch Chemie Fabriken,
durch Reste von Medikamenten in Urin oder durch
Unfälle bei der Gewinnung von Rohöl in die Umwelt gelangen.
Neben den flüssigen Schadstoffen sind auch gasförmige Schadstoffe eine starke Belastung für die Umwelt.

Diese gasförmige Schadstoffe treten in Form von Luftschadstoffen und Feinstaub auf.
Luftschadstoffe können durch verschiedene Arten entstehen, die sowohl natürlichen Ursprung aber vor allem
durch menschliches Handeln entstehen.
Neben der Erzeugung von Energie trägt auch das hohe Verkehrsaufkommen zum großen Teil der Luftverschmutzung bei.
Diese Arten von Luftverschmutzung können ebenso verantwortlich für Krankheiten und vorzeitigen Tod von Menschen sein.
Gerade Kinder können davon betroffen sein, da ihr Körper noch nicht vollständig ausgewachsen ist,
oder ältere Personen dessen Immunsystem nicht mehr gut funktioniert.
Feinstaub kann in den Körper eindringen und schwerwiegende Krankheiten auslösen.
Durch unsachgemäße Wiederverwertung können Luftverschmutzungen entstehen,
wie zum Beispiel die Verbrennung von
Stromkabeln um das Kupfer aus der Isolierung zu trennen,
Tierhaltung sowie durch den Einsatz von Pestiziden in der Landwirtschaft.
Ein weiterer Faktor der Luftverschmutzung sind Verbrennungen von fossilen Kraftstoffen und Biomasse.

Die aktuell rasante Entwicklung im Verkehr durch die technische Realisierung von ersten autonomen Kraftfahrzeugen
erschließt die Möglichkeit, den Individualverkehr erheblich zu verändern.

Gerade durch diese Möglichkeiten der technischen Realisierung können von verschiedenen Personenkreisen
vielfältige Chancen wie ein geringerer Ausstoß von Feinstaub gesehen werden.

\section{Ausgangssituation}
Die Folge von globaler Erwärmung sowie die Veränderung des Klimas sind allgegenwärtig und betreffen den gesamten Globus.
Regionale Veränderungen von Wetter fallen immer extremer aus.
Starke unregelmäßige Wetterereignisse führen zu materiellen Schäden an Gebäuden und Infrastrukturen, vor allem zu nennenswerten Ertragsverlusten in der Landwirtschaft.
Die globale Erderwärmung steigt kontinuierlich an.
% Benötigt werden Maßnahmen, die sich an unsere Umwelt anpassen kann, 
% und weitreichende Anstrengungen zur Reduzierung von Umweltbelastungen durch menschliches Handeln. 
Stark betroffen sind Entwicklungs- und Schwellenländer, die mit Überschwemmungen oder Hitzewellen zu kämpfen haben.

% Die Resultate der Umweltkatastrophen sind verehrend.

Luftschadstoffe haben massive negative Auswirkungen auf unsere Gesundheit.
Viele Menschen sind Feinstaubpartikeln ausgesetzt, die als gesundheitsschädlich für den Menschen eingestuft sind.
Die Reduktion von Feinstaub ist daher ein wichtiger Teil der Ausgangssituation, die für eine Begrenzung der Erderwärmung unternommen werden müssen.

Deswegen werden in dieser Fallstudie verschiedene Einsparpotenziale für Umweltbelastungen durch autonome Kraftfahrzeuge untersucht.


\section{Ziel der Arbeit}
Das Ziel dieser Arbeit ist die folgende Forschungsfrage nach dem aktuellen Stand der Technik zu beantworten:
Welche Auswirkungen haben Kraftfahrzeuge auf die Umwelt und wie kann autonomes Fahren die negativen Auswirkungen reduzieren?
Mithilfe von frei zugänglicher Literatur soll diese Frage bearbeitet werden.
Folgende Hypothese wird untersucht:
Je mehr Kraftfahrzeuge autonom fahren, desto geringer fällt die Feinstaubbelastung durch Kraftfahrzeuge aus.

\section{Aufbau der Arbeit}
Der Aufbau dieser Arbeit besteht aus mehreren Teilen.
Zuerst wird ein Überblick über die technischen Teilsysteme von Kraftfahrzeugen gegeben und wie diese
in Klassen eingeordnet werden.
Da autonomes fahren eine aktuelle Entwicklung ist, wird in einem Teil erläutert in welchen Stufen die Fähigkeiten der Kraftfahrzeuge eingeordnet werden können.
Ferner werden negative Belastungen auf die Umwelt durch Kraftfahrzeuge eingegangen und wie diese entstehen.
Untersucht wird auch Reduzierung von negativen Einflüssen und an welchen Stellen sie möglich ist.

\newpage
\chapter{Umweltbelastung}
Durch die voranschreitende technische Entwicklung in Industrie, Gewerbe und Landwirtschaft verändert und belastet der Mensch zunehmend die Umwelt.
Umweltbelastungen können viele Ursachen haben, möglicherweise sind bessere Lösungen nicht umsetzbar oder wirtschaftlich nicht attraktiv.
Die Umweltbelastung entsteht auf verschieden Ebenen, die sich in ihrer Gegebenheit unterscheiden.
Es gibt energetischen Belastungen, wie Strahlen, Lärm und Erschütterungen.
Es gibt Umweltbelastungen durch feste Stoffe wie Abfälle die durch Bau und Abbruch entstehen, Abfälle aus Produktionen und Abfälle aus der Gewinnung von Bodenschätzen.
Auch flüssige Stoffe belasten die Umwelt. Sie entstehen durch Chemie Fabriken, Reste von Medikamenten die durch den Urin in das Abwasser gelangen oder durch Umweltkatastrophen bei der sich das Wasser mit andren Stoffen vermischt.
Ein Beispiel hierfür könnte ein Erbebben sein, welches ein Atomkraftwerk beschädigt und radioaktives Wasser ausläuft.
Die größte Umweltbelastung für die Umwelt ist aber die gasförmige Verschmutzung, welche die Luft verschmutzt.
Die Gasformringe Verschmutzung welche die Luft verunreinigt ist die eine von den größten Belastungen für die Umwelt.
Durch unsachgemäße Wiederverwertung können Luftverschmutzungen entstehen, wie zum Beispiel die Verbrennung von
Stromkabeln um das Kupfer aus der Isolierung zu trennen.
Die Luftverschmutzung ist ebenso verantwortlich für Krankheiten und vorzeitigen Tod von Menschen.
Feinstaub kann in den Körper eindringen und schwerwiegende Krankheiten auslösen.
Luftverschmutzung entsteht bei Tierhaltung sowie durch den Einsatz von Pestiziden.

Die Hauptursache sind Abgase die bei der Verbrennung von fossilen Kraftstoffen entstehen.
Ein großer Träger bei der Verbrennung von fossilen Kraftstoffen sind Kraftfahrzeuge.

\section{Kraftfahrzeuge}
Das deutsche Straßenverkehrsgesetz beschreibt Kraftfahrzeuge als Landfahrzeuge, die durch Maschinenkraft bewegt werden, aber nicht an Bahngleise gebunden sind.
\cite{str}

Da Kraftfahrzeuge Landfahrzeuge sind gehören Flugzeuge, Schiffe oder Boote nicht zu der Kategorie, obwohl sie durch Maschinenkraft bewegt werden.
Auch Züge oder Trambahnen gehören nicht in in die Kategorien, da sie an Bahngleise gebunden sind.

\subsection{Teilsysteme von Kraftfahrzeugen}
Moderne Kraftfahrzeuge werden aus folgenden Teilsysteme gebildet:
\begin{itemize}
	\item Antriebseinheit
	\item Energieübertragungseinheit
	\item Stütz- und Trageeinheit
	\item Steuerungs- und Regelungseinheit
	\item Arbeitseinheit
\end{itemize}


\begin{figure}[!ht]
	\caption{Teilsysteme des Kraftfahrzeugs}
	\includegraphics[scale=0.1]{assets/figures/Teilsysteme des Kraftfahrzeugs.jpg}
	\begin{flushleft}
		Quelle: Westermann S. 19
	\end{flushleft}
	\label{fig:birds}
\end{figure}


\subsubsection{Antriebseinheit}
Die Antriebseinheit wandelt die zugeführte Energie in die erforderliche Antriebsenergie um.
Diese Umwandlung wird im Motor durchgeführt.
Hauptsächlich werden Elektro- und Verbrennungsmotoren eingesetzt.

Verbrennungsmotoren unterscheiden sich von Elektromotoren durch ihre Energieerzeugung.
Die Energieerzeugung wird durch die Verbrennung von Kraftstoff erzeugt.
Dazu wird ein Kraftstoff-Luft-Gemisch in einem Brennraum mit Kolben zur Verbrennung verwendet.
Durch die Verbrennung steigt der Druck im Brennraum stark an und bewegt einen Kolben.

\subsubsection{Arbeitseinheit}
Die Arbeitseinheit ist die Verbindung zwischen den Antriebsrädern und der Fahrbahn.
Durch die Bewegung der Antriebsrädern wird das Kraftfahrzeug in Bewegung gesetzt.

\subsubsection{Energieübertragungseinheit}
Die Energieübertragungseinheit leitet die Energie in der geforderten Bewegungsart und Bewegungsgeschwindigkeit zur Arbeitseinheit weiter.

Energieübertragungseinheiten sind Baugruppen einer Maschine, die zur Übertragung von Energie in benötigt werden.
Beispiel hierfür sind Kabel die Elektrische Energie leiten oder Wellen, Zahnräder oder Riemen die mechanische Energie weiterleiten.

\subsubsection{Stütz- und Trageeinheit}
Stütz- und Trageeinheit der Rahmen oder der selbsttragende Aufbau des Kraftfahrzeuges haben hauptsächlich die Aufgabe, die Teilsysteme aufzunehmen und zu einer Einheit zu verbinden.


\subsubsection{Steuerungs- und Regelungseinheit}
Die Steuerungs- und Regelungseinheit beeinflusst die Stoff- und Energieumsetzung durch Informationsverarbeitung

\subsubsection{Steuerungseinheit}
Bei der Steuerungseinheit werden verschiedene Eingangsgrößen durch das System in eine oder mehrere Ausgangsgrößen verändert.
Beispiele für Steuerungen sind:
\begin{itemize}
	\item Klimaanlage: Es wird eine Solltemperatur eingestellt.
	      Die Klimaanlage kühlt konstant.
	      Die Klimaanlage kühlt solange mit dieser eingestellten Temperatur solange sie nicht verändert wird.
	      Die Umgebungstemperatur wird nicht berücksichtigt.
	\item Licht: Der Schalter wird betätigt und das Licht wird eingeschaltet.
	      Das Licht bleibt permanent eingeschaltet.
	      Das Licht geht erst aus wenn der Schalter ausgeschaltet wird.
	      Das Umgebungslicht wird nicht berücksichtigt.
\end{itemize}
\subsubsection{Regelungseinheit}
Bei einer Regelungseinheit werden die Eingangsgrößen mit einem Sollwert verglichen und so lange angepasst bis der Sollwert erreicht wird.
Beispiele für Regelungen sind:
\begin{itemize}
	\item Klimaautomatik: es wird eine Solltemperatur eingestellt.
	      Es wird gemessen wie warm oder wie Kalt die Temperatur ist.
	      Sollte die Temperatur unter der Solltemperatur liegen, wird die Klimaautomatik auf Heizen gestellt.
	      Sollte die Temperatur über der Solltemperatur liegen, wird die Klimaautomatik auf Kühlen gestellt.
	\item Lichtautomatik: Es gibt eine Schwelle bei der das Licht eingeschaltet werden soll.
	      Es gemessen wie hell das Umgebungslicht ist.
	      Sollte das Umgebungslicht zu gering sein wie zum Beispiel im Tunnel oder bei Dämmerung wird das Licht eingeschaltet.
	      Sobald das Umgebungslicht wieder hell genug ist zum Beispiel beim verlassen des Tunnels oder bei Sonnenaufgang, wird das Licht wieder ausgeschaltet.
\end{itemize}

\subsection{Fahrzeugklassen}

Kraftfahrzeuge können Bauartbedingt in Kategorien eingeordnet werden.
Die EU Kommission hat hierfür acht Klassen definiert.\footnote{VERORDNUNG (EU) Nr. 678/2011 DER KOMMISSION
	vom 14. Juli 2011, TEIL A ABS.1 - https://eur-lex.europa.eu/eli/reg/2011/678/oj?locale=de}

\begin{itemize}
	\item Klasse L: Leichte ein- und zweispurige Kraftfahrzeuge
	\item Klasse M: Vorwiegend für die Beförderung von Fahrgästen und deren Gepäck ausgelegte und gebaute Kraftfahrzeuge
	\item Klasse N: Vorwiegend für die Beförderung von Gütern ausgelegte und gebaute Kraftfahrzeuge
	\item Klasse O: Anhänger, die sowohl für die Beförderung von Gütern und Fahrgästen als auch für die Unterbringung von Personen ausgelegt und gebaut sind
	\item Klasse S: unvollständige Fahrzeuge, die der Unterklasse der Fahrzeuge mit besonderer Zweckbestimmung zugeordnet werden soll
	\item Klasse R: Anhänger, die in der Land- und Forstwirtschaft verwendet werden
	\item Klasse S: Maschinen, die in der Land- und Forstwirtschaft zum Einsatz kommen und gezogen werden
	\item Klasse T: Zugmaschinen, die in der Land- und Forstwirtschaft verwendet werden wie Traktoren
	\item Klasse C: Zugmaschinen, die in der Land- und Forstwirtschaft verwendet werden und auf Ketten laufen wie ein Bagger
\end{itemize}

Die relevantesten Klassen sind M und N.
\vspace{0.5cm}

\subsection{Klasse M}
In der Klasse M werden Kraftfahrerzeuge eingeordnet die für die Beförderung von Personen und Gepäck zuständig sind und mindestens 4 Räder haben sowie eine Hochgeschwindigkeit von über 25 \ac{kmh} haben.
\newline
Die Klasse M spaltet sich in 3 Unterklassen auf:
\begin{itemize}
	\item {Klasse M1}
	\item {Klasse M2}
	\item {Klasse M3}
\end{itemize}
\subsubsection{Klasse M1}
Kraftfahrzeuge der Klasse M1 haben über die Eigenschaften der Klasse M noch folgende weitere Eigenschaften:
\begin{itemize}
	\item {nicht mehr als 8 Sitzplätze und 1 Platz für den Fahrer}
	\item {keine Stehplätze}
	\item {zulässiges Gesamtgewicht von maximal 3,5 \ac{t}}
\end{itemize}

In der Klasse M1 sind Kraftfahrzeuge wie Personenkraftwagen(Limousine, Cabrio) und Wohnmobile zu finden.

\subsubsection{Klasse M2}
Kraftfahrzeuge der Klasse M2 haben über die Eigenschaften der Klasse M noch folgende weitere Eigenschaften:
\begin{itemize}
	\item {mehr als 8 Sitzplätze}
	\item {zulässiges Gesamtgewicht von maximal 5 \ac{t}}
\end{itemize}

In der Klasse M2 sind Kraftfahrzeuge wie ein Eindecker-Bus bis 5 \ac{t} oder ein Doppeldecker-Bus bis 5 \ac{t} zu finden.

\subsubsection{Klasse M3}

Die dritte Unterklasse der Klasse M ist M3.

Kraftfahrzeuge der Klasse M3 haben über die Eigenschaften der Klasse M noch folgende weitere Eigenschaften:
\begin{itemize}
	\item {mehr als 8 Sitzplätze}
	\item {zulässiges Gesamtgewicht von über 5 \ac{t}}
\end{itemize}

In der Klasse M3 sind Kraftfahrzeuge wie ein Eindecker-Bus über 5 \ac{t} oder Doppeldecker-Bus über 5 \ac{t} zu finden.

\subsection{Klasse N}
In der Klasse N werden Kraftfahrerzeuge eingeordnet die für die Beförderung von Gütern zuständig sind und mindestens 3 Räder haben sowie ein zulässiges Gesamtgewicht von über 1 \ac{t} haben.
Die Klasse N spaltet sich in 3 Unterklassen auf:
\begin{itemize}
	\item {Klasse N1}
	\item {Klasse N2}
	\item {Klasse N3}
\end{itemize}

\subsubsection{Klasse N1}
Fahrzeuge zur Güterbeförderung mit einer zulässigen Gesamtmasse bis zu 3,5 \ac{t}.
In der Klasse N1 sind Kraftfahrzeuge die in dicht besiedelten Regionen gut zurecht kommen, wie Paketzusteller oder Fahrzeuge der Post.


\subsubsection{Klasse N2}
Fahrzeuge zur Güterbeförderung mit einer zulässigen Gesamtmasse von zu 3,5 \ac{t} bis 12 \ac*{t}.
In der Klasse N2 sind Kraftfahrzeuge die regional Güterbefördern, dies könnten Kraftfahrzeuge die Waren aus einem Zentrallager in die Filialen transportieren.
Diese Kraftfahrzeuge sind darauf ausgelegt hunderte Kilometer zurückzulegen.


\subsubsection{KLasse N3}
Fahrzeuge zur Güterbeförderung mit einer zulässigen Gesamtmasse von mehr als 12 \ac{t}.
In der Klasse N3 sind Kraftfahrzeuge die überregional Güterbefördern, wie ein Kraftfahrzeug das große Mengen an Ladung fassen kann und darauf ausgelegt sind tausende Kilometer zurückzulegen.

\subsection{Autonomes Fahren}
Beim autonomen Fahren, fährt ein Kraftfahrzeug Verwaltungsgemäß selbständig.
Für Kraftfahrzeuge wurden von der \ac{SAE} Institut in der Norm SAE J3016\footnote{SAE J3016\textunderscore202104 - https://www.sae.org/standards/content/j3016\textunderscore202104} Automatisierungsgrade definiert.
\begin{itemize}
	\item Stufe 0 (Keine Automation)
	\item Stufe 1 (Assistenzsysteme)
	\item Stufe 2 (Teilautomatisierung)
	\item Stufe 3 (Bedingte Automatisierung)
	\item Stufe 4 (Hochautomatisierung)
	\item Stufe 5 (Vollautomatisierung)
\end{itemize}
\subsubsection{Was passiert in den Stufen?}
Die Stufen unterscheiden sich im wesentlichen nur durch die Anzahl der Automatisierungsgrade.

\vspace{0.5cm}

In der Stufe 0 (Keine Automation):
\begin{itemize}
	\item keine Assistenzsysteme
	\item \ac{Kfz} kann keine Fahraufgaben übernehmen
	\item Fahrer ist unter permanenter Kontrolle
\end{itemize}

\vspace{0.5cm}

In der Stufe 1 (Assistenzsysteme):
\begin{itemize}
	\item Assistenzsysteme wie ein System zur automatischen Geschwindigkeitsregelung oder eine Berganfahrhilfe
	\item Fahrer hat eine passive Unterstützung bei Fahraufgaben
	\item \ac{Kfz} kann keine Fahraufgaben übernehmen
	\item Fahrer muss jeder Zeit die Fahraufgabe übernehmen können
\end{itemize}

\vspace{0.5cm}

In der Stufe 2 (Teilautomatisierung):
\begin{itemize}
	\item Assistenzsysteme, wie der Spurführungsassistent oder Stauassistent
	      \begin{itemize}
		      \item automatisch bremsen
		      \item automatisch beschleunigen
		      \item automatisch lenken
	      \end{itemize}
	\item \ac{Kfz} kann Fahraufgaben teilautomatisiert übernehmen
	\item Fahrer kann sich für kurze Zeit von den Fahraufgaben abwenden
	\item Fahrer muss jeder Zeit die Fahraufgabe übernehmen können
\end{itemize}

\vspace{0.5cm}

In der Stufe 3 (Bedingte Automatisierung):
\begin{itemize}
	\item hochautomatisierte Assistenzsysteme
	\item \ac{Kfz} kann Fahraufgaben unter bestimmten Voraussetzungen vollständig übernehmen
	\item Fahrer kann sich unter bestimmten Voraussetzungen dauerhaft von den Fahraufgaben abwenden
	\item Fahrer muss innerhalb wenigen Sekunden die Fahraufgabe übernehmen können
\end{itemize}

\vspace{0.5cm}

In der Stufe 4 (Hochautomatisierung):
\begin{itemize}
	\item hochautomatisierte Assistenzsysteme
	\item \ac{Kfz} kann Fahraufgaben in hochkomplexen Verkehrssituationen vollständig übernehmen
	\item Fahrer dauerhaft von den Fahraufgaben abwenden
	\item Fahrer muss fahrtüchtig sein, um im Bedarfsfall die Fahraufgabe übernehmen zu können
\end{itemize}

\vspace{0.5cm}

In der Stufe 5 (Vollautomatisierung):
\begin{itemize}
	\item hochautomatisierte Assistenzsysteme
	\item \ac{Kfz} übernimmt alle Fahraufgaben vollständig
	\item Fahrer ist nicht erforderlich
	\item alle Personen im Wagen werden zu Passagieren
\end{itemize}

\subsection{Autonome Kraftfahrzeuge}
Sind Fahrzeuge die nicht nur automatisch fahren sondern von einem System gesteuert werden.
Somit sind diese Fahrzeuge aus Sicht der Nutzenden autonom.

Während manche in der Verbreitung autonomer Fahrzeuge die Lösung vieler Probleme sehen können,
vermuten andere eine Verschlechterung der Verkehrs- und Umweltlage.

Die Bedeutung von autonomen Fahrzeugen, hängt sowohl von der technischen Komplexität sowie von politischen Regulierung ab.

In welchem Maß die Level 5 Systeme im Straßenverkehr teilnehmen entscheidet vorerst der gesetzliche Rahmen.
Dies ist wiederum abhängig wie der Verkehr von morgen aussehen soll.
\section{Umweltbelastungen durch Kraftfahrzeuge}
Kraftfahrzeuge belasten die Umwelt auf verschiedene Arten. Hierunter fallen
die Erzeugung von Rohstoffen für Materialien die für die Produktion von Kraftfahrzeugen benötigt werden,
die tatsächliche Produktion von Kraftfahrzeugen,
der Betrieb von Kraftfahrzeugen,
sowie die Entsorgung von Kraftfahrzeugen.

Gerade aber der Betrieb von Kraftfahrzeugen belastet die Umwelt durch die verschieden Arten von Schadstoffen.
Unterschieden wird durch die Art der Belastung,
giftige Verbrennungsabgase die durch die Verbrennung entstehen,
Feinstaub der sowohl durch die Verbrennung und auch durch den Abrieb von Reifen und Bremsen freigesetzt wird
und die Infrastruktur der Kraftfahrzeuge für Straßen, Parkplätze und andere Einrichtungen.

\subsection{Verbrennungsabgase}
Die größten Anteile der giftigen Schadstoffe die durch die Verbrennung von Kraftstoff entstehen sind:
\begin{itemize}
	\item {\ac{CO}}
	\item {\ac{NO}}
	\item unverbrannte Kohlenwasserstoffe (HC)
\end{itemize}
Die Abgase strömen nach der Verbrennung im Verbrennungsraum durch die Abgasanlage in die Umwelt.
Es gibt auch ungiftige Stoffe die durch die Verbrennung abgegeben werden wie \ac{zb} Wasser und \ac{CO2}.
Die Menge der Abgase die durch die Abgasanlage strömen ist von der Größe des Motors sowie dem Lastzustand des Motors abhängig.

\subsection{Feinstaub}
Feinstaub ist ein fester oder flüssiger Stoff der nicht sofort zu Boden sinkt. Bis der Feinstaub zu Boden sinkt, und we sich Feinstaub verbreitet ist neben der Art des Feinstaubes auch noch die Wetterlage entscheidend.
Bei Trockenheit kann sich Feinstaub gut ausbreiten und verweilt länger in der Luft, hohe Luftfeuchtigkeit beeinträchtigen die Ausbreitung.
Feinstäube werden als Particle Matter (PM, zu deutsch Stoffteilchen) bezeichnet. Diese Luftschadstoffe sind gesundheitsschädlich. \footnote{Westermann S. 327}

Es wird Unterschieden zwischen Feinstaub der aus natürlichen Quellen entstanden ist und Feinstaub der durch menschliches Handeln entstanden ist.

\subsubsection{Feinstaub aus natürlichen Quellen}
Natürlich Feinstäube sind ohne das Handeln durch den Menschen entstanden.
Quellen für natürlichen Feinstaub sind:
\begin{itemize}
	\item Vulkane
	\item Wald- und Buschbrände
	\item Pollen
	\item Sporen
\end{itemize}


\subsubsection{Feinstaub durch menschliches Handeln}
Feinstaub der durch menschliches Handeln entstanden ist wird auch anthropogener Feinstaub genannt.
Quellen für Feinstaub durch menschliches Handeln sind:
\begin{itemize}
	\item vom Straßenverkehr durch Verbrennung und Abrieb
	\item Verbrennungsabgase von Kraftwerken und Müllverbrennungsanlagen
	\item Brände von Gebäuden
	\item Industrieprozesse wie die Stahlerzeugung
\end{itemize}

Zur Verbesserung der Luftreinhaltung können Kommunen und Städte Umweltzonen einrichten und Fahrverbote festlegen.
Das befahren einer Umweltzone ist dann nur mit einer entsprechenden Kennzeichnung des Fahrzeuges möglich, die man bei der zuständigen Behörde erlangen kann.




\subsection{Infrastruktur}
Auch die Infrastruktur belastet die Umwelt, indem:
\begin{itemize}
	\item Parkplätze Flächen versiegeln
	\item Wälder abgeholzt werden um die Verkehrsanbindung zu verbessern
	\item Straßen vergrößert werden umd mehr Fahrzeuge zu ermöglichen
	\item starke Erhitzung durch Sonneneinstrahlung auf dunklen Verkehrswegen
	\item fehlende Bäume die Schatten spenden
\end{itemize}

\section{Umweltbelastung nach Bedingungen}
Die Umweltbelastung kann stark nach Betriebszuständen variieren.
So verbraucht ein Fahrzeuge das bergab fährt weniger Kraftstoff und stößt somit auch weniger Luftschadstoffe aus.
Die Umweltbelastung durch Luftschadstoffe hängt von folgendem ab:
\begin{itemize}
	\item dem Fahrverhalten des Fahrers, wie dem Beschleunigungsverhalten und der Fahrgeschwindigkeit ab
	\item der Effizienz des Fahrzeugs, je effizienter desto besser
	\item dem Gewicht des Fahrzeugs, je leichter desto weniger Gewicht muss beschleunigt und gebremst werden
	\item der Fahrstrecke, fährt das Fahrzeug eine Steigung wird mehr Kraftstoff benötigt
	\item dem Wetter, je nach Wind wird mehr oder weniger Kraftstoff benötigt
	\item dem Betriebszustand, wenn sich das Fahrzeug nicht im Betriebszustand befindet wird Energie verwendet um den Betriebszustand zu erreichen
\end{itemize}







\section{Assistenzsysteme}
\subsection{Keine Assistenzsysteme}
\subsection{Assistenzsysteme}
\subsection{Teilautomatisierung}
\subsection{Bedingte Automatisierung}
\subsection{Hochautomatisierung}
\subsection{Vollautomatisierung}

\chapter{Problemlösung durch Assistenzsysteme}
\section{}


% !!! hier kann man Beispiele aufführen, wie zb Bus, PKW, LKW; Traktor
% !!! Wie groß ist der Anteil der Kraftfahrzeuge in den Klassen in Deutschland?
% !!! Warum und wie blesatst jede Klasse die Umwelt?
% !!! PKW durch unnötige Kurzstecke
% !!! Sinnloses laufenlassen von LKW
% !!! Zustellungen von Gütern
% !!! Leerfahrten


% \subsection{Autonomes Fahren}

% !!! Was sind autonome fahrzeuge?
% !!! Welche 


% Beim autonomen Fahren, fährt ein \ac{Kfz} Verwaltungsgefäß selbständig.
% Für \ac{Kfz} wurden von der \ac{SAE} Institut in der Norm SAE J3016\footnote{SAE J3016\textunderscore202104 - https://www.sae.org/standards/content/j3016\textunderscore202104} Automatisierungsgrade definiert.
% \begin{itemize}
% 	\item Stufe 0 (Keine Automation)
% 	\item Stufe 1 (Assistenzsysteme)
% 	\item Stufe 2 (Teilautomatisierung)
% 	\item Stufe 3 (Bedingte Automatisierung)
% 	\item Stufe 4 (Hochautomatisierung)
% 	\item Stufe 5 (Vollautomatisierung)
% \end{itemize}
% \subsubsection{Was passiert in den Stufen?}
% Die Stufen unterscheiden sich im wesentlichen nur durch die Anzahl der Automatisierungsgrade.

% \vspace{0.5cm}

% In der Stufe 0 (Keine Automation):
% \begin{itemize}
% 	\item keine Assistenzsysteme
% 	\item \ac{Kfz} kann keine Fahraufgaben übernehmen
% 	\item Fahrer ist unter permanenter Kontrolle
% \end{itemize}

% \vspace{0.5cm}

% In der Stufe 1 (Assistenzsysteme):
% \begin{itemize}
% 	\item Assistenzsysteme wie \ac{GRA} oder eine Berganfahrhilfe
% 	\item Fahrer hat eine passive Unterstützung bei Fahraufgaben
% 	\item \ac{Kfz} kann keine Fahraufgaben übernehmen
% 	\item das \ac{Kfz} ist unter permanenter Kontrolle des Fahrers
% \end{itemize}

% \vspace{0.5cm}

% In der Stufe 2 (Teilautomatisierung):
% \begin{itemize}
% 	\item Assistenzsysteme, wie der Spurführungsassistent oder Stauassistent
% 	      \begin{itemize}
% 		      \item automatisch bremsen
% 		      \item automatisch beschleunigen
% 		      \item automatisch lenken
% 	      \end{itemize}
% 	\item \ac{Kfz} kann Fahraufgaben teilautomatisiert übernehmen
% 	\item Fahrer kann sich für kurze Zeit von den Fahraufgaben abwenden
% 	\item Fahrer muss jeder Zeit die Fahraufgabe übernehmen können
% \end{itemize}

% \vspace{0.5cm}

% In der Stufe 3 (Bedingte Automatisierung):
% \begin{itemize}
% 	\item hochautomatisierte Assistenzsysteme
% 	\item \ac{Kfz} kann Fahraufgaben unter bestimmten Voraussetzungen vollständig übernehmen
% 	\item Fahrer kann sich unter bestimmten Voraussetzungen dauerhaft von den Fahraufgaben abwenden
% 	\item Fahrer muss innerhalb wenigen Sekunden die Fahraufgabe übernehmen können
% \end{itemize}

% \vspace{0.5cm}

% In der Stufe 4 (Hochautomatisierung):
% \begin{itemize}
% 	\item hochautomatisierte Assistenzsysteme
% 	\item \ac{Kfz} kann Fahraufgaben in hochkomplexen Verkehrssituationen vollständig übernehmen
% 	\item Fahrer dauerhaft von den Fahraufgaben abwenden
% 	\item Fahrer muss fahrtüchtig sein, um im Bedarfsfall die Fahraufgabe übernehmen zu können
% \end{itemize}

% \vspace{0.5cm}

% In der Stufe 5 (Vollautomatisierung):
% \begin{itemize}
% 	\item hochautomatisierte Assistenzsysteme
% 	\item \ac{Kfz} übernimmt alle Fahraufgaben vollständig
% 	\item Fahrer ist nicht erforderlich
% 	\item alle Personen im Wagen werden zu Passagieren
% \end{itemize}



