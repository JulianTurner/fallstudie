\chapter{Hauptteil}
\section{Kraftfahrzeuge}
Das deutsche Straßenverkehrsgesetz beschreibt Kraftfahrzeuge als Landfahrzeuge, die durch Maschinenkraft bewegt werden, aber nicht an Bahngleise gebunden sind.
\cite{str}

Da Kraftfahrzeuge Landfahrzeuge sind gehören Flugzeuge, Schiffe oder Boote nicht zu der Kategorie, obwohl sie durch Maschinenkraft bewegt werden.
Auch Züge oder Trambahnen gehören nicht in in die Kategorien, da sie an Bahngleise gebunden sind.

\subsection{Teilsysteme von Kraftfahrzeugen}
Moderne Kraftfahrzeuge werden aus folgenden Teilsysteme gebildet:
\begin{itemize}
	\item Antriebseinheit
	\item Energieübertragungseinheit
	\item Stütz- und Trageeinheit
	\item Steuerungs- und Regelungseinheit
	\item Arbeitseinheit
\end{itemize}


\begin{figure}[!ht]
	\caption{Teilsysteme des Kraftfahrzeugs}
	\includegraphics[scale=0.1]{assets/figures/Teilsysteme des Kraftfahrzeugs.jpg}
	\begin{flushleft}
		Quelle: Westermann S. 19
	\end{flushleft}
	\label{fig:birds}
\end{figure}


\subsubsection{Antriebseinheit}
Die Antriebseinheit wandelt die zugeführte Energie in die erforderliche Antriebsenergie um.
Diese Umwandlung wird im Motor durchgeführt.
Hauptsächlich werden Elektro- und Verbrennungsmotoren eingesetzt.

Verbrennungsmotoren unterscheiden sich von Elektromotoren durch ihre Energieerzeugung.
Die Energieerzeugung wird durch die Verbrennung von Kraftstoff erzeugt.
Dazu wird ein Kraftstoff-Luft-Gemisch in einem Brennraum mit Kolben zur Verbrennung verwendet.
Durch die Verbrennung steigt der Druck im Brennraum stark an und bewegt einen Kolben.

\subsubsection{Arbeitseinheit}
Die Arbeitseinheit ist die Verbindung zwischen den Antriebsrädern und der Fahrbahn.
Durch die Bewegung der Antriebsrädern wird das Kraftfahrzeug in Bewegung gesetzt.

\subsubsection{Energieübertragungseinheit}
Die Energieübertragungseinheit leitet die Energie in der geforderten Bewegungsart und Bewegungsgeschwindigkeit zur Arbeitseinheit weiter.

Energieübertragungseinheiten sind Baugruppen einer Maschine, die zur Übertragung von Energie in benötigt werden.
Beispiel hierfür sind Kabel die Elektrische Energie leiten oder Wellen, Zahnräder oder Riemen die mechanische Energie weiterleiten.

\subsubsection{Stütz- und Trageeinheit}
Stütz- und Trageeinheit der Rahmen oder der selbsttragende Aufbau des Kraftfahrzeuges haben hauptsächlich die Aufgabe, die Teilsysteme aufzunehmen und zu einer Einheit zu verbinden.


\subsubsection{Steuerungs- und Regelungseinheit}
Die Steuerungs- und Regelungseinheit beeinflusst die Stoff- und Energieumsetzung durch Informationsverarbeitung

\subsubsection{Steuerungseinheit}
Bei der Steuerungseinheit werden verschiedene Eingangsgrößen durch das System in eine oder mehrere Ausgangsgrößen verändert.
Beispiele für Steuerungen sind:
\begin{itemize}
	\item Klimaanlage: Es wird eine Solltemperatur eingestellt.
	      Die Klimaanlage kühlt konstant.
	      Die Klimaanlage kühlt solange mit dieser eingestellten Temperatur solange sie nicht verändert wird.
	      Die Umgebungstemperatur wird nicht berücksichtigt.
	\item Licht: Der Schalter wird betätigt und das Licht wird eingeschaltet.
	      Das Licht bleibt permanent eingeschaltet.
	      Das Licht geht erst aus wenn der Schalter ausgeschaltet wird.
	      Das Umgebungslicht wird nicht berücksichtigt.
\end{itemize}
\subsubsection{Regelungseinheit}
Bei einer Regelungseinheit werden die Eingangsgrößen mit einem Sollwert verglichen und so lange angepasst bis der Sollwert erreicht wird.
Beispiele für Regelungen sind:
\begin{itemize}
	\item Klimaautomatik: es wird eine Solltemperatur eingestellt.
	      Es wird gemessen wie warm oder wie Kalt die Temperatur ist.
	      Sollte die Temperatur unter der Solltemperatur liegen, wird die Klimaautomatik auf Heizen gestellt.
	      Sollte die Temperatur über der Solltemperatur liegen, wird die Klimaautomatik auf Kühlen gestellt.
	\item Lichtautomatik: Es gibt eine Schwelle bei der das Licht eingeschaltet werden soll.
	      Es gemessen wie hell das Umgebungslicht ist.
	      Sollte das Umgebungslicht zu gering sein wie zum Beispiel im Tunnel oder bei Dämmerung wird das Licht eingeschaltet.
	      Sobald das Umgebungslicht wieder hell genug ist zum Beispiel beim verlassen des Tunnels oder bei Sonnenaufgang, wird das Licht wieder ausgeschaltet.
\end{itemize}

\subsection{Fahrzeugklassen}

Kraftfahrzeuge können Bauartbedingt in Kategorien eingeordnet werden.
Die EU Kommission hat hierfür acht Klassen definiert.\footnote{VERORDNUNG (EU) Nr. 678/2011 DER KOMMISSION
	vom 14. Juli 2011, TEIL A ABS.1 - https://eur-lex.europa.eu/eli/reg/2011/678/oj?locale=de}

\begin{itemize}
	\item Klasse L: Leichte ein- und zweispurige Kraftfahrzeuge
	\item Klasse M: Vorwiegend für die Beförderung von Fahrgästen und deren Gepäck ausgelegte und gebaute Kraftfahrzeuge
	\item Klasse N: Vorwiegend für die Beförderung von Gütern ausgelegte und gebaute Kraftfahrzeuge
	\item Klasse O: Anhänger, die sowohl für die Beförderung von Gütern und Fahrgästen als auch für die Unterbringung von Personen ausgelegt und gebaut sind
	\item Klasse S: unvollständige Fahrzeuge, die der Unterklasse der Fahrzeuge mit besonderer Zweckbestimmung zugeordnet werden soll
	\item Klasse R: Anhänger, die in der Land- und Forstwirtschaft verwendet werden
	\item Klasse S: Maschinen, die in der Land- und Forstwirtschaft zum Einsatz kommen und gezogen werden
	\item Klasse T: Zugmaschinen, die in der Land- und Forstwirtschaft verwendet werden wie Traktoren
	\item Klasse C: Zugmaschinen, die in der Land- und Forstwirtschaft verwendet werden und auf Ketten laufen wie ein Bagger
\end{itemize}

Die relevantesten Klassen sind M und N.
\vspace{0.5cm}

\subsection{Klasse M}
In der Klasse M werden Kraftfahrerzeuge eingeordnet die für die Beförderung von Personen und Gepäck zuständig sind und mindestens 4 Räder haben sowie eine Hochgeschwindigkeit von über 25 \ac{kmh} haben.
\newline
Die Klasse M spaltet sich in 3 Unterklassen auf:
\begin{itemize}
	\item {Klasse M1}
	\item {Klasse M2}
	\item {Klasse M3}
\end{itemize}
\subsubsection{Klasse M1}
Kraftfahrzeuge der Klasse M1 haben über die Eigenschaften der Klasse M noch folgende weitere Eigenschaften:
\begin{itemize}
	\item {nicht mehr als 8 Sitzplätze und 1 Platz für den Fahrer}
	\item {keine Stehplätze}
	\item {zulässiges Gesamtgewicht von maximal 3,5 \ac{t}}
\end{itemize}

In der Klasse M1 sind Kraftfahrzeuge wie Personenkraftwagen(Limousine, Cabrio) und Wohnmobile zu finden.

\subsubsection{Klasse M2}
Kraftfahrzeuge der Klasse M2 haben über die Eigenschaften der Klasse M noch folgende weitere Eigenschaften:
\begin{itemize}
	\item {mehr als 8 Sitzplätze}
	\item {zulässiges Gesamtgewicht von maximal 5 \ac{t}}
\end{itemize}

In der Klasse M2 sind Kraftfahrzeuge wie ein Eindecker-Bus bis 5 \ac{t} oder ein Doppeldecker-Bus bis 5 \ac{t} zu finden.

\subsubsection{Klasse M3}

Die dritte Unterklasse der Klasse M ist M3.

Kraftfahrzeuge der Klasse M3 haben über die Eigenschaften der Klasse M noch folgende weitere Eigenschaften:
\begin{itemize}
	\item {mehr als 8 Sitzplätze}
	\item {zulässiges Gesamtgewicht von über 5 \ac{t}}
\end{itemize}

In der Klasse M3 sind Kraftfahrzeuge wie ein Eindecker-Bus über 5 \ac{t} oder Doppeldecker-Bus über 5 \ac{t} zu finden.

\subsection{Klasse N}
In der Klasse N werden Kraftfahrerzeuge eingeordnet die für die Beförderung von Gütern zuständig sind und mindestens 3 Räder haben sowie ein zulässiges Gesamtgewicht von über 1 \ac{t} haben.
Die Klasse N spaltet sich in 3 Unterklassen auf:
\begin{itemize}
	\item {Klasse N1}
	\item {Klasse N2}
	\item {Klasse N3}
\end{itemize}

\subsubsection{Klasse N1}
Fahrzeuge zur Güterbeförderung mit einer zulässigen Gesamtmasse bis zu 3,5 \ac{t}.
In der Klasse N1 sind Kraftfahrzeuge die in dicht besiedelten Regionen gut zurecht kommen, wie Paketzusteller oder Fahrzeuge der Post.


\subsubsection{Klasse N2}
Fahrzeuge zur Güterbeförderung mit einer zulässigen Gesamtmasse von zu 3,5 \ac{t} bis 12 \ac*{t}.
In der Klasse N2 sind Kraftfahrzeuge die regional Güterbefördern, dies könnten Kraftfahrzeuge die Waren aus einem Zentrallager in die Filialen transportieren.
Diese Kraftfahrzeuge sind darauf ausgelegt hunderte Kilometer zurückzulegen.


\subsubsection{KLasse N3}
Fahrzeuge zur Güterbeförderung mit einer zulässigen Gesamtmasse von mehr als 12 \ac{t}.
In der Klasse N3 sind Kraftfahrzeuge die überregional Güterbefördern, wie ein Kraftfahrzeug das große Mengen an Ladung fassen kann und darauf ausgelegt sind tausende Kilometer zurückzulegen.

\subsection{Autonomes Fahren}
Beim autonomen Fahren, fährt ein Kraftfahrzeug Verwaltungsgemäß selbständig.
Für Kraftfahrzeuge wurden von der \ac{SAE} Institut in der Norm SAE J3016\footnote{SAE J3016\textunderscore202104 - https://www.sae.org/standards/content/j3016\textunderscore202104} Automatisierungsgrade definiert.
\begin{itemize}
	\item Stufe 0 (Keine Automation)
	\item Stufe 1 (Assistenzsysteme)
	\item Stufe 2 (Teilautomatisierung)
	\item Stufe 3 (Bedingte Automatisierung)
	\item Stufe 4 (Hochautomatisierung)
	\item Stufe 5 (Vollautomatisierung)
\end{itemize}
\subsubsection{Was passiert in den Stufen?}
Die Stufen unterscheiden sich im wesentlichen nur durch die Anzahl der Automatisierungsgrade.

\vspace{0.5cm}

In der Stufe 0 (Keine Automation):
\begin{itemize}
	\item keine Assistenzsysteme
	\item \ac{Kfz} kann keine Fahraufgaben übernehmen
	\item Fahrer ist unter permanenter Kontrolle
\end{itemize}

\vspace{0.5cm}

In der Stufe 1 (Assistenzsysteme):
\begin{itemize}
	\item Assistenzsysteme wie ein System zur automatischen Geschwindigkeitsregelung oder eine Berganfahrhilfe
	\item Fahrer hat eine passive Unterstützung bei Fahraufgaben
	\item \ac{Kfz} kann keine Fahraufgaben übernehmen
	\item Fahrer muss jeder Zeit die Fahraufgabe übernehmen können
\end{itemize}

\vspace{0.5cm}

In der Stufe 2 (Teilautomatisierung):
\begin{itemize}
	\item Assistenzsysteme, wie der Spurführungsassistent oder Stauassistent
	      \begin{itemize}
		      \item automatisch bremsen
		      \item automatisch beschleunigen
		      \item automatisch lenken
	      \end{itemize}
	\item \ac{Kfz} kann Fahraufgaben teilautomatisiert übernehmen
	\item Fahrer kann sich für kurze Zeit von den Fahraufgaben abwenden
	\item Fahrer muss jeder Zeit die Fahraufgabe übernehmen können
\end{itemize}

\vspace{0.5cm}

In der Stufe 3 (Bedingte Automatisierung):
\begin{itemize}
	\item hochautomatisierte Assistenzsysteme
	\item \ac{Kfz} kann Fahraufgaben unter bestimmten Voraussetzungen vollständig übernehmen
	\item Fahrer kann sich unter bestimmten Voraussetzungen dauerhaft von den Fahraufgaben abwenden
	\item Fahrer muss innerhalb wenigen Sekunden die Fahraufgabe übernehmen können
\end{itemize}

\vspace{0.5cm}

In der Stufe 4 (Hochautomatisierung):
\begin{itemize}
	\item hochautomatisierte Assistenzsysteme
	\item \ac{Kfz} kann Fahraufgaben in hochkomplexen Verkehrssituationen vollständig übernehmen
	\item Fahrer dauerhaft von den Fahraufgaben abwenden
	\item Fahrer muss fahrtüchtig sein, um im Bedarfsfall die Fahraufgabe übernehmen zu können
\end{itemize}

\vspace{0.5cm}

In der Stufe 5 (Vollautomatisierung):
\begin{itemize}
	\item hochautomatisierte Assistenzsysteme
	\item \ac{Kfz} übernimmt alle Fahraufgaben vollständig
	\item Fahrer ist nicht erforderlich
	\item alle Personen im Wagen werden zu Passagieren
\end{itemize}

\subsection{Autonome Kraftfahrzeuge}
Sind Fahrzeuge die nicht nur automatisch fahren sondern von einem System gesteuert werden.
Somit sind diese Fahrzeuge aus Sicht der Nutzenden autonom.

Während manche in der Verbreitung autonomer Fahrzeuge die Lösung vieler Probleme sehen können,
vermuten andere eine Verschlechterung der Verkehrs- und Umweltlage.

Die Bedeutung von autonomen Fahrzeugen, hängt sowohl von der technischen Komplexität sowie von politischen Regulierung ab.

In welchem Maß die Level 5 Systeme im Straßenverkehr teilnehmen entscheidet vorerst der gesetzliche Rahmen.
Dies ist wiederum abhängig wie der Verkehr von morgen aussehen soll.
\section{Umweltbelastungen durch Kraftfahrzeuge}
Kraftfahrzeuge belasten die Umwelt auf verschiedene Arten. Hierunter fallen
die Erzeugung von Rohstoffen für Materialien die für die Produktion von Kraftfahrzeugen benötigt werden,
die tatsächliche Produktion von Kraftfahrzeugen,
der Betrieb von Kraftfahrzeugen,
sowie die Entsorgung von Kraftfahrzeugen.

Gerade aber der Betrieb von Kraftfahrzeugen belastet die Umwelt durch die verschieden Arten von Schadstoffen.
Unterschieden wird durch die Art der Belastung,
giftige Verbrennungsabgase die durch die Verbrennung entstehen,
Feinstaub der sowohl durch die Verbrennung und auch durch den Abrieb von Reifen und Bremsen freigesetzt wird
und die Infrastruktur der Kraftfahrzeuge für Straßen, Parkplätze und andere Einrichtungen.

\subsection{Verbrennungsabgase}
Die größten Anteile der giftigen Schadstoffe die durch die Verbrennung von Kraftstoff entstehen sind:
\begin{itemize}
	\item {\ac{CO}}
	\item {\ac{NO}}
	\item unverbrannte Kohlenwasserstoffe (HC)
\end{itemize}
Die Abgase strömen nach der Verbrennung im Verbrennungsraum durch die Abgasanlage in die Umwelt.
Es gibt auch ungiftige Stoffe die durch die Verbrennung abgegeben werden wie \ac{zb} Wasser und \ac{CO2}.
Die Menge der Abgase die durch die Abgasanlage strömen ist von der Größe des Motors sowie dem Lastzustand des Motors abhängig.

\subsection{Feinstaub}
Feinstaub ist ein fester oder flüssiger Stoff der nicht sofort zu Boden sinkt. Bis der Feinstaub zu Boden sinkt, und we sich Feinstaub verbreitet ist neben der Art des Feinstaubes auch noch die Wetterlage entscheidend.
Bei Trockenheit kann sich Feinstaub gut ausbreiten und verweilt länger in der Luft, hohe Luftfeuchtigkeit beeinträchtigen die Ausbreitung.
Feinstäube werden als Particle Matter (PM, zu deutsch Stoffteilchen) bezeichnet. Diese Luftschadstoffe sind gesundheitsschädlich. \footnote{Westermann S. 327}

Es wird Unterschieden zwischen Feinstaub der aus natürlichen Quellen entstanden ist und Feinstaub der durch menschliches Handeln entstanden ist.

\subsubsection{Feinstaub aus natürlichen Quellen}
Natürlich Feinstäube sind ohne das Handeln durch den Menschen entstanden.
Quellen für natürlichen Feinstaub sind:
\begin{itemize}
	\item Vulkane
	\item Wald- und Buschbrände
	\item Pollen
	\item Sporen
\end{itemize}


\subsubsection{Feinstaub durch menschliches Handeln}
Feinstaub der durch menschliches Handeln entstanden ist wird auch anthropogener Feinstaub genannt.
Quellen für Feinstaub durch menschliches Handeln sind:
\begin{itemize}
	\item vom Straßenverkehr durch Verbrennung und Abrieb
	\item Verbrennungsabgase von Kraftwerken und Müllverbrennungsanlagen
	\item Brände von Gebäuden
	\item Industrieprozesse wie die Stahlerzeugung
\end{itemize}

Zur Verbesserung der Luftreinhaltung können Kommunen und Städte Umweltzonen einrichten und Fahrverbote festlegen.
Das befahren einer Umweltzone ist dann nur mit einer entsprechenden Kennzeichnung des Fahrzeuges möglich, die man bei der zuständigen Behörde erlangen kann.




\subsection{Infrastruktur}
Auch die Infrastruktur belastet die Umwelt, indem:
\begin{itemize}
	\item Parkplätze Flächen versiegeln
	\item Wälder abgeholzt werden um die Verkehrsanbindung zu verbessern
	\item Straßen vergrößert werden umd mehr Fahrzeuge zu ermöglichen
	\item starke Erhitzung durch Sonneneinstrahlung auf dunklen Verkehrswegen
	\item fehlende Bäume die Schatten spenden
\end{itemize}

\section{Umweltbelastung nach Bedingungen}
Die Umweltbelastung kann stark nach Betriebszuständen variieren.
So verbraucht ein Fahrzeuge das bergab fährt weniger Kraftstoff und stößt somit auch weniger Luftschadstoffe aus.
Die Umweltbelastung durch Luftschadstoffe hängt von folgendem ab:
\begin{itemize}
	\item dem Fahrverhalten des Fahrers, wie dem Beschleunigungsverhalten und der Fahrgeschwindigkeit ab
	\item der Effizienz des Fahrzeugs, je effizienter desto besser
	\item dem Gewicht des Fahrzeugs, je leichter desto weniger Gewicht muss beschleunigt und gebremst werden
	\item der Fahrstrecke, fährt das Fahrzeug eine Steigung wird mehr Kraftstoff benötigt
	\item dem Wetter, je nach Wind wird mehr oder weniger Kraftstoff benötigt
	\item dem Betriebszustand, wenn sich das Fahrzeug nicht im Betriebszustand befindet wird Energie verwendet um den Betriebszustand zu erreichen
\end{itemize}








\subsection{Aktueller Stand}
\subsection{Gesetzliche Regelungen}
Was ist bereits wo erlaubt?\\
Welche Länder haben was freigegeben?


\section{Erkenntnisse}
Hauptsächlich wird Sicherheit und die Senkung schädlicher Emissionen sowie das Erreichen der
Klimaschutzziele im Verkehr die größte Rolle spielen.

\subsection{Umwelt- und Klimaeffekte}
Die Umwelt- und Klimaeffekte durch autonomes Fahren ist durch die Vielzahl der Verknüpfungen mit anderen Technologien wie Elektromobilität
oder Dienstleistungen von Fahrdiensten und Car-Sharing-Angeboten im direkten Vergleich kaum noch bestimmbar.

Grundsätzlich werden Klimaeffekte abhängig von:
\begin{itemize}
	\item den eingesetzten Technologien
	\item den Kosten künftiger Mobilitätsdienstleister
	\item gesetzlichen Bestimmungen
	\item der Nachfrage von Nutzern
\end{itemize}

Im direkten Vergleich von autonomen Fahrzeugen zu konventionellen Fahrzeugen fällt ein geringer der Kraftstoffverbrauch und der geringere Ausstoß Feinstaub auf.

Andere Faktoren können die Umwelt entlasten wie der Abbau von Fahrzeugbeständen, und die dadurch wegfallenden negativen Einflüsse durch die Produktion von Fahrzeugen.

\subsection{Kraftstoffeinsparungen}
Kraftstoffeinsparungen durch autonomes Fahren im Straßenverkehr können sich durch
eine Steigerung der Effizienz im Verkehrsfluss und
einer abgestimmten Fahrweise bei einer optimalen Routenführung ergeben.

Erste Wirkungen können sich bereits im Mischverkehr aus autonom und konventionell gesteuerten Kraftfahrzeugen bemerkbar machen durch weniger
Brems- und Beschleunigungsvorgängen.

Die Wirkungen könnten sich mit einer steigenden Marktdurchdringung von autonomen Fahrzeugen
durch sinkenden stockenden Verkehr sowie eine Reduzierung von Staus bemerkbar machen.

\subsubsection{Kraftstoffeinsparungen auf Autobahnen}
Das Fraunhofer Institut für Arbeitswirtschaft und Organisation (IAO) hat erste Schätzungen zu Kraftstoffeinsparungen auf Autobahnen im Individualverkehr in Deutschland vorgenommen.

Dort wurde die Wirkung von drei bereits existierenden Assistenzsystemen (Stau-Chauffeur, Spurwechsel-Chauffeur, Autobahn-Chauffeur)
der Automatisierungsstufe 3 auf Autobahnen begutachtet.

Die Wirkung wurde für zwei verschiedene Szenarien ausgewiesen.
Das erste Szenario basiert darauf dass alle Kraftfahrzeuge autonom fahren.
Im zweiten Szenario betrug der Anteil ungefähr 45.000 autonome Fahrzeuge.

Auf der Basis von einer 10-20 prozentigen Reduzierung des Kraftstoffverbrauchs ergab sich
im ersten Szenario ein jährliches Sparpotential zwischen 360 und 720 Millionen Euro und
im zweiten Szenario von 0,4 bis 0,8 Millionen Euro.

Rechnet man diese Werte auf einzelne autonome Kraftfahrzeuge um,
ergäbe dies eine Einsparung für jedes autonome Kraftfahrzeug von 8 bis 16 € pro Jahr.
\footnote{Fraunhofer-Institut für Arbeitswirtschaft und Organisation IAO, Hochautomatisiertes
	Fahren auf Autobahnen – Industriepolitische Schlussfolgerungen, Studie im Auftrag des
	Bundesministeriums für Wirtschaft und Energie, November 2015, S. 264ff.}

Eine weitere Einsparung auf Autobahnen durch autonome Fahrzeuge könnte mit
dem systembedingten verzicht der Überschreitung von Fahrgeschwindigkeiten über der Richtgeschwindigkeit.
Der Verzicht von höheren Geschwindigkeiten auf Autobahnen kann zu enormen Einsparungen im Kraftstoffverbrauch führen,
da mit steigender Geschwindigkeit der Kraftstoffverbrauch überproportional rasant ansteigt.

\subsubsection{Kraftstoffeinsparungen im Individualverkehr im städtischen Verkehr}
Zu den Kraftstoffeinsparungen auf Autobahnen werden von
autonomen Kraftfahrzeugen besonders im städtischen Verkehr
eine deutliche Reduzierung im Verbrauch von Kraftstoffen erwartet.

Große Einsparpotentiale bieten hierfür:
\begin{itemize}
	\item die Steigerung des Verkehrsflusses insbesondere an Knotenpunkten
	\item Verstetigung der Geschwindigkeit
	\item die Vermeidung von Verkehrsstörungen
\end{itemize}

Wie die Auswertung internationaler Studien
\footnote{Milakis, D., van Arem, B., van Wee, B., Policy and society related implications of
	automated driving: A review of literature and directions for future research, Journal of
	Intelligent Transportation Systems, 2017, Vol. 21, No. 4, 335f} gezeigt hatsss,
können Einsparungen beim Kraftstoffverbrauch von
bis zu 31 Prozent im städtischen Bereich sowie
bis zu 45 Prozent bei einer optimierten Knotenpunktsteuerung erreicht werden.
Diese Abschätzungen basieren auf Simulations- und Modellrechnungen.
Die Menge der  Kraftstoffeinsparungen hängt unter anderem von mehreren Faktoren ab:
\begin{itemize}
	\item die verwendete Automatisierungsstufe
	\item der Marktanteil von autonomen Fahrzeugen
	\item die Vernetzung zwischen den Fahrzeugen
	\item die Vernetzung zwischen den Fahrzeugen und der Infrastruktur
\end{itemize}

\subsubsection{Kraftstoffeinsparungen bei Flottenfahrzeugen}
Bei Fahrzeugen die in einer Flotte betrieben werden, ist eine intensivere Nutzung der autonomen Fahrzeuge zu erwarten, wodurch sich der Nutzungszeitraum verkürzt.
Dadurch werden Flottenfahrzeuge früher ausgetauscht und erneuert.
Durch den schnellen Wechsel der Flottenfahrzeuge werden können modernere Fahrzeuge mit technologischen Erneuerung
die weniger Luftschadstoffe und Emissionen produzieren schneller in den Einsatz gebracht werden.
Dies könnte im dicht besiedelten Gebieten zu einer Reduktion von Verkehrsbedingen Luftschadstoffen und Feinstaub führen.

\subsection{Positive Umwelteinwirkung durch die Reduktion von Fahrzeugen}
Positive Umwelteinwirkung durch autonomes Fahren kann durch eine
Reduktion am Fahrzeugbestand erwartet werden.

Mobilitätsdienstleister mit autonomen Taxen können
Kunden abholen und zu Zielort bringen,
und vermindern die Notwendigkeit zum Kauf eines Fahrzeugs.

Verschiedene Studien haben in verschiedenen Szenarien berechnet, ob und wie viele Kraftfahrzeuge sich durch autonome Fahrzeuge zusammenfassen ließen.
Eine Studie kam zu dem Resultat, dass in etwa 18.000 Flottenfahrzeuge den kompletten Individualverkehr von München innerhalb eines Einzuggebietes decken könnte.
Dadurch könnte auf ungefähr 200 Tausend Fahrzeuge verzichtet werden.
\footnote{Alternative Antriebe, Autonomes Fahren,
	Mobilitätsdienstleistungen: Neue Infrastrukturen für
	die Verkehrswende im Automobilsektor, S. 42}

\subsection{Beschreibung der verwendeten Untersuchungsmittel}

\subsection{Diskussion der Forschungsfrage}

\subsection{Gründe für die Wahl der Hypothese}

\subsection{Darstellung und Diskussion der Erkenntnisse aus der Literatur}

\subsection{Schlussfolgerungen auf die Forschungsfrage}

\subsection{Verifikation der Hypothese}

\subsection{Praktische Konsequenzen}
