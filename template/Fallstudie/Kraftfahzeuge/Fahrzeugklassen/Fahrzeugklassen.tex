\section{Fahrzeugklassen}

Kraftfahrzeuge können Bauartbedingt in Kategorien eingeordnet werden.
Die EU Kommission hat hierfür acht Klassen definiert.\footnote{VERORDNUNG (EU) Nr. 678/2011 DER KOMMISSION
	vom 14. Juli 2011, TEIL A ABS.1 - https://eur-lex.europa.eu/eli/reg/2011/678/oj?locale=de}

\begin{itemize}
	\item Klasse L: Leichte ein- und zweispurige Kraftfahrzeuge
	\item Klasse M: Vorwiegend für die Beförderung von Fahrgästen und deren Gepäck ausgelegte und gebaute Kraftfahrzeuge
	\item Klasse N: Vorwiegend für die Beförderung von Gütern ausgelegte und gebaute Kraftfahrzeuge
	\item Klasse O: Anhänger, die sowohl für die Beförderung von Gütern und Fahrgästen als auch für die Unterbringung von Personen ausgelegt und gebaut sind
	\item Klasse S: unvollständige Fahrzeuge, die der Unterklasse der Fahrzeuge mit besonderer Zweckbestimmung zugeordnet werden soll
	\item Klasse R: Anhänger, die in der Land- und Forstwirtschaft verwendet werden
	\item Klasse S: Maschinen, die in der Land- und Forstwirtschaft zum Einsatz kommen und gezogen werden
	\item Klasse T: Zugmaschinen, die in der Land- und Forstwirtschaft verwendet werden wie Traktoren
	\item Klasse C: Zugmaschinen, die in der Land- und Forstwirtschaft verwendet werden und auf Ketten laufen wie ein Bagger
\end{itemize}

Die relevantesten Klassen sind M und N.
\vspace{0.5cm}

\subsection{Klasse M}
In der Klasse M werden Kraftfahrerzeuge eingeordnet die für die Beförderung von Personen und Gepäck zuständig sind und mindestens 4 Räder haben sowie eine Hochgeschwindigkeit von über 25 \ac{kmh} haben.
\newline
Die Klasse M spaltet sich in 3 Unterklassen auf:
\begin{itemize}
	\item {Klasse M1}
	\item {Klasse M2}
	\item {Klasse M3}
\end{itemize}
\subsubsection{Klasse M1}
Kraftfahrzeuge der Klasse M1 haben über die Eigenschaften der Klasse M noch folgende weitere Eigenschaften:
\begin{itemize}
	\item {nicht mehr als 8 Sitzplätze und 1 Platz für den Fahrer}
	\item {keine Stehplätze}
	\item {zulässiges Gesamtgewicht von maximal 3,5 \ac{t}}
\end{itemize}

In der Klasse M1 sind Kraftfahrzeuge wie Personenkraftwagen(Limousine, Cabrio) und Wohnmobile zu finden.

\subsubsection{Klasse M2}
Kraftfahrzeuge der Klasse M2 haben über die Eigenschaften der Klasse M noch folgende weitere Eigenschaften:
\begin{itemize}
	\item {mehr als 8 Sitzplätze}
	\item {zulässiges Gesamtgewicht von maximal 5 \ac{t}}
\end{itemize}

In der Klasse M2 sind Kraftfahrzeuge wie ein Eindecker-Bus bis 5 \ac{t} oder ein Doppeldecker-Bus bis 5 \ac{t} zu finden.

\subsubsection{Klasse M3}

Die dritte Unterklasse der Klasse M ist M3.

Kraftfahrzeuge der Klasse M3 haben über die Eigenschaften der Klasse M noch folgende weitere Eigenschaften:
\begin{itemize}
	\item {mehr als 8 Sitzplätze}
	\item {zulässiges Gesamtgewicht von über 5 \ac{t}}
\end{itemize}

In der Klasse M3 sind Kraftfahrzeuge wie ein Eindecker-Bus über 5 \ac{t} oder Doppeldecker-Bus über 5 \ac{t} zu finden.

\subsection{Klasse N}
In der Klasse N werden Kraftfahrerzeuge eingeordnet die für die Beförderung von Gütern zuständig sind und mindestens 3 Räder haben sowie ein zulässiges Gesamtgewicht von über 1 \ac{t} haben.
Die Klasse N spaltet sich in 3 Unterklassen auf:
\begin{itemize}
	\item {Klasse N1}
	\item {Klasse N2}
	\item {Klasse N3}
\end{itemize}

\subsubsection{Klasse N1}
Fahrzeuge zur Güterbeförderung mit einer zulässigen Gesamtmasse bis zu 3,5 \ac{t}.
In der Klasse N1 sind Kraftfahrzeuge die in dicht besiedelten Regionen gut zurecht kommen, wie Paketzusteller oder Fahrzeuge der Post.


\subsubsection{Klasse N2}
Fahrzeuge zur Güterbeförderung mit einer zulässigen Gesamtmasse von zu 3,5 \ac{t} bis 12 \ac*{t}.
In der Klasse N2 sind Kraftfahrzeuge die regional Güterbefördern, dies könnten Kraftfahrzeuge die Waren aus einem Zentrallager in die Filialen transportieren.
Diese Kraftfahrzeuge sind darauf ausgelegt hunderte Kilometer zurückzulegen.


\subsubsection{KLasse N3}
Fahrzeuge zur Güterbeförderung mit einer zulässigen Gesamtmasse von mehr als 12 \ac{t}.
In der Klasse N3 sind Kraftfahrzeuge die überregional Güterbefördern, wie ein Kraftfahrzeug das große Mengen an Ladung fassen kann und darauf ausgelegt sind tausende Kilometer zurückzulegen.
