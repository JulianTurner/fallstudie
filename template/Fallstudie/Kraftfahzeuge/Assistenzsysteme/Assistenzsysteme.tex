\subsection{Autonomes Fahren}
Beim autonomen Fahren, fährt ein Kraftfahrzeug Verwaltungsgemäß selbständig.
Für Kraftfahrzeuge wurden von der \ac{SAE} Automatisierungsgrade definiert\cite{PRACTICE}.
\begin{itemize}
	\item Stufe 0 (Keine Automation)
	\item Stufe 1 (Assistenzsysteme)
	\item Stufe 2 (Teilautomatisierung)
	\item Stufe 3 (Bedingte Automatisierung)
	\item Stufe 4 (Hochautomatisierung)
	\item Stufe 5 (Vollautomatisierung)
\end{itemize}
\subsubsection{Was passiert in den Stufen?}
Die Stufen unterscheiden sich im wesentlichen nur durch die Anzahl der Automatisierungsgrade.

\vspace{0.5cm}

In der Stufe 0 (Keine Automation):
\begin{itemize}
	\item keine Assistenzsysteme
	\item Kraftfahrzeug kann keine Fahraufgaben übernehmen
	\item Fahrer kontrolliert permanent das Fahrzeug
\end{itemize}

\vspace{0.5cm}

In der Stufe 1 (Assistenzsysteme):
\begin{itemize}
	\item Assistenzsysteme wie zum Beispiel ein System zur automatischen Geschwindigkeitsregelung oder eine Berganfahrhilfe
	\item Fahrer hat eine passive Unterstützung bei Fahraufgaben
	\item Kraftfahrzeug kann keine Fahraufgaben übernehmen
	\item Fahrer kontrolliert permanent das Fahrzeug
\end{itemize}

\vspace{0.5cm}

In der Stufe 2 (Teilautomatisierung):
\begin{itemize}
	\item Assistenzsysteme, wie zum Beispiel der Spurführungsassistent oder Stauassistent
	      \begin{itemize}
		      \item automatisch bremsen
		      \item automatisch beschleunigen
		      \item automatisch lenken
	      \end{itemize}
	\item Kraftfahrzeug kann Fahraufgaben teilautomatisiert übernehmen
	\item Fahrer kann sich für kurze Zeit von den Fahraufgaben abwenden
	\item Fahrer muss jeder Zeit die teilautomatisierte Fahraufgabe übernehmen können
\end{itemize}

\vspace{0.5cm}

In der Stufe 3 (Bedingte Automatisierung):
\begin{itemize}
	\item hochautomatisierte Assistenzsysteme
	\item Kraftfahrzeug kann Fahraufgaben unter bestimmten Voraussetzungen vollständig übernehmen
	\item Fahrer kann sich unter bestimmten Voraussetzungen von den Fahraufgaben abwenden
	\item Fahrer muss innerhalb von wenigen Sekunden die Fahraufgabe übernehmen können
\end{itemize}

\vspace{0.5cm}

In der Stufe 4 (Hochautomatisierung):
\begin{itemize}
	\item hochautomatisierte Assistenzsysteme
	\item Kraftfahrzeug kann Fahraufgaben in hochkomplexen Verkehrssituationen vollständig übernehmen
	\item Fahrer kann sich von den Fahraufgaben abwenden
	\item Fahrer muss fahrtüchtig sein, um im Bedarfsfall die Fahraufgabe übernehmen zu können
\end{itemize}

\vspace{0.5cm}

In der Stufe 5 (Vollautomatisierung):
\begin{itemize}
	\item hochautomatisierte Assistenzsysteme
	\item Kraftfahrzeug übernimmt alle Fahraufgaben vollständig
	\item Fahrer ist nicht erforderlich
	\item alle Personen im Wagen werden zu Passagieren
\end{itemize}

\subsection{Autonome Kraftfahrzeuge}
Autonome Kraftfahrzeuge sind Fahrzeuge die nicht nur automatisch fahren sondern von einem System gesteuert werden.
Somit sind diese Fahrzeuge aus Sicht der Nutzenden autonom.

Während manche in der Verbreitung autonomer Fahrzeuge die Lösung vieler Probleme sehen können,
vermuten andere eine Verschlechterung der Verkehrs- und Umweltlage.

Die Bedeutung von autonomen Fahrzeugen hängt sowohl von der technischen Komplexität als auch von den politischen Regulierung ab.

In welchem Maß das Level 5 System im Straßenverkehr teilnimmt entscheidet vorerst der gesetzliche Rahmen.
Dies ist wiederum abhängig wie der zukünftige Verkehr aussehen soll.