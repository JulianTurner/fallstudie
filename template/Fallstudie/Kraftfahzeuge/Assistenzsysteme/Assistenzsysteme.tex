\section{Autonomes Fahren}
Beim autonomen Fahren, fährt ein \ac{Kfz} Verwaltungsgefäß selbständig.
Für \ac{Kfz} wurden von der \ac{SAE} Institut in der Norm SAE J3016\footnote{SAE J3016\textunderscore202104 - https://www.sae.org/standards/content/j3016\textunderscore202104} Automatisierungsgrade definiert.
\begin{itemize}
	\item Stufe 0 (Keine Automation)
	\item Stufe 1 (Assistenzsysteme)
	\item Stufe 2 (Teilautomatisierung)
	\item Stufe 3 (Bedingte Automatisierung)
	\item Stufe 4 (Hochautomatisierung)
	\item Stufe 5 (Vollautomatisierung)
\end{itemize}
\subsubsection{Was passiert in den Stufen?}
Die Stufen unterscheiden sich im wesentlichen nur durch die Anzahl der Automatisierungsgrade.

\vspace{0.5cm}

In der Stufe 0 (Keine Automation):
\begin{itemize}
	\item keine Assistenzsysteme
	\item \ac{Kfz} kann keine Fahraufgaben übernehmen
	\item Fahrer ist unter permanenter Kontrolle
\end{itemize}

\vspace{0.5cm}

In der Stufe 1 (Assistenzsysteme):
\begin{itemize}
	\item Assistenzsysteme wie \ac{GRA} oder eine Berganfahrhilfe
	\item Fahrer hat eine passive Unterstützung bei Fahraufgaben
	\item \ac{Kfz} kann keine Fahraufgaben übernehmen
	\item das \ac{Kfz} ist unter permanenter Kontrolle des Fahrers
\end{itemize}

\vspace{0.5cm}

In der Stufe 2 (Teilautomatisierung):
\begin{itemize}
	\item Assistenzsysteme, wie der Spurführungsassistent oder Stauassistent
	      \begin{itemize}
		      \item automatisch bremsen
		      \item automatisch beschleunigen
		      \item automatisch lenken
	      \end{itemize}
	\item \ac{Kfz} kann Fahraufgaben teilautomatisiert übernehmen
	\item Fahrer kann sich für kurze Zeit von den Fahraufgaben abwenden
	\item Fahrer muss jeder Zeit die Fahraufgabe übernehmen können
\end{itemize}

\vspace{0.5cm}

In der Stufe 3 (Bedingte Automatisierung):
\begin{itemize}
	\item hochautomatisierte Assistenzsysteme
	\item \ac{Kfz} kann Fahraufgaben unter bestimmten Voraussetzungen vollständig übernehmen
	\item Fahrer kann sich unter bestimmten Voraussetzungen dauerhaft von den Fahraufgaben abwenden
	\item Fahrer muss innerhalb wenigen Sekunden die Fahraufgabe übernehmen können
\end{itemize}

\vspace{0.5cm}

In der Stufe 4 (Hochautomatisierung):
\begin{itemize}
	\item hochautomatisierte Assistenzsysteme
	\item \ac{Kfz} kann Fahraufgaben in hochkomplexen Verkehrssituationen vollständig übernehmen
	\item Fahrer dauerhaft von den Fahraufgaben abwenden
	\item Fahrer muss fahrtüchtig sein, um im Bedarfsfall die Fahraufgabe übernehmen zu können
\end{itemize}

\vspace{0.5cm}

In der Stufe 5 (Vollautomatisierung):
\begin{itemize}
	\item hochautomatisierte Assistenzsysteme
	\item \ac{Kfz} übernimmt alle Fahraufgaben vollständig
	\item Fahrer ist nicht erforderlich
	\item alle Personen im Wagen werden zu Passagieren
\end{itemize}

\subsection{Autonome Kraftfahrzeuge}
Sind Fahrzeuge die nicht nur automatisch fahren sondern von einem System gesteuert werden.
Somit sind diese Fahrzeuge aus Sicht der Nutzenden autonom.

Während manche in der Verbreitung autonomer Fahrzeuge die Lösung vieler Probleme sehen können,
vermuten andere eine Verschlechterung der Verkehrs- und Umweltlage.

Die Bedeutung von autonomen Fahrzeugen, hängt sowohl von der technischen Komplexität sowie von politischen Regulierung ab.

In welchem Maß die Level 5 Systeme im Straßenverkehr teilnehmen entscheidet vorerst der gesetzliche Rahmen.
Dies ist wiederum abhängig wie der Verkehr von morgen aussehen soll.