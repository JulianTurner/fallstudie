\section{Umweltbelastungen durch Kraftfahrzeuge}
Kraftfahrzeuge belasten die Umwelt auf verschiedene Arten.
Hierunter fallen
die Erzeugung von Rohstoffen für Materialien die für die Produktion von Kraftfahrzeugen benötigt werden,
die tatsächliche Produktion von Kraftfahrzeugen,
der Betrieb von Kraftfahrzeugen,
sowie die Entsorgung von Kraftfahrzeugen.

Gerade der Betrieb von Kraftfahrzeugen belastet die Umwelt durch die verschieden Arten von Schadstoffen.
Unterschieden wird die Art der Belastung,
durch die Verbrennung entstandene Abgase,
Feinstaub der durch die Verbrennung, sowohl auch durch den Abrieb von Reifen und Bremsen freigesetzt wird
und die Infrastruktur der Straßen, Parkplätze und anderer Einrichtungen.

\subsection{Verbrennungsabgase}
Durch Verbrennung von Kraftstoffen entstehen verschied giftige Schadstoffe:
\begin{itemize}
	\item {\ac{CO}}
	\item {\ac{NO}}
	\item unverbrannte Kohlenwasserstoffe (HC)
\end{itemize}
Die Abgase strömen nach der Verbrennung im Verbrennungsraum durch die Abgasanlage in die Umwelt.
Es gibt auch ungiftige Stoffe die durch die Verbrennung abgegeben werden wie \ac{zb} \ac{Wasser} und \ac{CO2}.
Die Menge der Abgase die durch die Abgasanlage strömen ist von der Größe des Motors sowie dem Lastzustand des Motors abhängig.

\subsection{Feinstaub}
Feinstaub ist ein fester oder flüssiger Stoff der nicht sofort zu Boden sinkt.
Neben der Art des Feinstaubes ist unter anderem die Wetterlage für die Verbreitung und Absenkung von Feinstaub verantwortlich.

Bei Trockenheit kann sich Feinstaub gut ausbreiten und verweilt länger in der Luft, hohe Luftfeuchtigkeit beeinträchtigen die Ausbreitung.
Feinstäube werden als Particle Matter (PM, zu deutsch Stoffteilchen) bezeichnet. Diese Luftschadstoffe sind gesundheitsschädlich. \footnote{Westermann S. 327}

Es wird Unterschieden zwischen Feinstaub der aus natürlichen Quellen entstanden ist und Feinstaub der durch menschliches Handeln entstanden ist.

\subsubsection{Feinstaub aus natürlichen Quellen}
Natürlich Feinstäube sind ohne das Handeln durch den Menschen entstanden.
Quellen für natürlichen Feinstaub sind:
\begin{itemize}
	\item Vulkane
	\item Wald- und Buschbrände
	\item Pollen
	\item Sporen
\end{itemize}


\subsubsection{Feinstaub durch menschliches Handeln}
Feinstaub der durch menschliches Handeln entstanden ist wird auch anthropogener Feinstaub genannt.
Quellen für Feinstaub durch menschliches Handeln sind:
\begin{itemize}
	\item vom Straßenverkehr durch Verbrennung und Abrieb
	\item Verbrennungsabgase von Kraftwerken und Müllverbrennungsanlagen
	\item Brände von Gebäuden
	\item Industrieprozesse wie die Stahlerzeugung
\end{itemize}

Zur Verbesserung der Luftreinhaltung können Kommunen und Städte Umweltzonen einrichten und Fahrverbote festlegen.
Das befahren einer Umweltzone ist dann nur mit einer entsprechenden Kennzeichnung des Fahrzeuges möglich, die man bei der zuständigen Behörde erlangen kann.




\subsection{Infrastruktur}
Auch die Infrastruktur belastet die Umwelt, indem:
\begin{itemize}
	\item für Parkplätze werden Flächen versiegeln
	\item Wälder abgeholzt werden um die Verkehrsanbindung zu verbessern
	\item Straßen vergrößert werden umd mehr Fahrzeuge zu ermöglichen
	\item starke Erhitzung durch Sonneneinstrahlung auf dunklen Verkehrswegen
	\item fehlende Bäume die Schatten spenden
\end{itemize}